
\chapter{Statistical Coherence Foundations} \label{chapter-1-statistical-coherence-foundations}

SCM Volume I established identity as the return-closed resolution of symbolic structure. It defined coherence signatures, collapse thresholds, and RuleEvolution dynamics. However, the symbolic model introduced there was fundamentally deterministic.

In Volume II, we now introduce the statistical and thermodynamic interpretation of coherence behavior. This chapter establishes the foundations of:

\begin{itemize}
    \item Symbolic entropy,
    \item Resolution multiplicity, and
    \item Coherence temperature,
\end{itemize}

which allow us to treat return behavior probabilistically. These concepts serve as the basis for effort weighting, symbolic drift, and return-driven thermodynamics.

We begin by defining the entropy of return structures. We then formalize the notion of symbolic temperature as a drift potential across $\chi$-space. These structures will ultimately support a statistical version of RuleEvolution and provide the internal mechanics for mass, curvature, and collapse.

\medskip

As established in Volume I, coherent inversion structures that return to a structural inverse ($X_\pi = -1$) but resolve to $[0]$ belong to $\Omega_-$.

This volume extends that insight to examine how $\Omega_-$ influences symbolic effort, entropy flow, and system evolution.

\section{Return Entropy} \label{return-entropy}

Let $[A] \in \Omega_3$ be a resolved identity. Let $\mathcal{R}([A])$ denote the set of all permitted return loops through $[A]$ under the current Rule Set $R$.

\begin{definition}[Return Entropy] \label{def:return-entropy}
The return entropy of $[A]$ is defined as:
\begin{equation} \label{eq:return-entropy}
H([A]) := -\sum_{\mathcal{L}_i \in \mathcal{R}([A])} p_i \log p_i
\end{equation}
where
\begin{equation} \label{eq:return-entropy-probs}
p_i := \frac{1}{Z} \cdot \exp(-\Lambda(\mathcal{L}_i)), \quad
Z := \sum_{\mathcal{L}_j \in \mathcal{R}([A])} \exp(-\Lambda(\mathcal{L}_j))
\end{equation}
Here, $\Lambda(\mathcal{L}_i)$ denotes the latency of loop $\mathcal{L}_i$, i.e., the total effort required to complete that return.
\end{definition}

\paragraph{Remark (Exclusion of $\Omega_-$ Paths).}
Return loops that resolve to $[-A]$ via symmetry inversion ($X_\pi = -1$), such that $[A] + [-A] = [0]$, are excluded from $\mathcal{R}([A])$. These paths define structural cancellation, not identity persistence, and belong to $\Omega_-$.

\paragraph{Interpretation}
\begin{itemize}
    \item $H([A])$ quantifies the distributional uncertainty over return loops.
    \item High entropy implies many return options with comparable latency.
    \item Low entropy indicates one or few energetically preferred return paths.
\end{itemize}
This entropy plays the role of symbolic freedom and becomes a thermodynamic quantity in later chapters.

\section{Latency Distribution and Return Potential} \label{latency-distribution-and-return-potential}

Each return loop $\mathcal{L}_i \in \mathcal{R}([A])$ contributes to the overall return behavior of $[A]$.

\begin{definition}[Return Potential] \label{def:return-potential}
The return potential $\Phi([A])$ is defined as the expected latency:
\begin{equation} \label{eq:return-potential}
\Phi([A]) := \sum_{\mathcal{L}_i \in \mathcal{R}([A])} p_i \cdot \Lambda(\mathcal{L}_i)
\end{equation}
This value plays the role of a symbolic free energy.
\end{definition}

\paragraph{Properties.}
\begin{itemize}
    \item $\Phi([A])$ is minimized when the return path distribution is sharply peaked (i.e., low $H([A])$).
    \item $\Phi([A])$ increases as return diversity grows or loop latencies become large.
\end{itemize}

$\Phi([A])$ will later be used in defining symbolic collapse regions and coherence gradients.

\section{Symbolic Temperature} \label{symbolic-temperature}

We now define the symbolic coherence temperature $T_s$ as a measure of drift susceptibility:

\begin{equation} \label{eq:symbolic-temperature}
T_s([A]) := \frac{d\Phi}{dH}
\end{equation}

That is, $T_s$ captures how much symbolic potential ($\Phi$) increases per unit increase in entropy ($H$). This serves as a local measure of coherence disorder.

\paragraph{Interpretation.}
\begin{itemize}
    \item $T_s \approx 0$ implies $[A]$ is coherence-stable (low entropy sensitivity).
    \item $T_s \gg 0$ implies $[A]$ is drift-prone (potential rises sharply with entropy).
\end{itemize}

This temperature will later be used to define symbolic force, identity pressure, and thermal collapse behavior.

\section{Resolution Capacity and Reuse Load} \label{resolution-capacity-and-reuse-load}

Let $\deg([A])$ denote the number of identities that reuse $[A]$ in their return resolution.

\begin{definition}[Resolution Capacity and Reuse Load]
\begin{itemize}
    \item \textbf{Resolution Capacity:} 
    \[
    C([A]) := \text{maximum number of return paths } [A] \text{ can participate in before } \partial\chi \text{ diverges}.
    \]
    \item \textbf{Reuse Load:}
    \[
    L([A]) := \text{current number of identities reusing } [A].
    \]
\end{itemize}
\end{definition}

\begin{definition}[Coherence Pressure] \label{def:coherence-pressure}
The coherence pressure is defined as:
\begin{equation} \label{eq:coherence-pressure}
P([A]) := \frac{L([A])}{C([A])}
\end{equation}
\end{definition}

When $P([A]) \to 1$, the identity $[A]$ approaches reuse saturation and may trigger RuleEvolution or collapse.

\section{Structural Summary} \label{structural-summary}

We have now introduced four statistical quantities that define the coherence state of a resolved identity:

\begin{table}[h!]
\centering
\begin{tabular}{|l|c|l|}
\hline
\textbf{Quantity} & \textbf{Symbol} & \textbf{Interpretation} \\
\hline
Return Entropy         & $H([A])$      & Diversity of resolution paths \\
Return Potential       & $\Phi([A])$   & Expected return effort         \\
Coherence Temperature  & $T_s([A])$    & Sensitivity of $\Phi$ to entropy \\
Coherence Pressure     & $P([A])$      & Reuse load relative to capacity \\
\hline
\end{tabular}
\caption{Statistical descriptors of identity $[A]$}
\end{table}

These define the statistical state of a resolved identity $[A]$ and form the foundation for effort-weighted RuleEvolution and symbolic thermodynamics in later chapters.

\medskip

The domain $\Omega_-$, introduced in Volume I, contains structures that resolve coherently to $[0]$ via inversion loops. These forms raise $\Phi$ and symbolic entropy globally but do not contribute to $H([A])$ for identity-based ensembles.
