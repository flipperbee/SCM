\chapter{Drift Fields and Collapse Zones in $\chi$-space} \label{chapter-4-drift-fields-and-collapse-zones-in-chi-space}

Coherence signatures ($\chi$) classify identities in $\Omega_3$, but signatures alone do not capture how identities evolve under reuse or symbolic stress. In real systems, coherence behavior is not static: reuse patterns shift, rules evolve, and coherence signatures change in response.

\medskip

This chapter introduces the drift vector $\partial\chi$ over $\chi$-space, which measures the local change in identity coherence under RuleEvolution. It also defines collapse zones: regions where structure becomes unstable due to fragility ($X_\phi$) or reuse overload ($X_\epsilon$).

\medskip

These concepts allow us to describe symbolic identity dynamics as a flow over $\chi$-space and provide the geometric groundwork for coherence thermodynamics.

\section{The Drift Vector $\partial\chi$} \label{the-drift-vector-chi}

Let $[A] \in \Omega_3$, and let $\chi_t([A])$ be the coherence signature of $[A]$ at symbolic time $t$.

\begin{definition}[Drift Vector] \label{def:drift-vector}
The drift vector of $[A]$ is defined as:
\begin{equation} \label{eq:drift-vector}
\partial\chi([A]) := \chi_{t+1}([A]) - \chi_t([A])
\end{equation}
\end{definition}

Each component of the signature evolves independently:
\begin{itemize}
    \item $\partial X_h$: change in minimal return depth,
    \item $\partial X_c$: change in coherence strength,
    \item $\partial X_\pi$: change in return symmetry,
    \item $\partial X_\phi$: change in fragility,
    \item $\partial X_\epsilon$: change in reuse elevation.
\end{itemize}

\section{Component-Wise Stability Classification} \label{component-wise-stability-classification}

To respect the non-metric nature of $\chi$-space, we define stability not through a norm, but via component-wise drift conditions.

\begin{definition}[Stability Classification]
Let $\partial X_i$ denote the drift in the $i$-th component of $\chi([A])$, where $i \in \{X_h, X_c, X_\pi, X_\phi, X_\epsilon\}$. Then:

\begin{itemize}
    \item \textbf{Stable:} $\partial X_i = 0$ for all $i$,
    \item \textbf{Collapsing:} $|\partial X_i| \to \infty$ for at least one $i$,
    \item \textbf{Reactive:} all other cases (bounded nonzero drift in at least one component).
\end{itemize}
\end{definition}

This preserves consistency with the heterogeneity of $\chi$ components and avoids imposing an artificial norm.

\section{Drift Fields over $\chi$-space} \label{drift-fields-over-chi-space}

Let $\chi$-space be the 5-dimensional product space of coherence signatures.

\begin{definition}[Drift Field]
The drift field is defined as the component-wise symbolic vector field:
\begin{equation} \label{eq:drift-field}
\nabla\chi : \Omega_3 \rightarrow \text{component-wise symbolic drift}
\end{equation}
\end{definition}

Each identity $[A] \in \Omega_3$ has an associated local drift vector $\partial\chi([A])$.

\paragraph{Flow Patterns.}
\begin{itemize}
    \item \textbf{Stable regions:} $\partial\chi = 0$
    \item \textbf{Collapsing regions:} $\partial X_i \to \infty$ for some $i$
    \item \textbf{Reactive regions:} bounded $\partial X_i \ne 0$ for at least one $i$
\end{itemize}

\section{Collapse Zones} \label{collapse-zones}

Let $\rho([A])$ denote the collapse robustness of identity $[A] \in \Omega_3$, and let $\rho_c$ be a critical robustness threshold.

\begin{definition}[Collapse Zone]
The collapse zone is the set of identities with insufficient robustness:
\begin{equation} \label{eq:collapse-zone}
\mathcal{Z}_{\text{collapse}} := \{ [A] \in \Omega_3 \mid \rho([A]) < \rho_c \}
\end{equation}
\end{definition}

Identities in this region exhibit one or more of the following:
\begin{itemize}
    \item High fragility ($X_\phi \gg 1$),
    \item Drift-prone behavior (some $\partial X_i$ diverges),
    \item Susceptibility to removal by RuleEvolution.
\end{itemize}

\section{Reuse Overload and Elevation Collapse} \label{reuse-overload-and-elevation-collapse}

Let $L([A])$ denote the number of identities reusing $[A]$, and $C([A])$ its resolution capacity.

\begin{definition}[Coherence Pressure]
The coherence pressure of $[A]$ is defined as:
\begin{equation} \label{eq:coherence-pressure-reuse}
P([A]) := \frac{L([A])}{C([A])}
\end{equation}
\end{definition}

\begin{definition}[Overload Zone]
The reuse overload zone is defined as:
\begin{equation} \label{eq:overload-zone}
\mathcal{Z}_{\text{overload}} := \{ [A] \in \Omega_3 \mid P([A]) > P_{\text{crit}} \}
\end{equation}
\end{definition}

Identities within $\mathcal{Z}_{\text{overload}}$ experience:
\begin{itemize}
    \item Drift in reuse elevation $X_\epsilon$,
    \item Increased latency $\Lambda([A])$,
    \item Structural instability under continued RuleEvolution.
\end{itemize}

\section{Return Path Bifurcation} \label{return-path-bifurcation}

When reuse pressure and structural drift combine, an identity may undergo bifurcation—splitting into distinct return structures.

\paragraph{Bifurcation Onset.}
The condition for return path bifurcation is:
\begin{equation} \label{eq:bifurcation-condition}
\partial X_c < 0 \quad \text{and} \quad \exists\, i : \partial X_i \ne 0
\end{equation}

This marks a transition from coherence lock to return multiplicity: the identity no longer stabilizes under a unique coherence signature but resolves through divergent return paths.

\section{Structural Visualization} \label{structural-visualization}

Drift behavior induces topological structure within $\chi$-space. Notable geometric features include:

\begin{itemize}
    \item \textbf{Attractors:} stable identities with $\partial\chi = 0$,
    \item \textbf{Collapse ridges:} regions where drift components diverge,
    \item \textbf{Elevation spirals:} trajectories of sustained increase in $X_\epsilon$.
\end{itemize}

These emergent forms establish the basis for symbolic flow and thermodynamic modeling in Chapter 6.

\section{Summary}

Symbolic evolution is classified using component-wise drift:

\begin{table}[h!]
\centering
\begin{tabular}{|l|l|}
\hline
\textbf{Behavior} & \textbf{Condition} \\
\hline
Stable     & $\partial X_i = 0$ for all $i$ \\
Reactive   & $\partial X_i$ bounded, $\partial X_i \ne 0$ for some $i$ \\
Collapsing & $\exists\, i$ such that $|\partial X_i| \to \infty$ \\
\hline
\end{tabular}
\caption{Component-wise classification of symbolic drift behavior}
\end{table}

\medskip

These symbolic dynamics describe the flow of identities under coherence pressure in $\chi$-space.
