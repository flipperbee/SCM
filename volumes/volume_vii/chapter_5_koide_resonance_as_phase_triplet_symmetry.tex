\chapter{The Koide Resonance as Phase Triplet Symmetry} \label{chapter:koide-resonance}

\section{Background and Formula}

The Koide mass formula is an empirical relationship between the masses of the electron, muon, and tau:
\begin{equation}
K := \frac{(\sqrt{m_1} + \sqrt{m_2} + \sqrt{m_3})^2}{m_1 + m_2 + m_3} = \frac{3}{2},
\end{equation}
where $m_1$, $m_2$, and $m_3$ are the masses of the three charged leptons, typically ordered as $m_e$, $m_\mu$, and $m_\tau$.

This relationship is accurate to within measurement precision, but has no known derivation from standard model physics. In SCM, however, this resonance is not an accident—it is a structural consequence of triplet coherence.

\section{Triplet Return Configuration on the Unit Circle}

Let three symbolic identities $[A_1], [A_2], [A_3] \in \Omega_3$ be arranged on the Eulerian unit circle with equal angular separation $\Delta \theta = 2\pi / 3$:
\[
\theta_1 = 0, \quad \theta_2 = \frac{2\pi}{3}, \quad \theta_3 = \frac{4\pi}{3}.
\]

These represent phase-locked return structures forming a closed, symmetric triplet.

\begin{definition}[Koide Triplet Configuration]
A set of identities $\{[A_1], [A_2], [A_3]\}$ forms a Koide triplet if:
\[
\sum_{i=1}^3 e^{i\theta_i} = 0.
\]
\end{definition}

\noindent This implies that their complex phase vectors form an equilateral triangle centered at the origin—i.e., **complete phase cancellation without collapse**.

\section{Interpretation as Coherence Condition}

This configuration satisfies:
\[
e^{i\theta_1} + e^{i\theta_2} + e^{i\theta_3} = 1 + e^{i \frac{2\pi}{3}} + e^{i \frac{4\pi}{3}} = 0.
\]

Thus, the average phase vector is zero, but the system is not collapsing—it is phase-balanced.

We interpret this as:
\begin{itemize}
    \item A minimal-contradiction configuration under reuse,
    \item A coherence-preserving triplet rotation with zero net deformation,
    \item A geometric resonance condition for return stability.
\end{itemize}

\section{Mass from Phase Separation}

Assume each identity has symbolic mass $m_i \propto \Lambda_i$, with:
\[
\Lambda_i = \text{latency corresponding to angle } \theta_i.
\]

Let us assume $\sqrt{m_i} \propto \cos(\theta_i / 2)$—i.e., mass arises from projection of phase return closure.

Then:
\[
\sum_{i=1}^3 \sqrt{m_i} \propto \sum \cos(\theta_i / 2),
\]
and the Koide formula becomes a structural symmetry condition among these phase-deformed coherence forms.

\section{Structural Derivation (Sketch)}

Let:
\begin{align*}
a &:= \sqrt{m_1}, \\
b &:= \sqrt{m_2}, \\
c &:= \sqrt{m_3}.
\end{align*}

Suppose the return structure imposes:
\[
a^2 + b^2 + c^2 = \frac{2}{3}(a + b + c)^2.
\]

Expanding RHS:
\[
\frac{2}{3}(a^2 + b^2 + c^2 + 2ab + 2ac + 2bc) = a^2 + b^2 + c^2 + \frac{4}{3}(ab + ac + bc)
\Rightarrow ab + ac + bc = 0.
\]

This implies the vectors $(a, b, c)$ form a **mutually canceling return system**, reinforcing the geometric triplet structure.

\section{Conclusion and Implications}

\begin{itemize}
    \item The Koide relation is not a numerical coincidence—it is a structural identity resonance.
    \item It arises from three identities placed at 120° separation on the unit circle.
    \item The $K = 3/2$ condition reflects **minimum contradiction and maximal phase stability**.
    \item The appearance of square roots is not mysterious: it reflects angular projections of return-based latency.
\end{itemize}

\noindent Therefore, the Koide triplet is the **first nontrivial solution to the SCM variational law** under triplet reuse constraints.

These insights lead to a deeper understanding of coherence geometry, which we summarize in the final chapter.
