\chapter{Constants of Structure and Physical Emergence} \label{chapter:constants-physical-emergence}

\section{Structural Thresholds from Eulerian Return}

SCM defines structural constants not as empirical inputs, but as symbolic thresholds that emerge from the logic of return closure.

Let:
\begin{itemize}
    \item $\theta$: phase parameter in Euler return structure,
    \item $\Lambda([A])$: latency of return path,
    \item $\rho([A])$: robustness to collapse,
    \item $\chi([A])$: coherence signature.
\end{itemize}

\noindent Then identity persistence is constrained by symbolic angular properties of $T([A]) = e^{i\theta}$.

\section{The Mass Gap from Minimal Return Closure}

The first permitted identity in SCM arises from a closed return loop with:
\[
X_h([A]) = 2 \quad \Rightarrow \quad \theta = 2\pi.
\]

\begin{definition}[Mass Gap]
The mass gap $\Lambda_0$ is the minimal nonzero latency associated with any identity $[A] \in \Omega_3$:
\[
\Lambda_0 := \min \left\{ \Lambda([A]) \ \middle| \ X_h([A]) \geq 2, \ \chi([A]) \text{ stable} \right\}.
\]
\end{definition}

\noindent In angular terms, this means:
\[
\Lambda_0 = \text{effort to rotate full $2\pi$ without collapse}.
\]

This is equivalent to the minimal coherence energy needed to sustain a phase-closed identity.

\section{Mass as Rotational Persistence}

Let return effort be additive over transformations:
\[
\Lambda([A]) = \sum_{i} \text{effort}(T_i).
\]

\begin{definition}[Symbolic Mass]
Mass is defined structurally as:
\[
m([A]) := f(\Lambda([A]), X_\phi([A])),
\]
where $f$ is a monotonic function of latency and fragility.
\end{definition}

\noindent The more fragile or phase-dense the return structure, the higher its symbolic mass.

Mass is thus:
\begin{itemize}
    \item Not a substance,
    \item But a measure of rotational persistence and reuse fragility in Eulerian coherence space.
\end{itemize}

\section{Planck-Scale Resolution and Collapse}

As return paths become increasingly dense (shorter effort per rotation), coherence fails due to symbolic overlap.

Let the minimal distinguishable angle $\Delta \theta_{\min}$ define the coherence resolution limit.

\begin{definition}[Planck Resolution Threshold]
Let:
\[
\hbar_{\text{eff}} := \min \left\{ \Delta \theta \cdot \Lambda([A]) \ \middle| \ [A] \in \Omega_3 \right\}.
\]
Then $\hbar_{\text{eff}}$ is the minimal resolution product for symbolic identity stability.
\end{definition}

This matches Planck's constant in quantum mechanics:
\[
\Delta x \Delta p \gtrsim \hbar \quad \leftrightarrow \quad \Delta \theta \cdot \Lambda \gtrsim \hbar_{\text{eff}}.
\]

\noindent Planck’s constant is not added to SCM—it is derived as the **lowest sustainable coherence product**.

\section{Collapse Threshold: $\rho_c$}

An identity collapses when too few coherent transformations support its return loop.

\begin{definition}[Collapse Robustness Threshold]
\[
\rho_c := \min \left\{ \rho([A]) \ \middle| \ [A] \in \Omega_3 \text{ and } \partial \chi = 0 \right\}
\]
where $\rho([A])$ is the minimal number of return-permitting transformations needed for closure.
\end{definition}

Collapse occurs when:
\[
\rho([A]) < \rho_c \quad \Rightarrow \quad [A] \to \Omega_-.
\]

\section{Reuse Saturation and Return Tension}

If an identity is reused by too many structures simultaneously, its return geometry becomes over-constrained and coherence begins to fail.

\begin{definition}[Reuse Pressure]
Let:
\[
P([A]) := \frac{L([A])}{C([A])},
\]
where $L([A])$ is the number of identities reusing $[A]$, and $C([A])$ is its reuse capacity.
\end{definition}

\noindent Then there exists a critical pressure $P_{\text{crit}}$ such that:
\[
P([A]) > P_{\text{crit}} \quad \Rightarrow \quad \partial X_\epsilon \to \infty \Rightarrow \text{elevation collapse}.
\]

This defines the upper limit of coherence network density.

\section{Summary of Constants and Structural Roles}

\begin{table}[h!]
\centering
\begin{tabular}{|l|c|l|}
\hline
\textbf{Constant} & \textbf{Symbol} & \textbf{Interpretation} \\
\hline
Mass Gap & $\Lambda_0$ & Minimum latency for stable return \\
Planck Scale & $\hbar_{\text{eff}}$ & Angular-effort resolution threshold \\
Collapse Robustness & $\rho_c$ & Minimum support for coherence preservation \\
Reuse Pressure Limit & $P_{\text{crit}}$ & Max density before elevation collapse \\
\hline
\end{tabular}
\caption{Structural constants emerging from Eulerian return logic}
\end{table}

\noindent These constants are not empirical. They are the structural consequences of return-closed identity on the unit circle.
