\chapter{Emergence of Mass} \label{chapter:mass}

\section{18.1 \textbar{} Structural Precondition: The Mass Gap}
Mass in SCM does not arise from intrinsic substance, but from symbolic latency. As shown in the previous chapter, the existence of identity $[A] \in \Omega_3$ requires a minimal latency:
\[
\Lambda([A]) \geq \Lambda_0
\]
This mass gap, $\Lambda_0$, defines the structural energy floor needed to sustain coherence over a closed return loop.

\section{18.2 \textbar{} Fragility as Return Instability}
Let $X_\phi([A])$ denote the fragility of an identity, defined as its likelihood to collapse under reuse or RuleEvolution pressure. Fragility reflects:
\begin{itemize}
    \item Sensitivity to deformation,
    \item Collapse-prone symmetry or low support,
    \item Instability in return coherence.
\end{itemize}

\section{18.3 \textbar{} Elevation as Reuse Role}
Return elevation $X_\epsilon([A])$ measures how far identity $[A]$ sits above its own reuse base. It counts the minimal number of coherence layers required to complete return, with $X_\epsilon = 0$ indicating a foundational identity (e.g., the photon), and $X_\epsilon > 0$ indicating structural elevation.

\section{18.4 \textbar{} Return Asymmetry and Resistance}
The signature component $X_\pi([A])$ captures return symmetry:
\[
X_\pi = -1 \Rightarrow \text{inversion/collapse}, \quad
X_\pi = 0 \Rightarrow \text{perfect balance}, \quad
X_\pi \to 1 \Rightarrow \text{highly asymmetrical loop}.
\]
This metric plays a key role in collapse resistance and coherence pressure.

\section{18.5 \textbar{} Definition: Symbolic Mass}
SCM defines mass not as an intrinsic quantity, but as a structural latency threshold.

\begin{definition}[Symbolic Mass]
Let $[A] \in \Omega_3$ be a resolved identity. Then:
\[
m([A]) := \Lambda([A]),
\]
where $\Lambda([A])$ is the total effort-weighted latency of the minimal return path resolving $[A]$.
\end{definition}

\noindent Mass represents the irreducible cost of symbolic return. It reflects the structural investment required to sustain persistence. This definition is invariant under transformation and applies across all volumes, including Volume IV (particle classification) and Volume V (Koide resonance).

\paragraph{Related Concept: Structural Inertia}

Not all massive identities are equally stable under reuse. We define a related quantity:

\begin{definition}[Structural Inertia]
Let $[A]$ have coherence signature components $(X_\phi, X_\pi, X_\epsilon)$. Then:
\[
\mathcal{I}([A]) := \frac{X_\phi([A]) \cdot X_\epsilon([A])}{(1 - X_\pi([A]))^2}
\]
\end{definition}

\noindent Inertia captures symbolic resistance to deformation—not total return cost, but structural stiffness. It plays a key role in interaction dynamics and will be used in later volumes to model interaction cross-section and coherence lifetime.

\section{18.6 \textbar{} Interpretation}
\begin{itemize}
    \item $m([A])$ (mass) measures the absolute effort needed to sustain coherence.
    \item $\mathcal{I}([A])$ (inertia) measures how easily $[A]$ is disturbed by reuse or drift.
    \item Mass is global; inertia is local.
\end{itemize}

\section{18.7 \textbar{} Massless Identities}
Let $[A] \in \Omega_3$ with $\Lambda([A]) = \Lambda_0$. Then $[A]$ has minimal mass and is often structurally invariant:
\[
X_\phi = 0, \quad X_\pi = 0, \quad X_\epsilon = 0
\]
Such identities include the photon and other return-closed, reuse-invariant forms. These satisfy:
\[
m([A]) = \Lambda_0
\]
and form the coherence ground layer of the SCM structure.

\section{18.8 \textbar{} The Uncertainty Principle as Structural Latency Tradeoff}
SCM predicts the following structural bound:
\[
\Lambda([A]) \cdot X_\phi([A]) \geq \Lambda_0
\]
This expresses a fundamental tradeoff: to achieve low latency, an identity must become fragile. Robust, reusable identities require effort to construct. This will be formally proven in Appendix A.

\section{18.9 \textbar{} Summary Table}

\begin{table}[h!]
\centering
\begin{tabular}{|l|l|l|}
\hline
\textbf{Quantity} & \textbf{Formula} & \textbf{Interpretation} \\
\hline
Mass $m([A])$ & $\Lambda([A])$ & Return effort to sustain coherence \\
Inertia $\mathcal{I}([A])$ & $\frac{X_\phi \cdot X_\epsilon}{(1 - X_\pi)^2}$ & Deformation resistance \\
Fragility $X_\phi$ & -- & Collapse tendency \\
Elevation $X_\epsilon$ & -- & Reuse height above base \\
Symmetry $X_\pi$ & -- & Return loop asymmetry \\
\hline
\end{tabular}
\caption{Symbolic quantities underlying mass and inertia}
\end{table}
