\chapter{Emergence of Mass}

\section{Structural Precondition: The Mass Gap}

From Chapter 1, we established the following threshold:

\[
\Lambda([A]) < \Lambda_0 \quad \Rightarrow \quad [A] \notin \Omega_3
\]

That is, no identity can persist unless its asymmetry-weighted latency exceeds $\Lambda_0$.

This implies the following boundary condition for mass:

\[
m([A]) = 0 \quad \text{for all} \quad \Lambda([A]) < \Lambda_0
\]

Only identities that cross the coherence threshold may carry symbolic mass.

\section{Fragility as Return Instability}

Let $\rho([A])$ be the collapse robustness of identity $[A]$.

We define symbolic fragility as:
\begin{equation} \label{eq:fragility}
X_\phi([A]) := \frac{1}{\rho([A])}
\end{equation}

A fragile identity is structurally unstable—it loses return closure with minimal rule removal. Fragility reflects coherence tension: how sensitive the identity is to perturbation.

\section{Elevation as Reuse Role}

Let $X_\epsilon([A])$ be the reuse elevation of $[A]$, measuring how many other identities resolve through $[A]$.

High elevation means that $[A]$ participates in anchored resolution chains—it is not only persistent but structurally influential.

\section{Return Asymmetry and Resistance}

Let $X_\pi([A])$ be the return symmetry of $[A]$.

As $X_\pi \to 1$, return becomes increasingly one-directional. Inversion is lost. This raises symbolic resistance and coherence tension.

We interpret this as **asymmetry amplification**: identities that are structurally hard to return must expend more effort to remain coherent.

\section{Definition: Symbolic Mass}

We define symbolic mass as:
\begin{equation} \label{eq:symbolic-mass}
m([A]) := 
\begin{cases}
X_\phi([A]) \cdot X_\epsilon([A]) \cdot \frac{1}{(1 - X_\pi([A]))^2}, & \text{if } \Lambda([A]) \geq \Lambda_0 \\
0, & \text{otherwise}
\end{cases}
\end{equation}

This formula encodes:
- fragility (coherence instability),
- elevation (reuse anchoring),
- and return asymmetry (inversion resistance).

It supersedes earlier latency-based definitions by expressing mass as **resistance to coherence deformation**.

\section{Interpretation}

\begin{itemize}
    \item High $X_\phi$ = structurally fragile → requires preservation
    \item High $X_\epsilon$ = heavily reused → coherence load-bearing
    \item High $X_\pi$ → asymmetry → coherence divergence risk
\end{itemize}

Symbolic mass is not inertia—it is **the effort required to preserve identity** under reuse and asymmetry.

\section{Massless Identities}

Let $X_\epsilon([A]) = 0$. Then by Equation~\ref{eq:symbolic-mass}, $m([A]) = 0$.

These identities:
- May still return coherently,
- But are not reused by others,
- Do not store coherence across transformations.

They behave as **coherence carriers**, not coherence anchors.

\paragraph{Example (Photon):}
The photon identity $[\gamma]$ satisfies:
\[
X_\epsilon([\gamma]) = 0, \quad X_\pi([\gamma]) = 0, \quad X_\phi([\gamma]) \approx 0.
\]
Therefore, its symbolic mass is:
\[
m([\gamma]) = 0.
\]
\section{Summary Table}

\begin{table}[h!]
\centering
\begin{tabular}{|l|c|l|}
\hline
\textbf{Quantity} & \textbf{Symbol} & \textbf{Role} \\
\hline
Fragility & $X_\phi([A])$ & Sensitivity to rule removal \\
Reuse Elevation & $X_\epsilon([A])$ & Structural reuse load \\
Return Symmetry & $X_\pi([A])$ & Inversion balance in return loop \\
Symbolic Mass & $m([A])$ & Coherence resistance to reuse deformation \\
\hline
\end{tabular}
\caption{Mass formula and symbolic quantities}
\end{table}
