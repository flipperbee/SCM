\chapter{The Mass Gap}

\section{Identity Persistence Requires Latency}

In SCM, identity is coherence-closed structure. A symbolic form $[A] \in \Omega_3$ exists if—and only if—it returns under permitted transformation. But not all return paths define persistence. Some collapse under minimal RuleEvolution; others never stabilize. 

This chapter formalizes the first nontrivial structural threshold in SCM: the minimum latency required for an identity to persist in $\Omega_3$. This is the symbolic \textbf{mass gap}.

\section{Latency and Coherence Survival}

Let $[A] \in \Omega_3$ be a candidate identity. Let $\mathcal{L}([A]) = \{T_1, T_2, \dots, T_n\}$ be a permitted return loop.

We define:
\begin{equation} \label{eq:euler-latency}
\Lambda([A]) := \text{len}(\mathcal{L}([A])) \cdot \frac{1}{1 - X_\pi([A])}
\end{equation}

This is the asymmetry-weighted latency. As $X_\pi([A]) \to 1$, latency diverges. As $X_\pi([A]) \to 0$, latency approximates path length.

\paragraph{Persistence condition.} Identity $[A]$ can persist in $\Omega_3$ only if:
\begin{enumerate}
  \item $\mathcal{L}([A])$ exists,
  \item Collapse robustness $\rho([A]) \geq 1$,
  \item $[A]$ survives reuse ($\partial\chi([A])$ bounded).
\end{enumerate}

\section{Theorem: Latency Threshold for Identity Persistence}

\begin{definition}[Mass Gap Theorem] \label{thm:mass-gap}
There exists a minimum latency $\Lambda_0 > 0$ such that:
\[
\Lambda([A]) < \Lambda_0 \quad \Rightarrow \quad [A] \notin \Omega_3
\]
\end{definition}

\begin{proof}[Sketch]
If $\Lambda \to 0$, either:
- the loop is degenerate (e.g., identity transformation), which is disallowed by SCM's return rules;
- or the loop is too fragile ($\rho([A]) = 1$) and eliminated by RuleEvolution;
- or drift is unbounded ($\partial\chi \ne 0$) and return does not stabilize.

In all cases, $[A]$ cannot persist. Thus, a minimum latency floor $\Lambda_0$ must exist.
\end{proof}

\section{Definition: The Mass Gap}

We define the mass gap as:
\begin{equation} \label{eq:mass-gap-definition}
\Lambda_0 := \inf \left\{ \Lambda([A]) \mid [A] \in \Omega_3 \text{ and } \rho([A]) \geq 1 \right\}
\end{equation}

This is the coherence threshold: no identity exists below it.

\section{Structural Consequences}

Identities with $\Lambda([A]) < \Lambda_0$:
- cannot act as reuse anchors,
- collapse under minimal RuleEvolution perturbation,
- fail to satisfy reuse persistence conditions.

This threshold behaves as a symbolic quantization condition. It defines the floor of persistence in SCM: only identities above $\Lambda_0$ are coherent under return.

\section{Interpretation}

The mass gap is not arbitrary. It arises from:
\begin{itemize}
  \item The minimum structure needed for return closure,
  \item The symbolic resistance required to prevent RuleEvolution collapse,
  \item The amplification of return cost under asymmetry.
\end{itemize}

This is the first quantized threshold in SCM. It marks the emergence of structure from permitted return.

\section{Summary Table}

\begin{table}[h!]
\centering
\begin{tabular}{|l|c|l|}
\hline
\textbf{Quantity} & \textbf{Symbol} & \textbf{Role} \\
\hline
Latency & $\Lambda([A])$ & Return cost under asymmetry \\
Mass Gap & $\Lambda_0$ & Minimum latency for identity to persist \\
Collapse Robustness & $\rho([A])$ & Rule removal threshold \\
Persistence Condition & $\Lambda([A]) \geq \Lambda_0$ & Required for $[A] \in \Omega_3$ \\
\hline
\end{tabular}
\caption{Structural terms defining the symbolic mass gap}
\end{table}
