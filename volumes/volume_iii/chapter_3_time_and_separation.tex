\chapter{Time and Separation}

\section{Symbolic Time as Reuse Interval}

In SCM, time does not exist a priori. It emerges from the reuse of resolved identities. Let $[A] \in \Omega_3$ be a coherence-resolved identity reused in multiple resolution chains.

We define the symbolic reuse interval:
\begin{equation} \label{eq:reuse-interval}
\Delta\tau([A]) := \min \left\{ n \mid [A] \text{ reappears in } \mathcal{L}([B_n]) \text{ after reuse separation} \right\}
\end{equation}

This interval acts as a symbolic time step: the structural delay between reuse instances of $[A]$.

\paragraph{Definition.} Symbolic time for identity $[A]$ is:
\begin{equation} \label{eq:symbolic-time}
\tau([A]) := \sum_i \Delta\tau_i([A])
\end{equation}

Time in SCM is cumulative reuse delay. It is identity-relative, not global.

\section{Latency Drift as Local Clock Rate}

Let $\Lambda_t([A])$ be the latency of $[A]$ at symbolic time $t$. Define the latency drift:
\begin{equation} \label{eq:latency-drift}
\partial_t \Lambda([A]) := \Lambda_{t+1}([A]) - \Lambda_t([A])
\end{equation}

This is the local clock rate for $[A]$. It indicates how much symbolic effort is accumulating across reuse.

\begin{itemize}
    \item $\partial_t \Lambda = 0$ → return loop is stable
    \item $\partial_t \Lambda > 0$ → resolution is decaying
    \item $\partial_t \Lambda < 0$ → resolution is collapsing
\end{itemize}

\section{Drift Geometry and Motion}

Let $\chi([A])$ be the coherence signature of $[A]$.

Define symbolic velocity:
\begin{equation} \label{eq:symbolic-velocity}
v_\chi([A]) := \frac{\partial \chi([A])}{\partial \tau}
\end{equation}

This is the rate of coherence signature change per reuse interval. It expresses motion not as trajectory, but as identity deformation over time.

\begin{itemize}
    \item $v_\chi = 0$: [A] is structurally stationary
    \item $v_\chi \ne 0$ but bounded: [A] is reactive
    \item $v_\chi \to \infty$: [A] is collapsing
\end{itemize}

\section{Symbolic Separation}

Let $[A], [B] \in \Omega_3$ be two identities.

Define structural separation:
\begin{equation} \label{eq:separation}
d([A], [B]) := \min \left\{ \text{length}(\mathcal{L}([C])) \mid [A], [B] \subset \mathcal{L}([C]) \right\}
\end{equation}

This is the minimal permitted transformation path connecting $[A]$ and $[B]$ via reuse. It is not a metric:
- $d([A], [B]) \ne d([B], [A])$
- Triangle inequality may not hold
- Distance reflects coherence embedding, not space

\section{Temporal Structure Without Coordinates}

In SCM:
- Time is reuse delay ($\tau$)
- Motion is deformation of $\chi$
- Distance is coherence separation

There is no underlying manifold. All structure is derived from return logic.

\section{Summary Table}

\begin{table}[h!]
\centering
\begin{tabular}{|l|c|l|}
\hline
\textbf{Concept} & \textbf{Symbol} & \textbf{Role} \\
\hline
Symbolic Time & $\tau([A])$ & Reuse-accumulated delay for identity \\
Reuse Interval & $\Delta\tau([A])$ & Time between permitted return events \\
Latency Drift & $\partial_t \Lambda([A])$ & Clock rate under reuse \\
Symbolic Velocity & $v_\chi([A])$ & Structural deformation rate \\
Symbolic Distance & $d([A], [B])$ & Minimal reuse separation \\
\hline
\end{tabular}
\caption{Symbolic definitions of time, motion, and separation}
\end{table}
