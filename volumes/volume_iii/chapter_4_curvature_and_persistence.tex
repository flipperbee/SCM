\chapter{Curvature and Persistence}

\section{From Drift to Curvature}

In SCM, identity evolves under symbolic reuse. Structural drift $\partial\chi$ captures this evolution. But drift itself may change—become suppressed, concentrated, or redirected.

\paragraph{Definition (Symbolic Curvature):}
We define symbolic curvature not as acceleration in metric space, but as the \textbf{gradient of structural robustness} in $\chi$-space. That is,
\[
\mathcal{K} := \nabla \rho([A]),
\]
where $\rho([A])$ is the collapse robustness of identity $[A]$. Regions with high $|\mathcal{K}|$ indicate structural instability—identities deform under reuse even without spatial displacement.

\noindent This is curvature \emph{without geometry}: a purely symbolic measure of deformation in return logic.

\section{Robustness Field and Gradient}

Let $\rho([A])$ be the collapse robustness of identity $[A]$.

Define the robustness gradient:
\begin{equation} \label{eq:robustness-gradient}
\nabla \rho := \left( \frac{\partial \rho}{\partial X_h}, \frac{\partial \rho}{\partial X_c}, \frac{\partial \rho}{\partial X_\pi}, \frac{\partial \rho}{\partial X_\phi}, \frac{\partial \rho}{\partial X_\epsilon} \right)
\end{equation}

Each partial derivative reflects how fragile the identity becomes as one coherence component changes.

\section{Definition: Symbolic Curvature}

We define symbolic curvature at $[A]$ as:
\begin{equation} \label{eq:symbolic-curvature}
\mathcal{K}([A]) := \max_i \left| \frac{\partial \rho([A])}{\partial X_i} \right|
\end{equation}

Where the maximum is taken across all coherence signature components.

\paragraph{Interpretation.}
- High $\mathcal{K}$: robustness changes rapidly → drift concentrates → identity destabilizes
- Low $\mathcal{K}$: robustness is flat → drift slows → identity is coherence-stable

\section{Persistence Without Norms}

Let:
\[
\chi([A]) = (X_h, X_c, X_\pi, X_\phi, X_\epsilon)
\]
Let:
\[
\partial\chi([A]) = (\partial X_h, \partial X_c, \partial X_\pi, \partial X_\phi, \partial X_\epsilon)
\]

We define symbolic persistence as:
\begin{equation} \label{eq:persistence}
\Pi([A]) := \frac{1}{\max_i |\partial X_i([A])|}
\end{equation}

\paragraph{Interpretation.}
\begin{itemize}
    \item $\Pi \to \infty$ if $\partial X_i = 0$ for all $i$ — the identity is fully persistent.
    \item $\Pi$ is small if $\partial X_i$ is large for any $i$ — the identity is fragile or collapsing.
\end{itemize}

This definition avoids norms or distance—it reflects coherence decay via fastest-deforming signature component.

\section{Curvature–Persistence Relationship}

Symbolic curvature and persistence are inversely related.

\begin{itemize}
    \item If $\mathcal{K}([A])$ increases, the robustness gradient steepens — identity becomes structurally unstable.
    \item If $\Pi([A])$ decreases, signature components change more rapidly — drift accelerates.
\end{itemize}

Thus:
\begin{equation} \label{eq:curvature-inverse-persistence}
\mathcal{K}([A]) \cdot \Pi([A]) \approx \text{bounded constant}
\end{equation}

This defines symbolic coherence balance: high curvature implies low persistence, and vice versa.

\section{Summary Table}

\begin{table}[h!]
\centering
\begin{tabular}{|l|c|l|}
\hline
\textbf{Quantity} & \textbf{Symbol} & \textbf{Interpretation} \\
\hline
Robustness Gradient & $\nabla \rho$ & Sensitivity of collapse stability to $\chi$ \\
Symbolic Curvature & $\mathcal{K}([A])$ & Max rate of robustness change \\
Signature Drift & $\partial\chi([A])$ & Rate of identity deformation \\
Persistence & $\Pi([A])$ & Inverse of max drift component \\
Curvature–Persistence Law & $\mathcal{K} \cdot \Pi \approx$ const & Structural tradeoff \\
\hline
\end{tabular}
\caption{Coherence curvature and symbolic persistence}
\end{table}
