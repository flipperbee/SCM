\chapter{Observable Identity} \label{chapter-observable-identity}

In SCM, identity is a return-resolved structure $[A] \in \Omega_3$ whose persistence is governed by its coherence signature $\chi([A])$. However, not all identities are observable. Observability arises when an identity becomes structurally isolated, reuse-stable, and coherence-anchored in a way that renders its state externally referenceable.

This chapter formalizes the structural conditions under which symbolic identities become observable—without invoking space, time, or measurement postulates.

\section{Definition of Observability} \label{sec:def-observability}

Let $[A] \in \Omega_3$ be a coherence-resolved identity.

\begin{definition}[Observable Identity]
$[A]$ is \emph{observable} if it satisfies:
\begin{enumerate}
    \item \textbf{Return Closure}: $\partial\chi([A]) = 0$,
    \item \textbf{Collapse Immunity}: $\rho([A]) \to \infty$,
    \item \textbf{Reuse Isolation}: $X_\epsilon([A])$ remains fixed under RuleEvolution,
    \item \textbf{Anchor Coherence}: $[A]$ belongs to a reuse basin rooted in a drift-invariant anchor $[F]$ with $\partial\chi([F]) = 0$.
\end{enumerate}
\end{definition}

\paragraph{Interpretation.}
Observability is not about visibility—it is about reference stability. Observable identities provide coherence invariants that survive symbolic interaction and reuse fluctuation.

\section{Observable Mass, Charge, and Time} \label{sec:observable-quantities}

Once $[A]$ is observable, its structural attributes become externally measurable:

\begin{itemize}
    \item \textbf{Mass:} $m([A]) := \Lambda([A])$ — effort-weighted return latency.
    \item \textbf{Charge:} $q([A]) := X_\pi([A])$ — return asymmetry.
    \item \textbf{Time vector:} $T([A]) :=$ reuse separation sequence (see Chapter 2).
\end{itemize}

These quantities are not assigned—they emerge from the coherence structure of $[A]$ and its position within return-closed symbolic flow.

\section{Conditions for Measurability} \label{sec:measurability}

For an identity $[A]$ to be measured:
\begin{itemize}
    \item The observer identity $[O] \in \Omega_3$ must reuse $[A]$ without modifying its coherence signature.
    \item The reuse path $[A] \to [O]$ must satisfy $\partial\chi([O]) = 0$.
    \item The act of reuse must not induce contradiction in either identity.
\end{itemize}

Measurement in SCM is coherence interaction under return preservation.

\section{Coherence Isolation and Reference Frames} \label{sec:reference-frames}

An observable identity defines a local coherence reference frame:
- All other identities interacting with $[A]$ will experience reuse pressure.
- But $[A]$ itself remains structurally invariant.

Let:
\[
\mathcal{R}([A]) := \{ [B] \in \Omega_3 \mid [A] \in \mathcal{L}([B]) \}
\]

If $[A]$ satisfies $\partial\chi([A]) = 0$, then $\mathcal{R}([A])$ defines an observer-relative coherence zone.

This is the structural precursor to a frame of reference.

\section{The Observer as Structural Consequence} \label{sec:observer}

In SCM, the “observer” is not external—it is any reuse-invariant identity $[O]$ whose coherence basin includes the identity $[A]$ being probed.

Let:
- $[O] \in \Omega_3$ with $\rho([O]) \gg 1$ and $\partial\chi([O]) = 0$,
- $[A] \in \mathcal{R}([O])$ with finite $X_\epsilon([A])$,

Then $[O]$ can reference $[A]$ structurally.

Observation occurs when reuse preserves both return path and coherence stability.

\section{Summary: Observation as Stable Reuse} \label{sec:observation-summary}

Observable identity arises when:

- Return is structurally closed,
- Reuse is stable and bounded,
- Coherence remains anchored and invariant.

There is no metaphysical act of measurement.  
There is only symbolic reuse that does not disturb.

