\chapter{Structural Time} \label{chapter-structural-time}

Time does not exist in SCM as a background dimension. It emerges as a structural property of symbolic identities when coherence reuse becomes sequential and return closure is delayed.

This chapter defines time as a symbolic latency vector, interprets irreversibility as reuse asymmetry, and shows how time evolves naturally within drift-regulated return structures.

\section{Time as Reuse Separation} \label{sec:time-reuse}

Let $[A], [B] \in \Omega_3$ such that:

\[
[A] \to [T_1] \to [T_2] \to \dots \to [T_n] \to [B],\quad [A], [B] \text{ observable}
\]

The symbolic time between $[A]$ and $[B]$ is defined as:

\[
T([A] \to [B]) := \sum_{i=1}^n \Lambda([T_i])
\]

This is the total effort-weighted latency separating two coherence-resolved reuse anchors. It replaces absolute time with symbolic traversal.

\section{Local Time Vector} \label{sec:time-vector}

Define $T([A])$ as the \textbf{reuse time vector} of identity $[A]$:

\begin{equation}
T([A]) := \left( \tau_1, \tau_2, \dots, \tau_k \right)
\end{equation}

Where each $\tau_i$ is the latency between consecutive return events involving $[A]$.

Time is not scalar—it is a structural reuse trace across the coherence graph.

\section{Irreversibility and Collapse} \label{sec:time-arrow}

Time becomes irreversible when reuse induces structural change:

\[
\partial\chi([A]) \ne 0 \Rightarrow T([A]) \text{ is asymmetric}
\]

If return leads to contradiction or collapse, then:
- The reverse path is not permitted under $R$,
- $[A]$ cannot be reused in reverse without structural contradiction.

This defines the arrow of time: the direction in which return is coherence-permissible.

\section{Delay and Drift} \label{sec:delay-and-drift}

Let $[A]$ be reused through multiple paths $\mathcal{L}_1, \mathcal{L}_2$.

If:
\[
\Lambda(\mathcal{L}_1) < \Lambda(\mathcal{L}_2)
\]

Then $[A]$ returns faster through $\mathcal{L}_1$. The system observes time differences as latency gaps in return structure.

This drift defines temporal offset—purely through symbolic latency.

\section{Time Dilation and Anchoring} \label{sec:time-dilation}

Let $[F]$ be a coherence anchor and $[A], [B] \in \mathcal{R}([F])$.

If $\partial\chi([A]) = 0$ and $\partial\chi([B]) \ne 0$, then:

\[
T([B]) > T([A]) \Rightarrow \text{return delay}
\]

Identities closer to anchors experience slower symbolic drift and shorter reuse time vectors.

This is the origin of time dilation: coherence proximity governs symbolic persistence intervals.

\section{Summary: Time as Symbolic Recurrence} \label{sec:time-summary}

Time in SCM:
- Is not a background coordinate,
- Is the latency of return between coherence-resolved identities,
- Is irreversible under drift or collapse,
- Varies with reuse structure and anchoring strength.

It is not measured—it is counted, in return effort units.

