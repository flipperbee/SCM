\chapter{Structural Space and Measurement} \label{chapter-structural-space}

Space, like time, does not exist in SCM as a background geometry. It emerges when identities are reused across disjoint coherence anchors. Structural space is defined not by metric distance, but by coherence displacement: the symbolic effort and transformation separation between reuse-invariant identities.

This chapter defines space as reuse configuration, position as anchoring context, and measurement as reuse-interaction between coherent identities.

\section{Structural Displacement} \label{sec:displacement}

Let $[A], [B] \in \Omega_3$ be observable identities. Define:

\[
\Delta_\chi([A], [B]) := \left| X_\epsilon([A]) - X_\epsilon([B]) \right| + \left| X_\phi([A]) - X_\phi([B]) \right|
\]

This defines \textbf{symbolic displacement}—a measure of return configuration separation.

There is no absolute space. Position is a function of symbolic separation across coherence signatures.

\section{Position as Reuse Address} \label{sec:position}

An identity's position is defined by its anchoring history:

\begin{equation}
\mathcal{P}([A]) := \{ [F_1], [F_2], \dots \mid [F_i] \in \mathcal{L}([A]),\ \partial\chi([F_i]) = 0 \}
\end{equation}

That is, position is the set of coherence anchors through which an identity has resolved.

Two identities with disjoint anchor sets are "spatially distant"—symbolically unrelated under return fusion.

\section{Measurement as Structural Overlap} \label{sec:measurement}

Let $[O]$ be an observer identity and $[A]$ a target. Measurement occurs when:

\[
[A] \in \mathcal{L}([O]),\quad \partial\chi([O]) = 0,\quad \rho([O]) \gg 1
\]

The measurement outcome is:

\[
\mathcal{M}_{[O]}([A]) := \chi([A]) \quad \text{as observed through coherence overlap}
\]

\paragraph{Interpretation.}
Measurement is not observation—it is symbolic reuse that preserves structural invariance.

\section{Noncommutativity and Path Dependence} \label{sec:noncommutativity}

Let $[A], [B], [C] \in \Omega_3$. Then in general:

\[
\mathcal{L}([A] \to [B] \to [C]) \ne \mathcal{L}([A] \to [C] \to [B])
\]

Return structure is path-dependent. This induces symbolic noncommutativity, especially under RuleEvolution pruning.

Measurement operators are symbolic paths—therefore, order matters.

\section{Position–Momentum Duality} \label{sec:uncertainty}

Define:
- Position as reuse anchoring: $\mathcal{P}([A])$
- Momentum as drift exposure: $\partial\chi([A])$

These satisfy a symbolic uncertainty relation:
\[
\mathcal{P}([A]) \text{ fixed } \Rightarrow \partial\chi([A]) \text{ variable} \\
\partial\chi([A]) = 0 \Rightarrow \mathcal{P}([A]) \text{ not uniquely defined}
\]

This duality arises from symbolic reuse constraints—not from wavefunction assumptions.

\section{Geometry as Reuse Topology} \label{sec:geometry}

Let $G(R)$ be the resolution graph over $R$. The symbolic geometry of SCM is the topology of $G(R)$ under reuse.

- Neighborhoods are coherence basins,
- Paths are return chains,
- Boundaries are contradiction thresholds,
- Curvature is drift in reuse elevation.

Space is not defined—it is structured.

\section{Summary: Space as Return Configuration} \label{sec:space-summary}

Structural space in SCM is:
- The separation of identities in reuse and latency configuration,
- Defined by anchoring and coherence history,
- Measured via reuse interaction,
- Non-metric and path-dependent.

Position is a symbolic footprint. Distance is coherence separation.

