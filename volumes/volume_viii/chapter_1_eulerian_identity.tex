\chapter{Eulerian Identity and Phase Rotation} \label{chapter:euler-identity}

SCM defines identity not as a fixed object, but as a resolved structure—a return-closed loop that satisfies coherence conditions in symbolic space. In the Eulerian extension, we now interpret identity as a phase-locked structure rotating in symbolic coherence space. This rotation is governed by the complex exponential:
\[
e^{i\theta}
\]
and provides the geometric foundation for quantum coherence, phase symmetry, and quantized interaction.

\section{Euler's Formula and Coherence Rotation}

\begin{definition}[Euler Coherence Element]
Each permitted transformation $T \in R$ in a return loop contributes a symbolic phase rotation of the form:
\[
T \sim e^{i\theta}
\]
where $\theta$ is the return angle of symbolic coherence induced by $T$.
\end{definition}

A return loop $\mathcal{L}([A])$ then corresponds to a net phase rotation:
\[
\prod_{T_i \in \mathcal{L}([A])} e^{i\theta_i} = e^{i\sum \theta_i}
\]

Persistence requires that the total return angle is a full coherence rotation:
\[
\sum \theta_i = 2\pi n \quad \Rightarrow \quad e^{i\sum \theta_i} = 1
\]

\begin{definition}[Euler-Stable Identity]
An identity $[A] \in \Omega_3$ is Euler-stable if its minimal return loop satisfies:
\[
\sum_{\mathcal{L}([A])} \theta_i = 2\pi n,\quad n \in \mathbb{Z}
\]
\end{definition}

\paragraph{Interpretation.}  
A coherent identity is a closed rotation in symbolic phase space. This is the Eulerian analogue of a conserved quantum phase: the identity returns to itself.

---

\section{Symbolic Mass as Phase Rotation Effort}

From Volume III, symbolic mass was defined as:
\[
m([A]) := \Lambda([A])
\]
the effort-weighted latency of the minimal return loop.

Now, we reinterpret mass as the **total angular effort** required to complete a coherence-closed rotation.

\begin{definition}[Angular Return Effort]
Let $\theta_{\text{total}}([A])$ be the net rotation angle of identity $[A]$. Then:
\[
\Lambda([A]) := \int_{\mathcal{L}([A])} \mathrm{d}\theta \cdot \text{effort}(\theta)
\]
\end{definition}

In the simplest case, effort is constant per angular step:
\[
\Lambda([A]) \propto |\theta_{\text{total}}([A])|
\]

Thus, greater mass corresponds to greater symbolic cost to complete a full return in phase space. The photon, with $\Lambda = \Lambda_0$, corresponds to the minimal nonzero return: a complete $2\pi$ rotation.

---

\section{Return Inversion and Identity Collapse}

\begin{definition}[Inversion Return]
An identity that resolves as:
\[
[A] \to [-A] \to [0]
\]
has net phase $\pi$, and lies in $\Omega_-$, the collapse domain.
\end{definition}

\[
e^{i\pi} = -1 \quad \Rightarrow \quad [A] + [-A] = [0]
\]

\paragraph{Interpretation.}  
Inversion is half-rotation. It returns to the phase-opposite point on the circle. Such identities cannot persist; they annihilate.

---

\section{The Unit Circle as Coherence Map}

We now assign coherence signature components as angular mappings on the unit circle:

\begin{itemize}
    \item \textbf{Mass ($\Lambda$)}: total angular traversal required for return ($\theta_{\text{total}}$),
    \item \textbf{Symmetry ($X_\pi$)}: orientation along the phase axis,
    \item \textbf{Fragility ($X_\phi$)}: phase instability (angular jitter),
    \item \textbf{Reuse elevation ($X_\epsilon$)}: quantized return overtones.
\end{itemize}

\begin{definition}[Eulerian Phase Configuration]
The coherence state of identity $[A]$ is a point on the complex unit circle with angle $\theta([A])$, where:
\[
\theta([A]) := \frac{2\pi \Lambda([A])}{\Lambda_{\text{norm}}}
\]
and $\Lambda_{\text{norm}}$ is the reference latency corresponding to a full $2\pi$ symbolic return rotation (e.g., the latency of a photon identity).
\end{definition}

---

\section{Conclusion: Identity as Phase Closure}

Under the Eulerian interpretation, all persistent identities arise from **phase-locked coherence return**. They are not static—they are symbolic phase orbits. Their quantization, symmetry, and structural roles will now be derived as consequences of return path structure on this Euler circle.

In the next chapter, we will derive the uncertainty principle from the trade-off between return latency and angular fragility.
