\chapter{Rethinking the Postulates of Quantum Mechanics} \label{chapter:copenhagen}

The Copenhagen interpretation frames quantum theory in terms of probabilistic states, wavefunction collapse, and indeterminacy. These principles are treated as axiomatic—taken for granted as postulates of reality.

In SCM, none of these postulates are assumed. Instead, each quantum phenomenon arises from symbolic structure and coherence constraints on the Euler circle. This chapter maps each canonical postulate to its structural origin in SCM.

---

\section{The Classical Postulates}

Let us recall the core postulates of standard quantum mechanics:

\begin{enumerate}
    \item A system is fully described by a wavefunction $\psi(x)$.
    \item Observables are represented by Hermitian operators.
    \item Measurement collapses $\psi$ to an eigenstate.
    \item The outcome is probabilistic, given by $|\psi|^2$.
    \item Identical fermions obey exclusion.
    \item Systems evolve unitarily under Schrödinger dynamics.
\end{enumerate}

---

\section{The SCM Reformulation}

\begin{itemize}
  \item \textbf{Postulate 1: The Wavefunction}  
    → Replaced by symbolic return structure $\mathcal{R}([A])$ and phase sum $\Psi([A]) := \sum \alpha_i e^{i\theta_i}$  
    Wavefunctions are not fundamental—they are effort-weighted coherence projections.

  \item \textbf{Postulate 2: Operators and Observables}  
    → Replaced by symbolic return transformations $T: \Sigma^* \to \Sigma^*$ and coherence signature $\chi([A])$  
    Observable behavior is encoded in drift: $\partial \chi([A])$.

  \item \textbf{Postulate 3: Collapse}  
    → Emerges from reuse conflict and structural contradiction.  
    RuleEvolution prunes invalid return loops when $P([A]) \to P_{\text{crit}}$.

  \item \textbf{Postulate 4: Probability}  
    → Emerges from symbolic entropy weighting:  
    \[
    p_i := \frac{e^{-\Lambda(\mathcal{L}_i)}}{Z}
    \]  
    This is not a rule—it is the coherence distribution over return loops.

  \item \textbf{Postulate 5: Exclusion}  
    → Derived from contradiction minimization under reuse saturation.  
    If $\rho([A]) < \rho_c$, identity $[A]$ collapses.

  \item \textbf{Postulate 6: Unitary Evolution}  
    → Recast as topological phase closure:  
    \[
    \prod e^{i\theta} = e^{i\sum \theta} = 1 \quad \text{iff } \sum \theta = 2\pi n
    \]  
    Evolution is not wavefunction rotation—it is symbolic phase accumulation under coherence stability.
\end{itemize}

---

\section{What SCM Removes}

SCM eliminates:

- Observer-dependence,
- Measurement paradoxes,
- Wavefunction collapse as a mystery,
- Arbitrary postulates of state and evolution.

All of these are replaced by:

- Symbolic return structure (phase-closed loops),
- Drift pressure and contradiction minimization,
- RuleEvolution and reuse dynamics.

---

\section{Quantum Reality as Eulerian Symbolic Structure}

| Classical QM            | SCM Eulerian Framework                          |
|-------------------------|--------------------------------------------------|
| $\psi(x)$               | $\mathcal{R}([A])$, $\Psi([A])$                 |
| Collapse                | Coherence pruning under pressure                |
| Probabilities           | Entropy-weighted latency distribution           |
| Fermionic Exclusion     | Reuse saturation + contradiction minimization   |
| Measurement             | External reuse constraint                       |
| Entanglement            | Phase-locked return dependencies                |

SCM does not interpret quantum theory. It **replaces it** with a symbolic, phase-based model that derives quantum behavior from structural necessity.

---

\section{Conclusion: From Interpretation to Origin}

What began as an effort to interpret quantum mechanics has culminated in a structural derivation. The mysteries of $\psi$, collapse, entanglement, and exclusion all reduce to **coherence logic on the Euler circle**.

There are no observers. No paradoxes. No axioms. Just return.

With this Eulerian foundation in place, we are now ready to apply it at cosmological scale. In Volume IX, we will explore how symbolic structure, coherence gradients, and phase topology give rise to the large-scale structure of the universe.
