\chapter{Quantization from Symbolic Closure} \label{chapter:quantization}

In traditional quantum mechanics, quantization is introduced through boundary conditions on differential wave equations, such as the Schrödinger equation. In SCM, quantization emerges not from wavefunctions but from symbolic return constraints: identities may only persist if their phase-rotated return structure closes coherently on the Euler circle.

This chapter shows that quantization arises as a structural condition for coherence, not as an imposed numerical rule.

\section{Symbolic Return Closure}

Let $[A] \in \Omega_3$ be a resolved identity with return loop $\mathcal{L}([A])$. Its coherence closure requires:
\[
\prod_{T_i \in \mathcal{L}} e^{i\theta_i} = 1 \quad \Rightarrow \quad \sum_i \theta_i = 2\pi n,\quad n \in \mathbb{Z}
\]

\begin{definition}[Quantized Phase Closure]
An identity $[A]$ is quantized if its return path corresponds to a full phase rotation of $2\pi n$ radians.
\end{definition}

These allowed return paths define the discrete phase bands of symbolic stability.

---

\section{Quantized Latency and Discrete Mass}

From Chapter 1, we define the symbolic angle:
\[
\theta([A]) := \frac{2\pi \Lambda([A])}{\Lambda_{\text{norm}}}
\quad \Rightarrow \quad \Lambda([A]) = n \Lambda_0
\]

\begin{proposition}[Latency Quantization]
If $\theta([A]) = 2\pi n$, then $\Lambda([A]) = n \Lambda_0$, where $\Lambda_0$ is the minimal latency corresponding to one full coherence rotation.
\end{proposition}

\paragraph{Interpretation.}  
Mass (via latency) is inherently quantized: only symbolic return loops that satisfy $n \in \mathbb{N}$ are coherence-permitting.

This defines a **symbolic mass ladder**:
\[
m_n := \Lambda_n = n \Lambda_0
\]

---

\section{Eigenstates and Symbolic Stationarity}

Let return loops define symbolic transformations $T_i$ that rotate identity by $\theta_i$.

\begin{definition}[Symbolic Eigenstate]
An identity $[A]$ is an eigenstate if all permitted return paths $\mathcal{L}_k([A])$ satisfy:
\[
\sum_{T_i \in \mathcal{L}_k} \theta_i = 2\pi n_k \quad \text{(with constant $n_k$)}
\]
\end{definition}

In this sense, eigenstates are **return-locked structures**—they cannot deform under symbolic drift. Their phase is stationary modulo $2\pi$.

---

\section{Forbidden States and Phase Discontinuity}

Any identity with $\sum \theta_i \notin 2\pi \mathbb{Z}$ cannot satisfy return closure:
\[
e^{i \sum \theta_i} \ne 1
\quad \Rightarrow \quad [A] \notin \Omega_3
\]

These identities:
- Cannot resolve,
- Accumulate contradiction pressure,
- Collapse or transform under RuleEvolution.

\begin{definition}[Forbidden Phase Configuration]
An identity is structurally unstable if its return phase $\theta$ is non-integral modulo $2\pi$.
\end{definition}

---

\section{Comparison to Wave-Based Quantization}

| Classical QM                      | SCM Euler View                                   |
|----------------------------------|--------------------------------------------------|
| Wavefunction $\psi(x)$           | Phase-rotated return structure $e^{i\theta}$     |
| Boundary conditions              | Return closure on Euler circle                   |
| Quantized eigenvalues            | Discrete latency bands $\Lambda_n = n\Lambda_0$ |
| Forbidden energy levels          | Non-integral phase returns                       |

---

\section{Conclusion: Quantization as Return Locking}

Quantization in SCM is not imposed—it is the result of **return loop closure under angular coherence**. Identities may only persist if they lock to stable phase rotation cycles on the unit circle.

In the next chapter, we will show that when identities possess multiple valid but unresolved return paths, they exhibit superposition behavior until reuse pressure forces collapse.
