\chapter{Superposition as Coherence Ambiguity} \label{chapter:superposition}

In standard quantum mechanics, superposition describes a state where a system exists in multiple possible configurations simultaneously, collapsing to a definite outcome only upon measurement. In SCM, superposition arises when an identity possesses multiple return-complete structures that are **coherence-compatible but unresolved**. This condition reflects ambiguity in symbolic return—not indeterminacy of state.

\section{Structural Basis of Superposition}

Let $[A]$ be a coherence-permitting identity with multiple distinct return loops:
\[
\mathcal{R}([A]) = \{\mathcal{L}_1, \mathcal{L}_2, \dots, \mathcal{L}_n\}
\]

If each $\mathcal{L}_i$ satisfies symbolic return closure:
\[
\sum_{T_j \in \mathcal{L}_i} \theta_j = 2\pi n_i,\quad n_i \in \mathbb{Z}
\]
then $[A]$ is **return-resolvable** via multiple coherence paths.

\begin{definition}[Symbolic Superposition]
An identity $[A] \in \Omega_3$ is in superposition if:
\begin{itemize}
  \item $|\mathcal{R}([A])| > 1$,
  \item Each $\mathcal{L}_i \in \mathcal{R}([A])$ satisfies phase closure independently,
  \item No reuse constraint or external coherence field currently selects among them.
\end{itemize}
\end{definition}

\paragraph{Interpretation.}  
Superposition is not indeterminacy. It is **coherence degeneracy**—a structurally unresolved identity with multiple allowed phase-closed return configurations.

---

\section{Symbolic Collapse as Reuse Resolution}

Let $\partial \chi([A])$ be the drift of $[A]$ under reuse interaction. Collapse occurs when contradiction pressure selects a dominant return path:
\[
\mathcal{L}_\text{selected} := \arg\min_{\mathcal{L}_i} \mathcal{C}([\mathcal{L}_i])
\]

\begin{definition}[Collapse Criterion]
Collapse of superposition occurs when coherence pressure or reuse saturation forces resolution:
\[
P([A]) := \frac{L([A])}{C([A])} \to P_{\text{crit}}
\]
\end{definition}

When this condition is met, the system selects a single return path and all competing coherence configurations are pruned by RuleEvolution.

---

\section{Eulerian Representation of Superposition}

Let each return path be represented by a closed Euler rotation:
\[
\mathcal{L}_i([A]) \sim e^{i\theta_i}
\]

Then superposition is represented by a **coherence sum**:
\[
\Psi([A]) := \sum_{i=1}^n \alpha_i e^{i\theta_i}
\]

This is not a wavefunction in the classical sense—but a **weighted symbolic projection of return options**.

\begin{itemize}
  \item The amplitudes $\alpha_i$ represent symbolic entropy weighting (effort-discounted),
  \item The phases $\theta_i$ are the closure angles of each path,
  \item Collapse corresponds to the pruning of all but one term in $\Psi([A])$.
\end{itemize}

---

\section{Measurement as Structural Selection}

In SCM, "measurement" does not collapse a wavefunction. It imposes **external reuse constraints** on the return graph, forcing selection of a specific resolution path.

Let $\Gamma$ be the extended return graph with:
- Internal symbolic structure (loop candidates),
- External coherence field (reuse environment).

Then:

\[
\text{Measurement} := \text{external reduction of } \Gamma \to \mathcal{L}_j
\]

\paragraph{Interpretation.}  
Collapse is symbolic, not epistemic. Measurement is not "seeing"—it is **forcing reuse** through one of the coherence paths, deleting the rest.

---

\section{Superposition Without Probability}

In contrast to standard quantum theory, SCM does not require postulating a Born rule.

Return probabilities emerge naturally from entropy-weighted effort distributions:
\[
p_i := \frac{e^{-\Lambda(\mathcal{L}_i)}}{Z},\quad Z = \sum_j e^{-\Lambda(\mathcal{L}_j)}
\]

This matches the functional form of quantum probability amplitudes—but arises from structural stability preferences, not axioms.

---

\section{Conclusion: Superposition as Path Ambiguity}

Superposition is not the coexistence of states—it is the **non-resolution of multiple structurally valid return paths**. Collapse occurs when coherence pressure forces commitment. Probabilities reflect return effort costs, not epistemic uncertainty.

In the next chapter, we will examine how multiple identities can **share phase-locked return paths**, producing nonlocal coherence and symbolic entanglement.
