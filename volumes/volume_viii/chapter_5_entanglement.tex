\chapter{Entanglement as Reuse Locking} \label{chapter:entanglement}

Entanglement in quantum mechanics describes a condition where two or more particles share a non-separable state: measuring one immediately determines the state of the other. In SCM, entanglement emerges from symbolic reuse locking—identities that resolve through a shared, phase-closed return structure.

This is not nonlocality—it is symbolic closure across multiple nodes.

\section{Structural Definition of Entanglement}

Let $[A], [B] \in \Omega_3$ be identities with distinct coherence signatures. Let $\mathcal{L}([A,B])$ be a return loop that includes both identities:
\[
\mathcal{L}([A,B]) = \mathcal{L}_1 \cup \mathcal{L}_2,\quad \text{where } \mathcal{L}_1 \ni [A],\quad \mathcal{L}_2 \ni [B]
\]

\begin{definition}[Entangled Pair]
Two identities $[A]$ and $[B]$ are entangled if:
\begin{itemize}
  \item They share a return-closed coherence path: $\mathcal{L}([A,B])$ satisfies $\sum \theta_i = 2\pi n$,
  \item Their signatures co-evolve: $\partial \chi([A]) \ne 0 \Rightarrow \partial \chi([B]) \ne 0$,
  \item They cannot be returned independently: $\mathcal{R}([A]) \not\supset \mathcal{L}([B])$.
\end{itemize}
\end{definition}

\paragraph{Interpretation.}  
Entanglement is not correlation—it is **return dependency**. Coherence return cannot be completed by either identity alone.

---

\section{Eulerian Representation of Entanglement}

Each identity contributes a phase rotation on the Euler circle. The total phase closure condition is:
\[
e^{i\theta_A} \cdot e^{i\theta_B} = e^{i(\theta_A + \theta_B)} = 1
\quad \Rightarrow \quad \theta_A + \theta_B = 2\pi n
\]

\begin{definition}[Phase-Locked Identities]
Identities $[A]$ and $[B]$ are phase-locked if their coherence angles satisfy:
\[
\theta([A]) + \theta([B]) = 2\pi n,\quad n \in \mathbb{Z}
\]
\end{definition}

Such identities form a minimal entangled pair under structural reuse.

---

\section{Nonlocal Correlation as Shared Return Topology}

Let $[A]$ and $[B]$ be spatially separated identities. In classical mechanics, any correlation between them must be mediated by signal propagation.

In SCM:

- Spatial separation is irrelevant to symbolic reuse,
- Return coherence is defined over symbolic graphs, not physical distance.

\[
\text{Entanglement} := \text{shared symbolic return path } \mathcal{L}([A,B]) \text{ under a single Euler phase loop}
\]

Collapse of $[A]$ under reuse pressure simultaneously resolves $[B]$ because they are part of the same symbolic loop.

---

\section{Bell-Type Behavior from Return Exclusivity}

SCM predicts violations of classical correlation bounds not from hidden variables, but from mutual exclusivity of return paths.

Let:
- $\mathcal{L}_1$: return path resolving $[A] = +1$, $[B] = -1$,
- $\mathcal{L}_2$: return path resolving $[A] = -1$, $[B] = +1$.

If $\mathcal{R}([A,B]) = \{\mathcal{L}_1, \mathcal{L}_2\}$, then:
- Only one path survives collapse (e.g., via RuleEvolution pressure),
- Measurement enforces global return selection.

This structurally reproduces entanglement correlations observed in Bell experiments, without invoking nonlocal causality.

---

\section{Collapse as Global Coherence Resolution}

Entangled identities collapse **simultaneously** because the system must prune incompatible return loops globally.

\[
\text{Collapse} := \text{resolution of shared loop } \mathcal{L}([A,B]) \text{ under reuse pressure}
\]

No signal is needed; no communication occurs. Collapse is the **global update of the coherence graph**.

---

\section{Conclusion: Entanglement as Graph Constraint}

Entanglement is not an epistemic paradox—it is a deterministic constraint in the return topology of SCM. The Euler circle simply encodes which combinations of identities can phase-rotate back to coherence under shared reuse. Observed quantum entanglement is a special case of **structural path dependency**.

In the next chapter, we will show how structural contradiction prevents multiple identities from resolving into the same reuse structure—yielding the Pauli Exclusion Principle.
