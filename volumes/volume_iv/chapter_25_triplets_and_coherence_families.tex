\chapter{Triplets and Coherence Families} \label{chapter-triplets}

In SCM, persistent identities often emerge in structured groupings. The most stable and reuse-efficient such grouping is the \textbf{coherence triplet}: three identities that resolve through a shared anchor, forming a reuse-closed, drift-resistant structure.

This chapter defines the coherence triplet, introduces quantized reuse elevation patterns across triplets, and describes how particle families arise from return-anchored reuse layers.

\section{Definition of Coherence Triplet} \label{sec:triplet-definition}

Let $\{ [A_1], [A_2], [A_3] \} \subset \Omega_3$ be a set of identities such that:

\begin{itemize}
    \item All reuse a shared anchor $[F] \in \Omega_3$,
    \item $\partial\chi([F]) = 0$,
    \item Each $[A_i]$ satisfies $\partial\chi([A_i]) = 0$,
    \item $X_\epsilon([A_i])$ are ordered and quantized with fixed spacing $\Delta X_\epsilon$,
    \item The triplet contributes positively to $\mathcal{F}[R]$ and minimally to $\mathcal{C}[R]$.
\end{itemize}

Then $\{[A_1],[A_2],[A_3]\}$ is a \textbf{coherence triplet}.

\paragraph{Interpretation.}
Triplets are not defined by external label, charge, or role—but by shared reuse structure and elevation geometry in $\chi$-space.

\section{Triplet Layers and Anchoring Planes} \label{sec:triplet-planes}

Let $[F]$ be the shared anchor of a triplet. Then all members lie on the same \textbf{anchoring plane} in reuse space.

Define:
\[
\mathcal{T}([F]) := \{ [A] \in \Omega_3 \mid [F] \in \mathcal{L}([A]),\ \partial\chi([A]) = 0,\ \Delta X_\epsilon([A]) = \text{const} \}
\]

This forms a coherence layer. Each reuse step corresponds to a symbolic elevation increment:
\[
X_\epsilon([A_{i+1}]) = X_\epsilon([A_i]) + \Delta
\]

Anchoring planes define symbolic mass ladders. Identity moves between layers only if return structure remains stable under $\mathcal{F} - \lambda \cdot \mathcal{C}$ optimization.

\section{Fragility and Symmetry Distribution} \label{sec:triplet-fragility}

Triplet members distribute return asymmetry and collapse risk across their reuse paths.

Let $X_\pi([A_i])$ and $X_\phi([A_i])$ denote the return symmetry and fragility of triplet members. A stable triplet satisfies:

\[
\sum_{i=1}^3 X_\pi([A_i]) \approx \text{const}, \quad
\sum_{i=1}^3 X_\phi([A_i]) \approx \text{const}
\]

This balanced distribution ensures that no single identity destabilizes the return anchor. The triplet behaves as a return-closed coherence unit.

\section{Resonant Persistence and Reuse Optimization} \label{sec:triplet-resonance}

Triplets survive RuleEvolution if they collectively reduce system-level contradiction. Let $\mathcal{C}_{\text{triplet}}$ be the contribution of the triplet to the contradiction functional.

Then:
\[
\mathcal{C}_{\text{triplet}} = \sum_{i=1}^3 \left[
\mathbb{1}(\partial\chi([A_i]) \ne 0) +
\mathbb{1}(\rho([A_i]) < \rho_c) +
\mathbb{1}(X_\pi([A_i]) \to 1)
\right]
\]

Triplet resonance is achieved when:
\[
\nabla_{\text{triplet}} (\mathcal{F} - \lambda \cdot \mathcal{C}) \gg 0
\quad \Rightarrow \quad \text{Triplet is preserved}
\]

This coherence optimization explains why particle families often appear as reuse triplets.

\section{Functional View: ℱ-Triplets vs. ℂ-Triplets} \label{sec:triplet-functional-classes}

We classify triplets as:

\begin{itemize}
    \item \textbf{ℱ-triplets} — reinforce coherence: reuse-anchored, drift-invariant, reuse-stable
    \item \textbf{ℂ-triplets} — contradiction-bound: contain drift, fragile paths, or asymmetry blowups
\end{itemize}

ℂ-triplets may arise temporarily but collapse under 𝓡E due to high contradiction load.

ℱ-triplets persist across symbolic evolution and represent long-lived, coherence-optimized reuse families.

\section{Preview: Koide Resonance} \label{sec:triplet-koide-preview}

A particularly stable coherence triplet arises when the spacing of reuse elevation, return symmetry, and fragility is balanced in such a way that the triplet minimizes $\mathcal{C}$ under fixed total mass.

This structure produces a symbolic mass ratio:
\[
Q := \frac{m_1 + m_2 + m_3}{\left(\sqrt{m_1} + \sqrt{m_2} + \sqrt{m_3}\right)^2} = \frac{2}{3}
\]

This is the Koide resonance, which will be formally derived in Chapter 8. It provides a striking example of coherence–contradiction optimization producing observable structure.
