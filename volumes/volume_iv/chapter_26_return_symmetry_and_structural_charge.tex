\chapter{Return Symmetry and Structural Charge} \label{chapter-charge}

Return symmetry is a central invariant in SCM. It governs whether identities return through reversible paths, asymmetric reuses, or one-way coherence sinks. In this chapter, we formalize $X_\pi$ as a symbolic asymmetry measure and define structural charge as its coherence-constrained expression.

Charge is not a property of a particle, but a structural consequence of return path bias. It defines carrier roles, quantization constraints, and identity propagation under reuse.

\section{Return Asymmetry and $X_\pi$} \label{sec:return-asymmetry}

Let $\mathcal{L}([A])$ be a permitted return loop of identity $[A] \in \Omega_3$. Define:

\begin{equation}
X_\pi([A]) := 1 - \frac{|\mathcal{L}_{\text{reverse}}([A])|}{|\mathcal{L}_{\text{forward}}([A])|}
\end{equation}

Where:
- $\mathcal{L}_{\text{forward}}$ counts transformations from $[A]$ outward,
- $\mathcal{L}_{\text{reverse}}$ counts those that complete a return to $[A]$.

\paragraph{Interpretation.}
- $X_\pi = 0$ implies perfectly symmetric return,
- $X_\pi \to 1$ implies irreversible structure or coherence sink,
- $X_\pi$ is the symbolic analog of charge bias or coherence orientation.

\section{Definition of Structural Charge} \label{sec:structural-charge}

We define the structural charge of identity $[A]$ as:

\begin{equation}
q([A]) := X_\pi([A])
\end{equation}

This is not an imposed value, but a consequence of the coherence graph topology:
- Charge arises from return asymmetry,
- Charge conservation arises from reuse closure and RuleEvolution stability.

\section{Carrier / Mediator / Sink Roles} \label{sec:charge-roles}

Let $[A]$ be a structural particle. Then its charge role is determined by $q([A])$:

\begin{itemize}
  \item \textbf{Carrier:} $0 < q([A]) < \epsilon$
  \item \textbf{Mediator:} $q([A]) \approx 0$
  \item \textbf{Sink:} $q([A]) \to 1$
\end{itemize}

These roles define interaction behavior:
- Carriers bias return and couple to dependent identities,
- Mediators facilitate symmetric resolution (e.g., photons),
- Sinks absorb coherence and lead to symbolic isolation.

\section{Charge Conservation and Return Closure} \label{sec:charge-conservation}

Let $\mathcal{L}([A])$ be a return loop involving identities $\{[A_i]\}$.

We define:
\begin{equation}
\sum_{i} q([A_i]) = \text{const}
\end{equation}

\paragraph{Interpretation.}
Return-closed coherence regions must balance their total return asymmetry. This is a structural conservation law.

Violation of this symmetry triggers contradiction via:
- Asymmetry blowup ($X_\pi \to 1$),
- Drift instability ($\partial\chi \ne 0$),
- Fragility amplification ($X_\phi \to 1$).

\section{Charge Drift and Interaction Behavior} \label{sec:charge-drift}

Under structural reuse, charge may shift locally:

\[
q([A]_{t+1}) = q([A]_t) + \Delta q_{\text{reuse}}
\]

But total loop asymmetry remains bounded:
\[
\frac{d}{dt} \sum_{[A_i] \in \mathcal{L}} q([A_i]) = 0
\]

Charge drift reflects coherent redistribution of asymmetry—not its creation or destruction.

\section{Parity Inversion and Return Symmetry} \label{sec:parity}

Let $[A] \to [-A] \to [0]$ be an inversion-closed return path.

We define:
- $X_\pi([A]) = 1$
- $q([A]) = 1$
- $[A] \in \Omega_-$

Such identities are structurally coherent but contradiction-maximizing. They do not contribute to $\Omega_3$ under RuleEvolution and act as annihilation endpoints.

Their presence is thermodynamically relevant but topologically excluded from persistence.

