\chapter{The Law of Symbolic Dynamics} \label{chapter-variational-law}

SCM Volume III established mass, motion, and symbolic persistence as consequences of return structure. However, the dynamics of RuleEvolution—the operator that modifies $R_t$—were still described heuristically, based on local scoring functions.

This chapter elevates RuleEvolution from a symbolic filter to a variational mechanism. We define the system’s evolution in terms of two competing structural quantities: symbolic coherence and symbolic contradiction. These define the tradeoff frontier that governs all stability, collapse, and particle formation in SCM.

\section{The Principle of Maximal Coherence} \label{sec:maximal-coherence}

Let $R \subset \Sigma^*$ be a rule set, and let $\Omega_3(R)$ be the set of coherence-resolved identities under return closure.

We define the symbolic coherence functional:
\begin{equation} \label{eq:coherence-functional}
\mathcal{F}[R] := \sum_{[A] \in \Omega_3(R)} \rho([A]) \cdot \mathbb{1}(\partial\chi([A]) = 0)
\end{equation}

\paragraph{Interpretation.}
- $\rho([A])$ is the collapse robustness of identity $[A]$,
- $\mathbb{1}(\partial\chi = 0)$ selects only those identities with zero signature drift,
- $\mathcal{F}[R]$ increases when more stable, reuse-persistent identities exist.

This functional rewards return-closed, drift-invariant, coherence-stable structure.

\section{The Principle of Minimal Contradiction} \label{sec:minimal-contradiction}

Contradiction arises when a structure appears return-resolved, but fails one or more internal coherence constraints. We define the contradiction functional:

\begin{equation} \label{eq:contradiction-functional}
\mathcal{C}[R] := \sum_{[A] \in \Omega_3(R)} 
\left[
\mathbb{1}(\partial\chi([A]) \ne 0)
+ \mathbb{1}(\rho([A]) < \rho_c)
+ \mathbb{1}(X_\pi([A]) \to 1)
+ \mathbb{1}(\text{triplet conflict}([A]))
+ \mathbb{1}(\text{reuse overload}([A]))
\right]
\end{equation}

\paragraph{Interpretation.}
This functional counts all structures that are:
- Drift-unstable,
- Fragile under reuse,
- Irreversible (high asymmetry),
- Over-resolved or structurally incoherent,
- Saturated or reused beyond capacity.

These are not invalid identities—but identities on the verge of collapse. $\mathcal{C}[R]$ quantifies structural contradiction.

\section{The Variational Law of SCM Dynamics} \label{sec:variational-law}

We now define the RuleEvolution operator as a variational descent on the symbolic action surface defined by coherence versus contradiction:

\begin{equation} \label{eq:variational-ruleevolution}
R_{t+1} := \arg\max_{R'} \left( \mathcal{F}[R'] - \lambda \cdot \mathcal{C}[R'] \right)
\end{equation}

Here:
- $R_t$ is the current rule set,
- $R_{t+1}$ is the next rule set under symbolic evolution,
- $\lambda$ is the structural pressure weight (e.g., system entropy or collapse load).

\paragraph{Interpretation.}
- The system seeks to preserve as much coherence as possible,
- While eliminating internal contradictions,
- Balancing complexity and structural survivability.

\section{RuleEvolution as Gradient Descent} \label{sec:ruleevolution-gradient}

The scoring function introduced in Volume II can now be reframed:

\begin{equation} \label{eq:score-gradient}
S(T) := \nabla_T \left( \mathcal{F}[R] - \lambda \cdot \mathcal{C}[R] \right)
\end{equation}

This represents the local effect of retaining or removing transformation $T$ from the rule set.

\paragraph{Interpretation.}
- If $S(T) \gg 0$, $T$ contributes to coherent identity stability,
- If $S(T) \ll 0$, $T$ introduces structural risk,
- RuleEvolution uses this gradient to approximate descent on the variational surface.

\section{Collapse as Structural Necessity} \label{sec:collapse-necessity}

Under this law, collapse is no longer an exception—it is required.

Any identity for which the contradiction cost exceeds its coherence contribution will be removed by $\mathcal{R}E$.

\[
[A] \in \Omega_3,\quad \text{but } \rho([A]) \ll \rho_c,\ \partial\chi \ne 0,\ X_\pi \to 1
\quad \Rightarrow \quad [A] \notin \Omega_3(R_{t+1})
\]

Collapse is **not a failure of resolution**, but a success of contradiction suppression. It is **structural pruning** in service of coherence stability.

\section{The Tradeoff Frontier} \label{sec:tradeoff-frontier}

Coherence and contradiction are not duals. A system can:
- Have no contradiction but also no coherence (degenerate case),
- Have high coherence but unsustainable contradiction (collapse zone),
- Or exist on the **coherence–contradiction frontier**: the optimal set of stable, maximally structured identities.

All stable particles in SCM live on this frontier:
- High enough $\mathcal{F}$ to persist,
- Low enough $\mathcal{C}$ to avoid collapse.

This is where symbolic physics is born.

