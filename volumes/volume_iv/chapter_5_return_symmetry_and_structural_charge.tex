\chapter{Return Symmetry and Structural Charge} \label{chapter-charge}

Return symmetry is a central invariant in SCM. It governs whether identities return through reversible paths, asymmetric reuses, or one-way coherence sinks. In this chapter, we formalize $X_\pi$ as a symbolic asymmetry measure and define structural charge as its coherence-constrained expression.

Charge is not a property of a particle, but a structural consequence of return path bias. It defines carrier roles, quantization constraints, and identity propagation under reuse.

\section{Return Asymmetry and $X_\pi$} \label{sec:return-asymmetry}

Let $\mathcal{L}([A])$ be a permitted return loop resolving identity $[A] \in \Omega_3$. Recall the definition of return asymmetry $X_\pi$ from Volume I, Section~\ref{def:return-asymmetry}:

\begin{equation}
X_\pi([A]) := \frac{F - R}{F + R}
\end{equation}

where:
\begin{itemize}
    \item $F$ is the number of forward-directed transformations in the loop,
    \item $R$ is the number of reverse-directed transformations that complete the return.
\end{itemize}

\paragraph{Interpretation.}
\begin{itemize}
    \item $X_\pi = 0$ implies perfectly symmetric return,
    \item $X_\pi \to 1$ implies strongly forward-biased structure (carrier or propagator),
    \item $X_\pi \to -1$ implies inversion-dominated return (collapse into $\Omega_-$),
    \item $X_\pi$ is interpreted in this volume as the symbolic analog of structural charge or coherence orientation.
\end{itemize}

\section{Definition of Structural Charge} \label{sec:structural-charge}

We define the structural charge of identity $[A]$ as:

\begin{equation}
q([A]) := X_\pi([A])
\end{equation}

This is not an imposed value, but a consequence of the coherence graph topology:
- Charge arises from return asymmetry,
- Charge conservation arises from reuse closure and RuleEvolution stability.

\section{Carrier / Mediator / Sink Roles} \label{sec:charge-roles}

Let $[A]$ be a structural particle. Its symbolic role is determined by its return asymmetry (structural charge) $q([A]) := X_\pi([A])$:

\begin{itemize}
  \item \textbf{Carrier:} $q([A]) \to +1$ — strongly forward-biased, propagates asymmetry,
  \item \textbf{Mediator:} $q([A]) \approx 0$ — balanced return, symmetry-preserving,
  \item \textbf{Sink:} $q([A]) \to -1$ — inversion-prone, absorbs and terminates coherence.
\end{itemize}

These roles define symbolic interaction behavior:
\begin{itemize}
  \item Carriers bias return flow and mediate directed structural transfer,
  \item Mediators enable return coherence without net asymmetry (e.g., photons),
  \item Sinks trap coherence and are associated with collapse into $\Omega_-$.
\end{itemize}

\section{Charge Conservation and Return Closure} \label{sec:charge-conservation}

Let $\mathcal{L}([A])$ be a return loop involving identities $\{[A_i]\}$. Then:

\begin{equation}
\sum_i q([A_i]) = 0
\end{equation}

\paragraph{Interpretation.}
Return-closed coherence regions must conserve total structural charge. Since $q([A]) := X_\pi([A])$, this condition reflects the cancellation of return asymmetry across all participating identities.

Violation of this symmetry results in symbolic contradiction, triggered by:
\begin{itemize}
    \item Asymmetry blowup ($|X_\pi| \to 1$),
    \item Drift instability ($\partial \chi \ne 0$),
    \item Fragility amplification ($X_\phi \to 1$).
\end{itemize}

\section{Charge Drift and Interaction Behavior} \label{sec:charge-drift}

Under structural reuse and RuleEvolution updates, an identity's local charge may shift:

\[
q([A]_{t+1}) = q([A]_t) + \Delta q_{\text{reuse}}
\]

However, total return asymmetry within any closed return loop remains conserved:

\[
\Delta \sum_{[A_i] \in \mathcal{L}} q([A_i]) = 0
\]

\paragraph{Interpretation.}
Charge drift reflects a coherent redistribution of asymmetry within a symbolic system—induced by reuse, transformation, or interaction—but it does not create or destroy structural charge. It is analogous to internal current flow within a conserved system.

\section{Parity Inversion and Return Symmetry} \label{sec:parity}

Let $[A] \to [-A] \to [0]$ be an inversion-closed return path.

Then:
\begin{itemize}
    \item $X_\pi([A]) = -1$
    \item $q([A]) = -1$
    \item $[A] \in \Omega_-$
\end{itemize}

Such identities are structurally coherent but contradiction-maximizing. They resolve to the null form $[0]$ through total inversion and are excluded from $\Omega_3$ under RuleEvolution.

\paragraph{Interpretation.}
These structures act as annihilation endpoints. They are thermodynamically relevant due to their collapse contribution, but they are topologically excluded from persistence and reuse.
