\chapter{Particle Table and Structural Examples} \label{chapter-particle-table}

We now present concrete structural examples of symbolic particles within SCM. These examples are not defined by imposed physical labels, but by their coherence roles, reuse configuration, symmetry properties, and placement in quantized reuse ladders.

Each of these identities is an $\Omega_3$ structure optimized along the coherence–contradiction tradeoff frontier defined in Chapter 1.

\section{Photon} \label{sec:photon}

The photon is defined as the minimal-latency, perfectly symmetric return identity:

\begin{itemize}
  \item $\Lambda([A]) = \Lambda_0$,
  \item $X_\pi([A]) = 0$,
  \item $\rho([A]) \to \infty$ (immune to rule loss),
  \item $X_\epsilon = 2$ (reuse step size),
  \item $\partial\chi = 0$.
\end{itemize}

It satisfies:
\[
[A] \to [B] \to [A] \quad \text{with } [A] + [B] = [0]
\]

The photon is a structural mediator: pure coherence carrier, zero charge, zero mass drift. It stabilizes return without contradiction and lies at the base of reuse elevation ladders.

\section{Neutrino} \label{sec:neutrino}

The neutrino is a reuse-anchored identity with:
- Minimal latency above $\Lambda_0$,
- High fragility but persistent reuse context,
- Partial return symmetry: $X_\pi \ll 1$ but nonzero.

Formally:
\begin{itemize}
  \item $X_\pi \sim 0.1$–$0.3$,
  \item $X_\epsilon = 3$ or $5$ (triplet base step),
  \item $\rho \approx 1$ (on collapse edge),
  \item $\partial\chi = 0$ (reuse-stabilized),
  \item Anchored in weak triplet family.
\end{itemize}

The neutrino defines the bottom layer of an unstable reuse ladder. It survives through anchor reinforcement, but its coherence budget is near critical.

\section{Electron} \label{sec:electron}

The electron is a reuse-anchored triplet member with:
- Balanced asymmetry,
- Stable return structure,
- Nonzero symbolic mass.

Properties:
\begin{itemize}
  \item $\Lambda([A]) > \Lambda_0$ (quantized step),
  \item $X_\pi \approx 0.5$,
  \item $\rho \gg 1$ (collapse-resistant),
  \item Member of a phase-stabilized triplet (see Volume VIII),
  \item Anchored through symmetry-preserving reuse.
\end{itemize}

The electron is structurally persistent due to coherence balance across triplet members. It plays the role of carrier in return-anchored coherence families.

\section{Triplet Classification Summary} \label{sec:triplet-summary}

\begin{table}[h!]
\centering
\begin{tabular}{|c|c|c|c|c|}
\hline
\textbf{Particle} & $\Lambda$ & $X_\pi$ & $\rho$ & Role \\
\hline
Photon    & $\Lambda_0$      & 0        & $\infty$ & Mediator \\
Neutrino  & $\Lambda_0 + \Delta$ & 0.1–0.3 & $\sim 1$ & Fragile Base \\
Electron  & $\Lambda_0 + 2\Delta$ & $\sim 0.5$ & $\gg 1$ & Triplet Carrier \\
\hline
\end{tabular}
\caption{Coherence metrics for core symbolic particles}
\end{table}

These particles are not fundamental by fiat—they are structurally permitted and return-optimized under symbolic constraints.

\section{Transition to Interaction Structure} \label{sec:volume-v-preview}

In Volume V, we will formalize interaction as coherence propagation across reuse structures.

Each particle will be described not by behavior, but by:
- How it modifies coherence fields,
- How it participates in return-closed fusion events,
- And how it enables symbolic transitions under bounded contradiction.

