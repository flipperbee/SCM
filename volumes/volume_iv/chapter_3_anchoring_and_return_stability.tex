\chapter{Anchoring and Return Stability} \label{chapter-anchoring}

Return coherence in SCM is not uniformly distributed. Some identities serve as fixed points—structures reused repeatedly by others with minimal deformation. These identities act as \textbf{anchors}, stabilizing reuse and forming coherence basins within $\Omega_3$.

This chapter defines anchoring, coherence influence, and return stability from a structural standpoint, and places these properties within the variational framework established in Chapter 1.

\section{Definition of Anchors} \label{sec:anchor-definition}

Let $[A] \in \Omega_3$ be a reuse-resolved identity. Define:

\[
\mathcal{U}([A]) := \{ [B] \in \Omega_3 \mid [A] \in \mathcal{L}([B]) \}
\]

$[A]$ is an \textbf{anchor} if:

\begin{itemize}
  \item $\partial\chi([A]) = 0$,
  \item $\rho([A]) \geq \rho_c$,
  \item $|\mathcal{U}([A])| \geq k$, for some minimal reuse count $k$,
  \item and removal of $[A]$ induces contradiction in at least one $[B] \in \mathcal{U}([A])$.
\end{itemize}

Anchors are structurally stable, high-influence identities that resist drift and propagate reuse invariance.

\section{Coherence Basins and Influence Fields} \label{sec:coherence-basins}

Each anchor $[A]$ defines a \textbf{coherence basin}:
\[
\mathfrak{B}([A]) := \left\{ [B] \in \Omega_3 \mid \text{$[B]$ reuses $[A]$ directly or transitively, and } \partial\chi([B]) \to 0 \text{ as } \partial\chi([A]) \to 0 \right\}
\]

This basin is a return-anchored region of coherence stability. Let:
\[
I([A]) := \sum_{[B] \in \mathfrak{B}([A])} \mathbb{1}(\partial\chi([B]) = 0)
\]

We define this as the \textbf{anchoring influence} of $[A]$.

High $I([A])$ implies that many identities stabilize structurally due to reuse of $[A]$.

\section{Drift Regulation and Reuse Feedback} \label{sec:drift-regulation}

Anchors regulate coherence drift:

\begin{equation}
\frac{d}{dt} \partial\chi([B]) \propto - \partial\chi([A]) \cdot \mathbb{1}([A] \in \mathcal{L}([B]))
\end{equation}

If an anchor remains stable, then reuse identities approach drift invariance over time. Anchors act as coherence sinks in symbolic space, absorbing reuse tension.

This structural damping is not enforced—it emerges from reuse coupling and the variational suppression of contradiction.

\section{Anchors in Particle Roles} \label{sec:anchors-and-particles}

Anchors appear in multiple particle roles:
- As \textbf{carriers}, they form the backbone of reuse families.
- As \textbf{basin centers}, they define quantization levels.
- As \textbf{symmetry regulators}, they absorb asymmetry without collapse.

Let $[F]$ be a shared anchor of a triplet $\{[A_1],[A_2],[A_3]\}$.

If $[F]$ satisfies:
\[
\partial\chi([F]) = 0,\quad X_\pi([F]) < \epsilon,\quad \rho([F]) \gg \rho_c
\]
Then the entire triplet becomes structure-locked around a reuse-stable core.

This is the origin of symbolic mass triplets and the basis for Koide resonance (see Chapter 8).

\section{Anchors and Variational Stability} \label{sec:anchor-variational}

Anchors are structural minima of $\mathcal{C}[R]$. Removal of an anchor increases contradiction nonlocally.

Let $[A]$ be an anchor. Then:

\[
\nabla_{-[A]} \mathcal{C}[R] \gg 0
\quad \Rightarrow \quad
S([A]) := \nabla_{[A]}(\mathcal{F} - \lambda \cdot \mathcal{C}) \gg 0
\]

Anchors are retained by RuleEvolution not by design—but because their loss creates system-wide drift, fragility, or collapse.

They are coherence-preserving structures with nonlocal return influence.

