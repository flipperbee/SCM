\chapter{Reuse Quantization and Latency Bands} \label{chapter-latency-bands}

Persistent identities in SCM are not continuously distributed in reuse space. Instead, they appear in quantized layers of return elevation and latency, anchored by fixed structural identities and bounded by coherence thresholds. This chapter defines these reuse layers as \textbf{latency bands} and describes the quantization conditions under which identities remain return-closed.

\section{Latency Bands and Coherence Wells} \label{sec:latency-bands}

Let $[A] \in \Omega_3$ be reuse-stable, and let $\Lambda([A])$ denote its total latency.

A \textbf{latency band} is a reuse region satisfying:
\[
\Lambda([A]) \in [n \cdot \Delta, (n+1) \cdot \Delta),\quad n \in \mathbb{N}
\]
for some fixed coherence step size $\Delta$.

Each band is defined by anchoring reuse and bounded symbolic asymmetry. Anchors serve as the bottom of these bands, stabilizing higher layers via return closure.

\section{Quantization from Return Symmetry} \label{sec:return-quantization}

Quantized reuse levels arise when:
\[
X_\pi([A_{i+1}]) = X_\pi([A_i]) + \delta,\quad \Lambda([A_{i+1}]) = \Lambda([A_i]) + \Delta
\]

Where:
- $\delta$ is the return asymmetry increment,
- $\Delta$ is the effort-weighted latency step.

Triplets that share an anchor and satisfy these relations form reuse ladders.

\section{Band Transitions and Structural Excitation} \label{sec:band-transitions}

Identity $[A]$ may shift to a higher latency band only if the following conditions are met:
\[
\partial\chi([A]) = 0,\quad
\Lambda([A']) = \Lambda([A]) + \Delta,\quad
X_\phi([A']) < \phi_c,\quad
q([A']) \in \text{bounded range}
\]

Otherwise, RuleEvolution will prune $[A']$ due to contradiction.

This structure defines symbolic excitation: transitions between return-closed layers within a reuse family.

\section{Family Triplets and Reuse Ladders} \label{sec:reuse-ladders}

Let a triplet $\{[A_1], [A_2], [A_3]\}$ satisfy:
\[
X_\epsilon([A_{i+1}]) = X_\epsilon([A_i]) + \Delta,\quad \text{for all } i
\]

Then the triplet lies on a symbolic reuse ladder. These ladders form the underlying structure of quantized families, such as leptons and (with variation) quarks.

Return elevation behaves discretely. There is no continuous deformation of reuse structure—only quantized, coherence-locked steps permitted by $\mathcal{R}E$.

\section{Latency Floor and Mass Gap} \label{sec:mass-gap}

Let $\Lambda_0$ be the minimal latency floor defined in Volume III. Then any identity $[A]$ with:
\[
\Lambda([A]) < \Lambda_0
\quad \Rightarrow \quad
[A] \notin \Omega_3
\]

This defines the symbolic mass gap. The lowest reuse-anchored persistence level is bounded below—not due to imposed limits, but because collapse becomes inevitable beneath this latency threshold.

This threshold gives physical meaning to particle “mass” in SCM:
- Mass is symbolic latency,
- Quantization emerges from reuse elevation,
- The gap is the minimal structural effort required for identity persistence.

