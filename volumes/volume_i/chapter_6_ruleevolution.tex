\chapter{RuleEvolution}

Until now, we have treated the rule set R as fixed. But identities in
SCM do not merely exist---they evolve. Under coherence pressure,
symbolic structures may drift, deform, collapse, or reorganize. A static
rule set is insufficient to model such behaviors.\\
\strut \\
This chapter introduces RuleEvolution---a second-order operator that
updates the permitted transformations R based on the coherence behavior
of resolved identities. We then classify identity behavior based on
their trajectories in χ-space, introducing stable, reactive, and
collapsing structures. These categories are not imposed---they arise
from the topological dynamics of symbolic return.

\section{6.1 \textbar{} Motivation: Identity Under Coherence
Strain}\label{motivation-identity-under-coherence-strain}

Let {[}A{]} ∈ Ω₃ be a resolved identity. Even if return path 𝓛({[}A{]})
exists, reuse may:

\begin{quote}
- Saturate: {[}A{]} appears in too many return loops.\\
- Drift: χ({[}A{]}) changes under transformation (∂χ ≠ 0).\\
- Collapse: ρ({[}A{]}) \textless{} ρ\_c under symbolic stress.
\end{quote}

We require a formal mechanism by which R can adapt: preserving
coherence, stabilizing reuse, or pruning collapse-prone forms. This is
RuleEvolution.

\section{6.2 \textbar{} Formal Definition of
RuleEvolution}\label{formal-definition-of-ruleevolution}

Let:

\begin{quote}
- Rₜ be the enabled Rule Set at evolution step n,\\
- G(Rₜ) be the resolution graph generated by Rₜ,\\
- Σₜ be the set of resolved identities at t,\\
- χₜ({[}A{]}) be the coherence signature of identity {[}A{]} at t,\\
- ∂χ({[}A{]}) := χₜ₊₁({[}A{]}) − χₜ({[}A{]}) be the signature drift of
{[}A{]},\\
- ρ({[}A{]}) be the collapse robustness of {[}A{]}.
\end{quote}

We define the RuleEvolution operator as:\\
  𝓡E: (G(Rₜ), χₜ) → updated subset Rₜ₊₁ ⊂ Σ*\\
\strut \\
We define the RuleEvolution operator as:

𝓡E: (G(Rₜ), χₜ) → Rₜ₊₁ ⊂ Σ*

RuleEvolution is not an algorithm that rewrites R---it is a structural
classifier. It re-enables or disables symbolic structures T ∈ Σ* based
on their participation in coherence-stabilizing return.

A transformation T becomes enabled (T ∈ R) if it:

\begin{quote}
- Appears in a return path 𝓛({[}A{]}) where ∂χ({[}A{]}) is bounded and
ρ({[}A{]}) ≥ ρ\_c,\\
- Does not produce cancellation (e.g., {[}S{]} + {[}--S{]} = {[}0{]})
rather than persistence.
\end{quote}

\subsection{Rule Update Conditions}\label{rule-update-conditions}

1. Coherence Drift (∂χ divergence):

- If ∂χ({[}A{]}) exceeds a symbolic threshold ε, coherence becomes
unstable.

2. Collapse Proximity (ρ \textless{} ρ\_c):

- If robustness falls below the collapse threshold, return closure is at
risk.

3. New Identity Resolution (𝓛-discovery):

- If a new valid return loop 𝓛({[}B{]}) emerges, R must be expanded to
stabilize {[}B{]}.

\subsection{RuleEvolution Scoring
Function}\label{ruleevolution-scoring-function}

Let T ∈ Σ*. Define the affected identity set:

Ω\_T := \{ {[}A{]} ∈ Ω₃ \textbar{} T ∈ 𝓛({[}A{]}) or T enables a new
𝓛({[}A{]}) \}

To compute the scoring function S(T), we define drift ∂χ({[}A{]}) using
a symbolic, componentwise average of absolute differences:

  ∂χ({[}A{]}) := (1/5) ∑\emph{\{i=1\}\^{}5
\textbar χ}\{t+1\}\^{}\{(i)\}({[}A{]}) −
χ\_t\^{}\{(i)\}({[}A{]})\textbar{}

This scalar drift measure respects the non-metric product topology of
χ-space. It avoids imposing a Euclidean structure while still enabling
RuleEvolution scoring. Please note that component-wise drift may be
optionally normalized or weighted for application-specific sensitivity.

Then define:

Δ∂χ\_total(T) := ∑\_\{{[}A{]} ∈ Ω\_T\} (∂χ\_old({[}A{]}) −
∂χ\_new({[}A{]}))

Δρ\_total(T) := ∑\_\{{[}A{]} ∈ Ω\_T\} (ρ\_new({[}A{]}) −
ρ\_old({[}A{]}))

Finally, compute the score:

S(T) := Δ∂χ\_total(T) − Δρ\_total(T)

Here:\\
- ∂χ\_old({[}A{]}) and ρ\_old({[}A{]}) are computed under rule set Rₜ\\
- ∂χ\_new({[}A{]}) and ρ\_new({[}A{]}) are computed under rule set Rₜ ∪
\{T\}

This ensures that S(T) is a real scalar value, and that RuleEvolution is
fully symbolic and mathematically well-defined.

\subsection{Formal expression}\label{formal-expression}

𝓡E(Rₜ) := (Rₜ ∪ R\_enable) \textbackslash{} R\_disable

Where:

\begin{quote}
- R\_enable = \{ T ∉ Rₜ \textbar{} S(T) \textgreater{} θₐ and T enables
new 𝓛({[}B{]}) \},\\
- R\_disable = \{ T ∈ Rₜ \textbar{} S(T) \textless{} −θ\_d or
destabilizes a saturated identity \}
\end{quote}

Thresholds θₐ and θ\_d control sensitivity.

\section{6.3 \textbar{} Drift and Structural
Classes}\label{drift-and-structural-classes}

Let ∂χ({[}A{]}) := χₜ₊₁({[}A{]}) - χₜ({[}A{]}) be the signature drift of
identity {[}A{]}.\\
\strut \\
We classify identity behavior into three structural types:

\begin{quote}
- Stable: ∂χ({[}A{]}) = 0; Structure invariant under reuse\\
- Reactive: ∂χ({[}A{]}) bounded, ∂χ ≠ 0; Structure deforms or fails\\
- Collapsing: ∂χ({[}A{]}) unbounded or ρ \textless{} ρ\_c
\end{quote}

These classes partition Ω₃ into regions of structural persistence,
adaptability, and fragility.

\section{6.4 \textbar{} RuleEvolution
Pressure}\label{ruleevolution-pressure}

Let {[}A{]} ∈ Ω₃ experience drift (∂χ ≠ 0) or reuse strain. Then {[}A{]}
exerts RuleEvolution pressure on R:

\begin{quote}
- If coherence is preserved, R may be reinforced (stabilizing
transformations),\\
- If coherence degrades, R may adapt (deleting or rewriting weak
transitions).
\end{quote}

This pressure arises from the internal state of the system---not from
external rules or inputs. SCM evolves structurally, not algorithmically.

\section{6.5 \textbar{} The Dynamic Identity
Landscape}\label{the-dynamic-identity-landscape}

Define the identity space Ω₃ as dynamically partitioned:

\begin{quote}
- Invariant Surface 𝓢 := \{ {[}A{]} \textbar{} ∂χ({[}A{]}) = 0 \}\\
- Reactive Basin 𝓑 := \{ {[}A{]} \textbar{} ∂χ bounded, ∂χ ≠ 0 \}\\
- Collapse Fringe ℱ := \{ {[}A{]} \textbar{} ∂χ diverges or ρ
\textless{} ρ\_c \}
\end{quote}

These regions define symbolic dynamics over χ-space. RuleEvolution acts
to stabilize Ω₃ by:

\begin{quote}
- Expanding 𝓢 (more stable forms),\\
- Containing 𝓑 (limiting drift),\\
- Pruning ℱ (removing collapse-prone paths).
\end{quote}

\section{6.6 \textbar{} Summary}\label{summary-4}

RuleEvolution is SCM's structural classifier. It evaluates symbolic
structures in Σ* and determines:

\begin{quote}
- Which transformations preserve identity stability (enable),\\
- Which degrade coherence and must be pruned (disable),\\
- Which enable new identity resolution (promote).
\end{quote}

It is deterministic, topology-driven, and fundamental to symbolic
adaptation.

Chapter 7 \textbar{} Triadic Principle

Not all identities are structurally equivalent. Some resolve return
independently. Others rely on internal reuse, and some require external
anchoring to remain coherent.\\
\strut \\
This chapter formalizes the Triadic Classification of identity in SCM.
Every coherence-resolved identity falls into one of three mutually
exclusive categories:\\
- Irreducible: return-closed without reuse,\\
- Composed: return-closed through internal reuse only,\\
- Elevated: return-closed only through external reuse.\\
\strut \\
These classes reflect structural constraints in χ-space and play a
central role in the emergence of particle structure and coherence
networks in later volumes.

\section{7.1 \textbar{} The Triadic
Principle}\label{the-triadic-principle}

Let {[}A{]} ∈ Ω₃. Then {[}A{]} falls into exactly one of the following
categories:\\
\strut \\
1. Irreducible:\\
  {[}A{]} resolves return without reusing any other structure.\\
  Formal conditions:\\
  - Xₑ = 0\\
  - support({[}A{]}) = ∅\\
  - 𝓛({[}A{]}) contains no substructure used elsewhere

Structures with minimal return but full inversion (e.g., Euler pairs
with Xₕ = 1 and Xπ = --1) do not qualify as irreducible---they cancel
rather than persist.\\
\strut \\
2. Composed:\\
  {[}A{]} resolves return through reuse of internal substructures.\\
  Formal conditions:\\
  - Xₑ = 0\\
  - support({[}A{]}) ≠ ∅\\
  - All reused structures lie within 𝓛({[}A{]})\\
\strut \\
3. Elevated:\\
  {[}A{]} resolves return only via support outside its minimal loop.\\
  Formal condition:\\
  - Xₑ \textgreater{} 0\\
\strut \\
We call this partition the Triadic Principle.

\section{7.2 \textbar{} Formal Proof of Exclusivity and
Exhaustiveness}\label{formal-proof-of-exclusivity-and-exhaustiveness}

Let {[}A{]} ∈ Ω₃. The three cases above are:\\
\strut \\
- Mutually exclusive:

\begin{quote}
- If support({[}A{]}) = ∅ → {[}A{]} cannot be composed or elevated.\\
- If Xₑ = 0 and support({[}A{]}) ≠ ∅ → {[}A{]} is composed.\\
- If Xₑ \textgreater{} 0 → support necessarily includes external
structures → {[}A{]} is elevated.
\end{quote}

- Collectively exhaustive:

\begin{quote}
- Either support({[}A{]}) is empty or non-empty.\\
- If empty → irreducible.\\
- If non-empty → either Xₑ = 0 (composed) or Xₑ \textgreater{} 0
(elevated).
\end{quote}

Thus, every identity in Ω₃ falls into exactly one of these three types.

\section{7.3 \textbar{} Examples and Structural
Consequences}\label{examples-and-structural-consequences}

- Irreducible:\\
 Example: The photon {[}γ{]} with 𝓛({[}γ{]}) = A → B → A, Xₑ = 0\\
 Interpretation: foundational identity, coherence baseline\\
\strut \\
- Composed:\\
 Example: {[}C{]} with internal loop A → B → C → A, where A and B are
reused but internal\\
 Interpretation: nested or modular structures\\
\strut \\
- Elevated:\\
 Example: {[}D{]} requires reuse of external anchor {[}E{]} for
coherence closure\\
 Interpretation: dependent identity, may be fragile or context-bound\\
\strut \\
These classes determine:\\
- Which identities can act as anchors,\\
- Which are reusable,\\
- Which are conditionally coherent or collapse-prone

\section{7.4 \textbar{} χ-Space Signatures by
Class}\label{ux3c7-space-signatures-by-class}

\begin{longtable}[]{@{}
  >{\raggedright\arraybackslash}p{(\linewidth - 6\tabcolsep) * \real{0.1900}}
  >{\raggedright\arraybackslash}p{(\linewidth - 6\tabcolsep) * \real{0.1452}}
  >{\raggedright\arraybackslash}p{(\linewidth - 6\tabcolsep) * \real{0.2099}}
  >{\raggedright\arraybackslash}p{(\linewidth - 6\tabcolsep) * \real{0.4549}}@{}}
\toprule\noalign{}
\begin{minipage}[b]{\linewidth}\raggedright
Class
\end{minipage} & \begin{minipage}[b]{\linewidth}\raggedright
Xₑ
\end{minipage} & \begin{minipage}[b]{\linewidth}\raggedright
Support({[}A{]})
\end{minipage} & \begin{minipage}[b]{\linewidth}\raggedright
χ-Space Behavior
\end{minipage} \\
\midrule\noalign{}
\endhead
\bottomrule\noalign{}
\endlastfoot
Irreducible & 0 & ∅ & Low-dimensional, stable \\
Composed & 0 & ≠ ∅ (internal) & Moderate Xₕ, low Xφ \\
Elevated & \textgreater0 & ≠ ∅ (external) & High Xₑ, possibly high Xφ \\
\end{longtable}

Elevated identities tend to cluster near the collapse boundary (Xφ ↑),
unless stabilized by anchors. Irreducible identities typically lie on or
near the invariant surface 𝓢 (∂χ = 0).

\section{7.5 \textbar{} Structural
Implications}\label{structural-implications}

The Triadic Principle forms the logical foundation for:

\begin{quote}
- Coherence inheritance\\
- Reuse-based emergence\\
- Structural modularity
\end{quote}

Reuse hierarchy:

\begin{quote}
- Irreducible identities → form the coherence base\\
- Composed identities → build internal reuse loops\\
- Elevated identities → float atop reuse scaffolds
\end{quote}

In later chapters, this structure enables:

\begin{quote}
- Definition of coherence anchors\\
- Stabilization of partial identities\\
- Formation of triplet reuse locks and symmetry-stabilized composites
\end{quote}

\section{7.6 \textbar{} Summary}\label{summary-5}

Every resolved identity in SCM is either:

\begin{quote}
- Irreducible: return-closed and reuse-free,\\
- Composed: internally reused and stable,\\
- Elevated: externally reuse-dependent.
\end{quote}

This classification is exhaustive and exclusive. It is grounded in
χ-space structure and encoded directly in the coherence signature.\\
\strut \\
In the next chapter, we examine what lies between identity and collapse:
partial identities, proto-anchors, and the fragile emergence of
coherence from dependency.

\section{Chapter 8 \textbar{} Partial
Identity}\label{chapter-8-partial-identity}

Not all symbolic structures in Ω₃ are fully independent. Some require
reuse to remain coherent. Others hover near collapse but persist due to
stabilization by surrounding anchors.\\
\strut \\
This chapter defines partial identity---a class of drift-prone,
reuse-dependent structures that resolve only within specific coherence
configurations. We also introduce proto-anchors, unstable patterns
reused by others, and the mechanism of drift suppression: how coherence
anchors stabilize fragile identities.\\
\strut \\
These concepts formalize the early phase of emergence---the symbolic
scaffolding that precedes particles, logic, or code.

\section{8.1 \textbar{} Definition of Partial
Identity}\label{definition-of-partial-identity}

Let {[}A{]} ∈ Ω₃ be a resolved identity. We say that {[}A{]} is a
partial identity if:

\begin{quote}
- 𝓛({[}A{]}) exists, but\\
- χ({[}A{]}) is unstable unless reuse anchors are present,\\
- Xₑ({[}A{]}) \textgreater{} 0, and\\
- ∂χ({[}A{]}) ≠ 0 under standalone reuse
\end{quote}

In short: {[}A{]} resolves return only when embedded within a coherence
cluster. It cannot stabilize in isolation.

\section{8.2 \textbar{} Reuse Dependence and Structural
Context}\label{reuse-dependence-and-structural-context}

Let {[}A{]} reuse a saturated anchor {[}B{]} such that:

\begin{quote}
- {[}B{]} lies on the invariant surface 𝓢 (∂χ({[}B{]}) = 0),\\
- {[}B{]} appears in G\_{[}A{]},\\
- Reuse stabilizes χ({[}A{]}) such that ∂χ({[}A{]}) → 0
\end{quote}

Then {[}A{]} is reuse-locked: coherence is inherited via coupling. This
is the first form of coherence inheritance in SCM. Return no longer
depends only on internal structure, but on coupling to a stable external
identity.

\section{8.3 \textbar{} Drift Suppression via Anchor
Coupling}\label{drift-suppression-via-anchor-coupling}

Let {[}A{]} be a partial identity. If:

\begin{quote}
- ∂χ({[}A{]}) ≠ 0 in isolation,\\
- but ∂χ({[}A{]}) → 0 under reuse of {[}B{]} with χ({[}B{]}) ∈ 𝓢,
\end{quote}

Then we define a drift suppression lock. This mechanism:

\begin{quote}
- Suppresses structural deformation in reuse-prone identities,\\
- Stabilizes near-collapse structures across coherence clusters,\\
- Creates symbolic analogues of binding, insulation, and symmetry
breaking
\end{quote}

Anchors must preserve identity under reuse. Inversion structures like
Euler pairs (Xₕ = 1, X\_π = --1) resolve to {[}0{]} and cannot function
as coherence anchors.

\section{8.4 \textbar{} Proto-Anchors}\label{proto-anchors}

Let {[}C{]} be a symbolic structure such that:

\begin{quote}
- 𝓛({[}C{]}) = ∅ (not yet resolved),\\
- {[}C{]} appears in G\_{[}A{]} for multiple {[}A{]} ∈ Ω₃,\\
- {[}C{]} is reused but not yet stable
\end{quote}

We call {[}C{]} a proto-anchor.

If over steps:

\begin{quote}
- {[}C{]} appears in more coherence paths,\\
- RuleEvolution reinforces closure paths to/from {[}C{]},\\
- ∂χ({[}C{]}) → 0 and ρ({[}C{]}) ≥ ρ\_c,
\end{quote}

Then {[}C{]} becomes a full coherence anchor. This transition is the
symbolic equivalent of structural emergence.

\section{8.5 \textbar{} Resolution Spectrum: Identity vs
Collapse}\label{resolution-spectrum-identity-vs-collapse}

We now classify all symbolic forms by their return and reuse conditions:

\begin{longtable}[]{@{}
  >{\raggedright\arraybackslash}p{(\linewidth - 8\tabcolsep) * \real{0.2546}}
  >{\raggedright\arraybackslash}p{(\linewidth - 8\tabcolsep) * \real{0.2260}}
  >{\raggedright\arraybackslash}p{(\linewidth - 8\tabcolsep) * \real{0.1776}}
  >{\raggedright\arraybackslash}p{(\linewidth - 8\tabcolsep) * \real{0.0808}}
  >{\raggedright\arraybackslash}p{(\linewidth - 8\tabcolsep) * \real{0.2612}}@{}}
\toprule\noalign{}
\begin{minipage}[b]{\linewidth}\raggedright
Form
\end{minipage} & \begin{minipage}[b]{\linewidth}\raggedright
Return Loop (𝓛)
\end{minipage} & \begin{minipage}[b]{\linewidth}\raggedright
\textbar{} ∂χ Behavior
\end{minipage} & \begin{minipage}[b]{\linewidth}\raggedright
Xₑ
\end{minipage} & \begin{minipage}[b]{\linewidth}\raggedright
Status
\end{minipage} \\
\midrule\noalign{}
\endhead
\bottomrule\noalign{}
\endlastfoot
Fully Resolved & Exists & ∂χ = 0 & ≥ 0 & Stable identity \\
Partial Identity & Exists & ∂χ ≠ 0 alone & \textgreater{} 0 &
Contextually stable \\
Proto-Anchor & Not yet exists & ∂χ undefined & --- & Reuse
scaffolding \\
Collapsed Structure & Fails & ∂χ diverges & --- & Incoherent \\
\end{longtable}

This spectrum explains why some structures persist only in certain reuse
environments---and why most symbolic forms in Σ* never become
identities.

\section{8.6 \textbar{} Structural Role of Partial
Identity}\label{structural-role-of-partial-identity}

Partial identities:

\begin{quote}
- Populate the boundary between incoherence and structure,\\
- Enable compositional reuse without structural stability,\\
- Bridge the gap between randomness and persistent identity
\end{quote}

They are not errors---they are emergence pathways.\\
\strut \\
Most elevated identities begin as partial identities. Most anchors pass
through a proto-anchor phase.\\
\strut \\
This chapter closes the foundational narrative of structural emergence.
In the next chapter, we show how triplet reuse locks and coherence
convergence produce the first fully symmetry-stabilized families of
identity.

\section{Chapter 9 \textbar{} Coherence
Triplets}\label{chapter-9-coherence-triplets}

Once partial identities stabilize, and reuse structures lock under
coherence, a new class of configurations appears: triplet-convergent
identities. These are not isolated; they cohere through mutual reuse of
a shared structural anchor. When the reuse configuration is symmetric
and stabilized under RuleEvolution, we call the result a coherence
triplet.\\
\strut \\
This chapter formalizes how such structures form, what conditions enable
them, and why they mark the symbolic threshold between generic identity
and structure.

\section{9.1 \textbar{} χ-Convergent Identity
Families}\label{ux3c7-convergent-identity-families}

Let \{{[}A₁{]}, {[}A₂{]}, {[}A₃{]}\} be three resolved identities in
Ω₃.\\
We say they are a χ-convergent family if:

\begin{quote}
- There exists a structure {[}T{]} such that:\\
- Each {[}Aᵢ{]} reuses {[}T{]} (T ∈ support({[}Aᵢ{]})),\\
- χ({[}Aᵢ{]}) lie within a shared χ-neighborhood,\\
- The coherence signature across the set is symmetric:\\
  ∑ χ({[}Aᵢ{]}) = 3χ({[}T{]})
\end{quote}

This structure {[}T{]} serves as a reuse anchor for the triplet.\\
If {[}T{]} is saturated (∂χ({[}T{]}) = 0), the configuration forms a
stable reuse-locked system.

\section{9.2 \textbar{} Minimal Reuse Lock and Return
Stabilization}\label{minimal-reuse-lock-and-return-stabilization}

Define {[}T{]} as a minimal reuse-locked anchor if:

\begin{quote}
- {[}T{]} is not independently return-closed (𝓛({[}T{]}) = ∅ or
unstable),\\
- {[}T{]} appears in 𝓛({[}Aᵢ{]}) for exactly three distinct {[}Aᵢ{]},\\
- The composite structure \{{[}Aᵢ{]}, {[}T{]}\} resolves return closure
when combined.
\end{quote}

This is the symbolic skeleton of reuse entanglement.\\
Return is no longer isolated---it is structurally distributed.

\section{9.3 \textbar{} Symmetry-Stabilized
Triplets}\label{symmetry-stabilized-triplets}

Let a triplet \{{[}A₁{]}, {[}A₂{]}, {[}A₃{]}\} satisfy:

\begin{quote}
- χ({[}Aᵢ{]}) have symmetric latency: Λ({[}Aᵢ{]}) = Λ₀ ± ε,\\
- Return symmetry: ∑ Xπ({[}Aᵢ{]}) = 0,\\
- External reuse load is minimal: Xₑ({[}T{]}) = 0,\\
- This symmetry balancing means one identity may act as a coherence
source (Xπ \textgreater{} 0), one as a sink (Xπ \textless{} 0), and one
as a mediator (Xπ ≈ 0)\\
- Anchor {[}T{]} must have ∂χ = 0 and X\_π ≈ 0
\end{quote}

Then the system is symmetry-stabilized. It forms a coherence triplet
lock, the smallest resolved unit of symmetry-based identity.

Inversion pairs ({[}S{]}, {[}--S{]}) cannot form triplets. They resolve
to {[}0{]} via structural annihilation and do not stabilize reuse

\section{9.4 \textbar{} Identity Clustering}\label{identity-clustering}

These triplet structures exhibit:

\begin{quote}
- Mutual drift suppression,\\
- Shared reuse pathways (G\_{[}Aᵢ{]} ∩ G\_{[}T{]} ≠ ∅),\\
- Invariant coherence under RuleEvolution,\\
- Return cycle invariance (loop-saturated, Λ fixed)
\end{quote}

Such configurations are reuse-resolved units. In χ-space, they appear as
compact clusters orbiting a coherence anchor.

\section{9.5 \textbar{} Summary and Threshold of Physical
Identity}\label{summary-and-threshold-of-physical-identity}

Coherence triplets mark a turning point in the SCM hierarchy.\\
They are:

\begin{quote}
- Return-closed,\\
- Drift-suppressed,\\
- Symmetry-locked,\\
- Coherence-anchored
\end{quote}

These are no longer arbitrary structures. They are symbolically stable
configurations.\\
\strut \\
This closes Volume I.
