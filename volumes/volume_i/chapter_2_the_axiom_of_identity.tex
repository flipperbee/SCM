\chapter{The Axiom of Identity}

\section*{Introduction}

Having established the symbolic substrate in Chapter 1, we now define what it means for a symbolic structure to possess identity in SCM. In the absence of external referents such as mass, spacetime location, or continuity, identity must emerge purely from structural self-consistency. This motivates the central axiom of SCM: a structure exists if and only if it participates in a permitted return loop.

To formalize this axiom, we introduce the \emph{Resolution Graph} $G(R)$, which encodes all permitted transformations under a given rule set $R$. Paths and loops in $G(R)$ represent transformation sequences; identity is defined in terms of such loops. We then classify symbolic structures into distinct domains according to their return behavior.

\section{Permitted Transformations and Rule Sets}

\begin{definition}[Rule Set]
A \emph{rule set} $R$ is a finite subset of the transformation space $\mathcal{T} = \{T : \Sigma^* \to \Sigma^*\}$. Each $T \in R$ is said to be \emph{permitted}.
\end{definition}

\begin{definition}[Permitted Transformation]
A transformation $T: A \to B$ is \emph{permitted} if $T \in R$.
\end{definition}

\noindent
At any configuration of the SCM system, the rule set $R$ defines which transformations are admissible. The existence of structure depends entirely on return paths formed from these permitted transformations.

\section{The Resolution Graph}

\begin{definition}[Resolution Graph]
Given a rule set $R$, the \emph{Resolution Graph} is a directed multigraph
\[
G(R) := (V, E),
\]
where:
\begin{itemize}
    \item $V := \{ A \in \Sigma^* \mid \exists B \in \Sigma^*, \, T: A \to B \text{ or } T: B \to A \text{ with } T \in R \}$.
    \item $E \subseteq V \times V$ is the set of directed edges $(A, B)$ such that there exists a permitted transformation $T: A \to B$ in $R$.
\end{itemize}
Each edge corresponds to a transformation; multiple transformations between the same pair of structures yield multiple edges.
\end{definition}

\begin{definition}[Path and Loop]
A \emph{path} in $G(R)$ is a finite sequence of structures $(A_0, A_1, \dots, A_n)$ such that for each $i$, $(A_i, A_{i+1}) \in E$. A path is a \emph{loop} (or \emph{cycle}) if $A_0 = A_n$.
\end{definition}

\section{The Axiom of Identity}

\begin{axiom}[Identity by Permitted Return]
A symbolic structure $[A] \in \Sigma^*$ is said to \emph{possess identity} if and only if there exists at least one permitted return loop in $G(R)$ passing through $A$.
\end{axiom}

\begin{definition}[Return Path]
Let $A \in \Sigma^*$. A \emph{return path} of $A$ under rule set $R$ is a loop $(A_0, A_1, \dots, A_n)$ in $G(R)$ such that $A_0 = A_n$ and $A = A_k$ for some $k$.
\end{definition}

\begin{definition}[Return Closure]
The \emph{return closure} of a structure $A$, denoted $L([A])$, is the set of all return paths in $G(R)$ that pass through $A$.
\end{definition}

\noindent
Under the axiom above, only structures with non-empty return closure are treated as identities.

\section{Identity Domains ($\Omega$-Space)}

Based on the return behavior of symbolic structures in $G(R)$, we classify them into four distinct domains:

\begin{definition}[Identity Domains]
Given a rule set $R$, define the following domains:
\begin{itemize}
    \item $\Omega_1$: \emph{Unresolved Forms} — structures with no return path in $G(R)$.
    \item $\Omega_2$: \emph{Unstable Returns} — structures with return paths that do not survive optimization (see Chapter 3).
    \item $\Omega_3$: \emph{Resolved Identities} — structures that persist under the globally optimal rule set $R^*$.
    \item $\Omega_{-}(R)$: \emph{Annihilated Structures} — structures that resolve via permitted contradiction to the null identity $[0]$. We define the null identity $[0]$ as the unique structure representing the symbolic collapse of coherence. It may be taken to be either the empty string $\varepsilon \in \Sigma^*$, or a canonical absorbing element adjoined to $\Sigma^*$ under contradiction closure.
\end{itemize}
\end{definition}

\noindent
This classification is dynamic: membership in $\Omega_k$ depends on the rule set $R$. Only $\Omega_3$ corresponds to stable, persistent identity under the global dynamics of SCM.

\paragraph{Note}
These domains are defined by their behavior in the system's final, optimized state. A formal, computable definition based on the coherence signature for any arbitrary rule set R will be provided in Chapter 3 after the variational principle is introduced.

\section{Remarks and Implications}

\begin{itemize}
    \item The identity of a structure is not intrinsic; it is a function of the return paths available under the current rule set.
    \item The Resolution Graph $G(R)$ provides the computational domain on which identity can be tested and evaluated.
    \item The domain $\Omega_3$ will later serve as the foundation for physical identity: particles, fields, and conserved structures.
\end{itemize}

\section{Forward Linkage}

In Chapter 3, we introduce the \emph{Law of Symbolic Dynamics}—the variational principle that selects the optimal rule set $R^*$ which governs the actual configuration of the SCM universe. This principle determines which structures ultimately survive in $\Omega_3$.
