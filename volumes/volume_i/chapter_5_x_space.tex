\chapter{$\chi$-Space}

Coherence metrics allow us to evaluate how stable an identity is under reuse, deformation, and collapse.  
But to compare identities structurally, we require a unified representation:  
a signature that encodes all relevant coherence properties in a single form.

This chapter introduces the \textit{coherence signature} $\chi([A])$,  
a five-component vector that structurally fingerprints any return-closed identity.  
We then define the space of all such signatures — $\chi$-space — as a topological product space over symbolic metrics.

Trajectories through this space represent structural evolution.  
Stability, collapse, and emergence are now modeled as topological transitions.

\section{Definition of the Coherence Signature $\chi$}

\begin{definition}[Coherence Signature] \label{def:chi-signature}
Let $[A] \in \Omega_3$ be a return-closed, collapse-robust identity.  
We define its coherence signature as the 5-tuple:
\begin{equation} \label{eq:chi-signature}
\chi([A]) := (X_h, X_c, X_\pi, X_\phi, X_\epsilon)
\end{equation}

Where:
\begin{itemize}
  \item $X_h$: Return depth (Chapter~2)
  \item $X_\pi$: Return symmetry (Chapter~4)
  \item $X_\phi$: Fragility, defined as $1 / \rho$ (Chapter~4)
  \item $X_\epsilon$: Reuse elevation (defined in this chapter)
  \item $X_c$: Coherence strength (defined next)
\end{itemize}
\end{definition}

This signature completely characterizes the identity structure of $[A]$ with respect to reuse, return, and collapse.

\section{Return Depth $X_h$ and Symmetry $X_\pi$}

These two coherence metrics were previously defined in Chapters~2 and 4:

\begin{itemize}
  \item $X_h$ is the length of the shortest return loop $\mathcal{L}([A])$
  \item $X_\pi$ is the directional imbalance of the return path (forward vs.\ reverse transitions)
\end{itemize}

Both are structural invariants derived from the resolution graph $G(R)$.

\section{Coherence Strength $X_c$}

Let $\mathcal{R}([A])$ be the set of all permitted return loops for $[A]$.  
Let $\chi_0 = \chi(\mathcal{L}_0([A]))$ be the coherence signature of the minimal return path.

For each alternate return path $\mathcal{L}_i \in \mathcal{R}([A])$, let $\chi_i := \chi(\mathcal{L}_i([A]))$  
be the coherence signature derived from that path.

\begin{definition}[Component-Wise Deviation] \label{def:component-deviation}
Let $[A] \in \Omega_3$. Let $\chi_0 = \chi(L_0([A]))$ be the coherence signature of the minimal return path, and let $\chi_i = \chi(L_i([A]))$ for alternate return paths $L_i \in \mathcal{R}([A])$.

We define the scaled component-wise deviation as:
\[
\Delta \chi_i := \frac{1}{5} \sum_{k=1}^5 \frac{\left| \chi_i^{(k)} - \chi_0^{(k)} \right|}{1 + \left| \chi_0^{(k)} \right|}
\]
\end{definition}

Here, each coherence component is compared independently.  
The result is a normalized average deviation across all five dimensions.

\begin{definition}[Coherence Strength] \label{def:coherence-strength}
Let $\mathcal{R}([A])$ be the set of return paths for $[A]$. The coherence strength of $[A]$ is:
\[
X_c([A]) := \frac{1}{|\mathcal{R}([A])|} \sum_{i} \left( 1 - \Delta \chi_i \right)
\]
\end{definition}

\textbf{Interpretation:}
\begin{itemize}
  \item $X_c \approx 1$: All return paths are consistent — identity is structurally coherent regardless of path
  \item $X_c \ll 1$: Return paths diverge — identity is path-sensitive, fragile, or degenerate
\end{itemize}

This formally encodes coherence consistency without requiring a global metric.  
It respects the non-normed, component-wise topology of $\chi$-space and ensures each dimension is weighted equally and independently.

\section{Reuse Elevation $X_\epsilon$} \label{reuse-elevation-xe}

Let $G_{[A]}$ be the resolution subgraph of $[A]$ (as defined in Chapter~2).  
Let $\text{support}([A])$ be the set of all structures reused in resolving $[A]$.

\begin{definition}[Reuse Elevation] \label{def:reuse-elevation}
We define reuse elevation as:
\begin{equation} \label{eq:reuse-elevation}
X_\epsilon([A]) := \left| \text{support}([A]) \setminus \mathcal{L}([A]) \right|
\end{equation}
\end{definition}

That is, $X_\epsilon$ is the number of symbolic structures required for return  
that are not part of $[A]$’s own minimal loop.

\textit{Note:} $\text{support}([A])$ and $\mathcal{L}([A])$ are interpreted here as node sets  
extracted from the return subgraph $G_{[A]}$, enabling set subtraction.

\textbf{Interpretation:}
\begin{itemize}
  \item $X_\epsilon = 0$: $[A]$ is internally return-closed (irreducible or composed)
  \item $X_\epsilon > 0$: $[A]$ depends on external reuse (elevated identity)
\end{itemize}

\section{Fragility $X_\phi$ and Collapse Threshold}

As previously defined in Chapter~4:
\begin{align*}
X_\phi([A]) &:= \frac{1}{\rho([A])} \\
\rho([A]) &:= \text{minimum symbolic effort to destroy } \mathcal{L}([A])
\end{align*}

Fragility encodes coherence brittleness.  
This metric defines the symbolic boundary of collapse in $\chi$-space.

\section{$\chi$-Space: The Topology of Identity} \label{chi-space-the-topology-of-identity}

Let:
\[
X := \mathbb{N} \times [0,1] \times \mathbb{R}_+ \times \mathbb{R}_+ \times \mathbb{N}
\]

\begin{definition}[$\chi$-Space] \label{def:chi-space}
We define $\chi$-space as the set of coherence signatures for all return-closed identities:
\begin{equation} \label{eq:chi-space}
\chi\text{-space} := \{ \chi([A]) \mid [A] \in \Omega_3 \}
\end{equation}
\end{definition}

This is a symbolic, non-metric product space.  
Each component of $\chi$ governs a distinct mode of coherence behavior.

There is no global norm on $\chi$-space.  
Comparison must occur componentwise or via defined $\chi$-neighborhoods.

\section{$\chi$-Neighborhoods and Clustering} \label{chi-neighborhoods-and-clustering}

Let $\chi_0 \in X$.  
We define a $\chi$-neighborhood around $\chi_0$ as:

\begin{definition}[$\chi$-Neighborhood] \label{def:chi-neighborhood}
\begin{equation} \label{eq:chi-neighborhood}
\mathcal{U}(\chi_0, \epsilon) := \left\{ \chi([A]) \,\middle|\, \forall i,\, \left| \chi^{(i)} - \chi_0^{(i)} \right| < \epsilon^{(i)} \right\}
\end{equation}
\end{definition}

These neighborhoods enable:
\begin{itemize}
  \item Local clustering of structurally similar identities
  \item Detection of reuse similarity across resolution graphs
  \item Drift tolerance bounds in RuleEvolution
\end{itemize}

Identities that lie within $\mathcal{U}(\chi_0, \epsilon)$ of an anchor may be stabilized through reuse.

\section{Trajectories and the Invariant Surface} \label{trajectories-and-the-invariant-surface}

Let $\chi_t([A])$ denote the coherence signature of $[A]$ at RuleEvolution time $t$.

\begin{definition}[Drift] \label{def:drift}
The drift of an identity is defined as:
\begin{equation} \label{eq:drift}
\partial \chi([A]) := \chi_{t+1}([A]) - \chi_t([A])
\end{equation}
\end{definition}

We classify identities based on drift behavior:
\begin{itemize}
  \item \textbf{Saturated:} $\partial \chi = 0$
  \item \textbf{Reactive:} $\partial \chi$ bounded but nonzero
  \item \textbf{Collapsing:} $\partial \chi$ unbounded, or $\rho([A]) < \rho_c$
\end{itemize}

\begin{definition}[Invariant Surface $\mathcal{S}$] \label{def:invariant-surface}
We define the invariant surface of $\chi$-space as:
\begin{equation} \label{eq:invariant-surface}
\mathcal{S} := \{ \chi \in \chi\text{-space} \mid \partial \chi = 0 \}
\end{equation}
\end{definition}

This surface defines structural stability.  
Anchors and conserved identities lie on $\mathcal{S}$.

\begin{definition}[Revised Identity Domain $\Omega_3$] \label{def:omega3-revised}
We restate $\Omega_3$ as the set of all identities satisfying:
\begin{equation} \label{eq:omega3-closure}
\Omega_3 := \left\{ [A] \,\middle|\, \mathcal{L}([A]) \neq \emptyset,\ \partial \chi([A]) \text{ is bounded},\ \rho([A]) \geq \rho_c \right\}
\end{equation}
\end{definition}

This is the space of stable, coherence-resolved identities.  
Only identities in $\Omega_3$ are considered valid elements of $\chi$-space and subject to RuleEvolution.

\section{Minimality of the Coherence Signature} \label{minimality-of-the-coherence-signature}

The coherence signature $\chi([A]) := (X_h, X_c, X_\pi, X_\phi, X_\epsilon)$ defines the structural identity of every return-closed symbolic form in SCM.  
This section addresses a foundational question:

\textit{Why are exactly five components used? Why not three, or seven?}

We assert that this five-variable signature is both minimal and sufficient for resolving identity behavior in $\Omega_3$.  
Each component satisfies a distinct functional requirement:

\begin{description}
  \item[$X_h$ (Return Depth)] Captures structural loop complexity; required to distinguish shallow vs.\ deep identities.
  \item[$X_c$ (Coherence Strength)] Measures return path consistency; required to differentiate stable vs.\ noisy identities.
  \item[$X_\pi$ (Return Symmetry)] Detects directional imbalance; required to separate symmetric (e.g., reversible) from asymmetric identities.
  \item[$X_\phi$ (Fragility)] Encodes susceptibility to collapse; required to predict breakdown under deformation.
  \item[$X_\epsilon$ (Reuse Elevation)] Measures external support; required to identify elevated vs.\ internally stable structures.
\end{description}

We claim that no proper subset of these five is sufficient to classify all resolved identities within $\Omega_3$:

\begin{itemize}
  \item Omitting $X_\epsilon$ makes it impossible to distinguish elevated identities from composed ones.
  \item Omitting $X_c$ obscures whether coherence is consistent across return paths.
  \item Omitting $X_\pi$ erases directional structure and breaks the ability to detect symbolic charge.
\end{itemize}

These variables are functionally independent over $\chi$-space and span the complete classification range for:
\begin{itemize}
  \item Identity type: irreducible, composed, elevated
  \item Return behavior: saturated, reactive, collapsing
  \item Reuse structure: self-contained, clustered, anchor-dependent
\end{itemize}

A full formal proof of this minimality claim — including counterexamples under $\chi' \subset \chi$ — is provided in Appendix~A.

\section{Injectivity and Classification Limits} \label{injectivity-and-classification-limits}

While the coherence signature $\chi([A]) = (X_h, X_c, X_\pi, X_\phi, X_\epsilon)$ is minimal and sufficient to classify identities in $\Omega_3$, it is not injective.

That is, there exist distinct identities $[A] \ne [B] \in \Omega_3$ such that:
\begin{equation} \label{eq:chi-noninjective}
\chi([A]) = \chi([B])
\end{equation}

This occurs when distinct resolution graphs $G([A])$ and $G([B])$ yield the same coherence signature across all five dimensions.  
In such cases, $\chi([A])$ and $\chi([B])$ occupy the same point in $\chi$-space, even though $[A]$ and $[B]$ are structurally distinct.

\textbf{Conclusion:}
\begin{itemize}
  \item The resolution graph $G(R)$ is injective: each identity $[A] \in \Omega_3$ has a unique graph $G([A])$
  \item The coherence signature $\chi$ is not injective: multiple identities may share the same coherence signature
\end{itemize}

Thus, $\chi$-space defines a classification topology, not an identity space.  
It groups structurally distinct forms by behavioral similarity under reuse, drift, and return.

\section{Summary}

The coherence signature $\chi([A]) = (X_h, X_c, X_\pi, X_\phi, X_\epsilon)$ defines the structural identity of every return-closed form.  
The space of all such signatures — $\chi$-space — is a symbolic topology over which identities evolve, couple, drift, or collapse.

In the next chapter, we define RuleEvolution formally and classify dynamic identity behavior according to trajectories in $\chi$-space.
