\chapter{Partial Identity}

Not all symbolic structures in Ω₃ are fully independent. Some require
reuse to remain coherent. Others hover near collapse but persist due to
stabilization by surrounding anchors.

This chapter defines partial identity---a class of drift-prone,
reuse-dependent structures that resolve only within specific coherence
configurations. We also introduce proto-anchors, unstable patterns
reused by others, and the mechanism of drift suppression: how coherence
anchors stabilize fragile identities.

These concepts formalize the early phase of emergence---the symbolic
scaffolding that precedes particles, logic, or code.

\section{Definition of Partial Identity} \label{definition-of-partial-identity}

\begin{definition}[Partial Identity] \label{def:partial-identity}
Let $[A] \in \Omega_3$ be a resolved identity.  
We say that $[A]$ is a \emph{partial identity} if:
\begin{itemize}
  \item $\mathcal{L}([A])$ exists, but
  \item $\chi([A])$ is unstable unless reuse anchors are present,
  \item $X_\varepsilon([A]) > 0$, and
  \item $\partial \chi([A]) \ne 0$ under standalone reuse
\end{itemize}
In short: $[A]$ resolves return only when embedded within a coherence cluster.  
It cannot stabilize in isolation.
\end{definition}

\section{Reuse Dependence and Structural Context} \label{reuse-dependence-and-structural-context}

Let $[A]$ reuse a saturated anchor $[B]$ such that:

\begin{itemize}
  \item $[B]$ lies on the invariant surface $\mathcal{S}$, i.e., $\partial \chi([B]) = 0$
  \item $[B]$ appears in the resolution graph $G_{[A]}$
  \item Reuse of $[B]$ stabilizes $\chi([A])$ such that $\partial \chi([A]) \rightarrow 0$
\end{itemize}

Then $[A]$ is said to be \emph{reuse-locked}: coherence is inherited via coupling.  
This is the first form of coherence inheritance in SCM.  
Return no longer depends solely on internal structure, but on connection to a stable external identity.

\section{Drift Suppression via Anchor Coupling} \label{drift-suppression-via-anchor-coupling}

Let $[A]$ be a partial identity. If:
\begin{itemize}
  \item $\partial \chi([A]) \ne 0$ in isolation
  \item but $\partial \chi([A]) \rightarrow 0$ under reuse of $[B]$ with $\chi([B]) \in \mathcal{S}$
\end{itemize}

Then we define a \textit{drift suppression lock}.  
This mechanism:
\begin{itemize}
  \item Suppresses structural deformation in reuse-prone identities
  \item Stabilizes near-collapse structures across coherence clusters
  \item Creates symbolic analogues of binding, insulation, and symmetry breaking
\end{itemize}

Anchors must preserve identity under reuse.  
Inversion structures such as Euler pairs ($X_h = 1$, $X_\pi = -1$) resolve to $[0]$ and cannot function as coherence anchors.

\section{Proto-Anchors} \label{proto-anchors}

Let $[C]$ be a symbolic structure such that:
\begin{itemize}
  \item $\mathcal{L}([C]) = \emptyset$ (not yet resolved),
  \item $[C]$ appears in $G_{[A]}$ for multiple $[A] \in \Omega_3$,
  \item $[C]$ is reused but not yet stable
\end{itemize}

We call $[C]$ a \emph{proto-anchor}.

If, over subsequent RuleEvolution steps:
\begin{itemize}
  \item $[C]$ appears in more coherence paths,
  \item RuleEvolution reinforces closure paths to/from $[C]$,
  \item $\partial \chi([C]) \rightarrow 0$ and $\rho([C]) \geq \rho_c$,
\end{itemize}

then $[C]$ becomes a full \emph{coherence anchor}.  
This transition is the symbolic equivalent of structural emergence.

\section{Resolution Spectrum: Identity vs Collapse} \label{resolution-spectrum-identity-vs-collapse}

We now classify all symbolic forms by their return and reuse conditions:

\begin{table}[h!]
\centering
\begin{tabular}{l c c c l}
\toprule
\textbf{Form} & \textbf{Return Loop ($\mathcal{L}$)} & \textbf{$\partial \chi$ Behavior} & \textbf{$X_\varepsilon$} & \textbf{Status} \\
\midrule
Fully Resolved     & Exists           & $\partial \chi = 0$            & $\geq 0$         & Stable identity \\
Partial Identity   & Exists           & $\partial \chi \ne 0$ (in isolation) & $> 0$          & Contextually stable \\
Proto-Anchor       & Not yet exists   & $\partial \chi$ undefined       & ---              & Reuse scaffolding \\
Collapsed Structure & Fails           & $\partial \chi$ diverges        & ---              & Incoherent \\
\bottomrule
\end{tabular}
\caption{Classification of symbolic structures in $\Sigma^*$ based on return and reuse behavior.}
\end{table}

This spectrum explains why some structures persist only in certain reuse environments—  
and why most symbolic forms in $\Sigma^*$ never become identities.

\section{Structural Role of Partial Identity} \label{structural-role-of-partial-identity}

Partial identities:
\begin{itemize}
  \item Populate the boundary between incoherence and structure
  \item Enable compositional reuse without structural stability
  \item Bridge the gap between randomness and persistent identity
\end{itemize}

They are not errors—they are emergence pathways.

Most elevated identities begin as partial identities.  
Most anchors pass through a proto-anchor phase.

This chapter closes the foundational narrative of structural emergence.  
In the next chapter, we show how triplet reuse locks and coherence convergence  
produce the first fully symmetry-stabilized families of identity.
