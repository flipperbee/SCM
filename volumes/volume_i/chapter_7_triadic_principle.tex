\chapter{Triadic Principle}

Not all identities are structurally equivalent. Some resolve return
independently. Others rely on internal reuse, and some require external
anchoring to remain coherent.

This chapter formalizes the Triadic Classification of identity in SCM.
Every coherence-resolved identity falls into one of three mutually
exclusive categories:
- Irreducible: return-closed without reuse,
- Composed: return-closed through internal reuse only,
- Elevated: return-closed only through external reuse.

These classes reflect structural constraints in $\chi$-space and play a
central role in the emergence of particle structure and coherence
networks in later volumes.

\section{The Triadic Principle} \label{the-triadic-principle}

Let $[A] \in \Omega_3$. Then $[A]$ falls into exactly one of the following categories:

\begin{enumerate}
  \item \textbf{Irreducible:}  
  $[A]$ resolves return without reusing any other structure.

  \begin{itemize}
    \item $X_\varepsilon = 0$
    \item $\text{support}([A]) = \emptyset$
    \item $\mathcal{L}([A])$ contains no reused substructure
  \end{itemize}

  Structures with minimal return but full inversion (e.g., Euler pairs with $X_h = 1$ and $X_\pi = -1$) do not qualify as irreducible — they cancel rather than persist.

  \item \textbf{Composed:}  
  $[A]$ resolves return through reuse of internal substructures.

  \begin{itemize}
    \item $X_\varepsilon = 0$
    \item $\text{support}([A]) \ne \emptyset$
    \item All reused structures lie within $\mathcal{L}([A])$
  \end{itemize}

  \item \textbf{Elevated:}  
  $[A]$ resolves return only via support outside its minimal loop.

  \begin{itemize}
    \item $X_\varepsilon > 0$
  \end{itemize}
\end{enumerate}

We call this partition of $\Omega_3$ the \textit{Triadic Principle}.

\section{Formal Proof of Exclusivity and Exhaustiveness} \label{formal-proof-of-exclusivity-and-exhaustiveness}

Let $[A] \in \Omega_3$. The three cases defined by the Triadic Principle are:

\subsubsection*{Mutually Exclusive}

\begin{itemize}
  \item If $\text{support}([A]) = \emptyset$, then $[A]$ cannot be composed or elevated — it is irreducible.
  \item If $X_\varepsilon = 0$ and $\text{support}([A]) \ne \emptyset$, then $[A]$ is composed.
  \item If $X_\varepsilon > 0$, then $\text{support}([A])$ necessarily includes external structures — so $[A]$ is elevated.
\end{itemize}

\subsubsection*{Collectively Exhaustive}

\begin{itemize}
  \item Either $\text{support}([A])$ is empty or non-empty.
  \item If $\text{support}([A]) = \emptyset$, then $[A]$ is irreducible.
  \item If $\text{support}([A]) \ne \emptyset$, then:
  \begin{itemize}
    \item If $X_\varepsilon = 0$, $[A]$ is composed.
    \item If $X_\varepsilon > 0$, $[A]$ is elevated.
  \end{itemize}
\end{itemize}

\medskip

\noindent
Thus, every identity in $\Omega_3$ falls into exactly one of these three structural types.

\section{Examples and Structural Consequences} \label{examples-and-structural-consequences}

\begin{itemize}
  \item \textbf{Irreducible:}  
  Example: The photon $[\gamma]$ with $\mathcal{L}([\gamma]) = A \rightarrow B \rightarrow A$, $X_\varepsilon = 0$  
  Interpretation: Foundational identity; serves as a coherence baseline.

  \item \textbf{Composed:}  
  Example: $[C]$ with internal loop $A \rightarrow B \rightarrow C \rightarrow A$, where $A$ and $B$ are reused but internal.  
  Interpretation: Nested or modular structure.

  \item \textbf{Elevated:}  
  Example: $[D]$ requires reuse of external anchor $[E]$ for coherence closure.  
  Interpretation: Dependent identity; may be fragile or context-bound.
\end{itemize}

These structural classes determine:
\begin{itemize}
  \item Which identities can act as coherence anchors
  \item Which are reusable or self-contained
  \item Which are conditionally coherent or collapse-prone
\end{itemize}

\section{$\chi$-Space Signatures by Class} \label{chi-space-signatures-by-class}

\begin{table}[h!]
\centering
\begin{tabular}{l c c l}
\toprule
\textbf{Class} & \textbf{$X_\varepsilon$} & \textbf{Support($[A]$)} & \textbf{$\chi$-Space Behavior} \\
\midrule
Irreducible & $0$ & $\emptyset$ & Low-dimensional, stable \\
Composed    & $0$ & $\ne \emptyset$ (internal) & Moderate $X_h$, low $X_\phi$ \\
Elevated    & $> 0$ & $\ne \emptyset$ (external) & High $X_\varepsilon$, possibly high $X_\phi$ \\
\bottomrule
\end{tabular}
\caption{$\chi$-space signature characteristics for each identity class.}
\end{table}

Elevated identities tend to cluster near the collapse boundary (as $X_\phi$ increases),  
unless stabilized by anchors. Irreducible identities typically lie on or near the invariant surface $\mathcal{S}$,  
where $\partial \chi = 0$.

\section{Structural Implications}

The Triadic Principle forms the logical foundation for:

\begin{itemize}
  \item Coherence inheritance
  \item Reuse-based emergence
  \item Structural modularity
\end{itemize}

\textbf{Reuse hierarchy:}
\begin{itemize}
  \item Irreducible identities form the coherence base
  \item Composed identities build internal reuse loops
  \item Elevated identities float atop reuse scaffolds
\end{itemize}

In later chapters, this structure enables:
\begin{itemize}
  \item Definition of coherence anchors
  \item Stabilization of partial identities
  \item Formation of triplet reuse locks and symmetry-stabilized composites
\end{itemize}

\section{Summary}

Every resolved identity in SCM is either:

\begin{itemize}
  \item \textbf{Irreducible:} Return-closed and reuse-free
  \item \textbf{Composed:} Internally reused and stable
  \item \textbf{Elevated:} Externally reuse-dependent
\end{itemize}

This classification is exhaustive and exclusive.  
It is grounded in $\chi$-space structure and encoded directly in the coherence signature.

In the next chapter, we examine what lies between identity and collapse:  
partial identities, proto-anchors, and the fragile emergence of coherence from dependency.
