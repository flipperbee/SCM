\chapter{Resolution Domains}

In this chapter, we introduce a structural classification of symbolic forms based on return behavior.  
Given a symbolic structure $S \in \Sigma^*$ and a permitted rule set $R \subset \Sigma^*$,  
we define the \textit{resolution space} as the partitioning of $\Sigma^*$ into four coherence domains:  
$\Omega_1$, $\Omega_2$, $\Omega_3$, and $\Omega_-$.

These domains represent increasing levels of return coherence, from unresolved forms to stable identities, and finally to structural annihilation.

\section{Domain Definitions}

Let $\mathcal{L}(S)$ denote the set of permitted return paths from a structure $S$ back to itself, as defined in Chapter~2.

\begin{definition}[Unresolved Domain $\Omega_1$] \label{def:omega1}
The domain $\Omega_1$ consists of unresolved structures. We define:
\begin{equation} \label{eq:omega1}
S \in \Omega_1 \iff \mathcal{L}(S) = \emptyset
\end{equation}
These structures admit no return path under the current rule set $R$.  
They are unresolved, unstable, and cannot form identities.
\end{definition}

\begin{definition}[Unstable Return Domain $\Omega_2$] \label{def:omega2}
The domain $\Omega_2$ consists of partially or unstably resolved structures. We define:
\begin{equation} \label{eq:omega2}
S \in \Omega_2 \iff \mathcal{L}(S) \neq \emptyset \text{ but return is unstable or fragile}
\end{equation}
These structures return, but the path may be incoherent, high-drift, or exhibit unbounded fragility ($X_\phi \rightarrow \infty$).  
$\Omega_2$ captures structures on the boundary of resolution, including unstable loops and transient identity forms.
\end{definition}

\begin{definition}[Resolved Identity Domain $\Omega_3$] \label{def:omega3}
The domain $\Omega_3$ consists of fully resolved, coherence-stable identities. We define:
\begin{equation} \label{eq:omega3}
S \in \Omega_3 \iff \mathcal{L}(S) \neq \emptyset \text{ and return is coherent, stable, and bounded}
\end{equation}
These structures satisfy all return conditions: finite loop length, bounded fragility, and minimal reuse.  
$\Omega_3$ is the space of symbolic identity stability and the foundation for all coherent structure in SCM.
\end{definition}

\section{Structural Interpretation} \label{structural-interpretation}

The $\Omega$-partition provides a coarse symbolic topology over $\Sigma^*$.  
A structure transitions through coherence domains as stability increases:
\begin{equation} \label{eq:omega-transition}
\Omega_1 \rightarrow \Omega_2 \rightarrow \Omega_3
\end{equation}

These domains are not axiomatic.  
They emerge from the behavior of return paths under a given rule set $R$.

Volume~I focuses primarily on $\Omega_3$, the space of coherent identities.  
Later volumes may develop models for instability and partial identity in $\Omega_2$.

\section{Constructive Identity in $\Omega_3$}

We now show that the coherence-resolved identity space $\Omega_3$ is non-empty by constructing a simple identity from first principles.

\begin{definition}[Minimal Rule System] \label{def:minimal-rule-system}
Let the symbolic alphabet be:
\[
\Sigma = \{a, b\}
\]
Define the following transformation rules:
\begin{align}
T_1 &: a \rightarrow b \label{eq:rule-t1} \\
T_2 &: b \rightarrow a \label{eq:rule-t2}
\end{align}
Assume $T_1, T_2 \in R$, the enabled rule set.
\end{definition}

\begin{definition}[Constructed Identity] \label{def:constructed-identity}
Let $S = a$. Apply the sequence of transformations:
\[
a \xrightarrow{T_1} b \xrightarrow{T_2} a
\]
This defines the return path:
\begin{equation} \label{eq:return-loop-identity}
\mathcal{L}([S]) = \{ a \rightarrow b \rightarrow a \}
\end{equation}
This loop is finite, coherent, and composed entirely of enabled rules.
\end{definition}

\begin{definition}[Verification of $\Omega_3$ Membership] \label{def:omega3-verification}
We now verify that $[a]$ satisfies the criteria for membership in $\Omega_3$:
\begin{itemize}
  \item \textbf{Return path exists:} $\mathcal{L}([a]) \neq \emptyset$
  \item \textbf{Drift is zero:} $\chi([a])$ remains invariant under reuse
  \item \textbf{Collapse robustness:} $\rho([a])$ is finite and nonzero
\end{itemize}
Thus:
\begin{equation} \label{eq:a-in-omega3}
[a] \in \Omega_3
\end{equation}
\end{definition}

\begin{definition}[Excluded Structure: Euler Inversion] \label{def:euler-inversion-exclusion}
Consider instead the Euler-style loop:
\begin{align}
T_3 &: a \rightarrow -a \label{eq:rule-t3} \\
T_4 &: -a \rightarrow a \label{eq:rule-t4}
\end{align}
This inversion pair completes a path with maximal return asymmetry $X_\pi = -1$ and resolves via cancellation:
\begin{equation} \label{eq:euler-cancellation-example}
[a] + [-a] = [0]
\end{equation}
Such cancellation loops do not qualify as identities. They resolve to the null structure $[0]$ and are excluded from $\Omega_3$.
\end{definition}

This constructive example confirms that $\Omega_3$ contains nontrivial, coherence-resolved identities, and is therefore non-empty.

\section{Structural Annihilation and the $\Omega_-$ Domain}

In addition to the previously defined identity resolution spaces $\Omega_1$, $\Omega_2$, and $\Omega_3$, the discovery of symmetry-inverted return loops introduces a fourth domain of structural behavior: $\Omega_-$.

\begin{definition}[Symmetry-Inverted Transformation] \label{def:symmetry-inversion}
Let $[A]$ be a symbolic structure with a permitted transformation $T_1$ such that:
\begin{equation} \label{eq:symmetry-inversion}
T_1 : [A] \rightarrow [-A]
\end{equation}
This one-step symmetry inversion is coherent, permitted, and of minimal return depth ($X_h = 1$), but it does not return to $[A]$ and therefore does not constitute an identity.
\end{definition}

Now consider the loop formed by a pair of symmetry-inverted transformations:
\begin{align}
T_1 &: [A] \rightarrow [-A] \label{eq:annihilation-t1} \\
T_2 &: [-A] \rightarrow [A] \label{eq:annihilation-t2}
\end{align}

This 2-step loop is coherent and symmetric under inversion. If $[A]$ and $[-A]$ together resolve to the null identity $[0]$ under structural addition, we define this as a case of structural annihilation.

\begin{definition}[Annihilation Domain $\Omega_-$] \label{def:omega-minus}
The annihilation domain is defined as:
\begin{equation} \label{eq:omega-minus}
\Omega_- := \{ [A], [-A] \mid [A] + [-A] = [0] \}
\end{equation}
These structures meet coherence requirements: they have return paths and bounded fragility, but they are not identities. They do not return to themselves, but cancel to the null identity $[0]$.
\end{definition}

This domain is structurally coherent but non-persistent.  
$\Omega_-$ forms the annihilation boundary of $\Omega_3$ — the symbolic equivalent of coherence collapse via structural inversion.

\subsection*{Summary: Full Identity Resolution Topology}

The complete symbolic topology of identity resolution in SCM includes four regions:

\begin{itemize}
  \item $\Omega_1$: Unresolved structures (no return path)
  \item $\Omega_2$: Fragile or unstable coherence ($\partial \chi$ unbounded or $\rho < \rho_c$)
  \item $\Omega_3$: Stable identity (return exists, bounded drift, and robustness)
  \item $\Omega_-$: Coherence-resolved but annihilating (returns to $[0]$, not $[A]$)
\end{itemize}

Volume~I now defines $\Omega_-$ as the symbolic terminal class of coherence: fully allowed under return, but excluded from identity due to complete inversion.
