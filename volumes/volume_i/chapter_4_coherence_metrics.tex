\chapter{Coherence Metrics}

Now that identity domains $\Omega_1$, $\Omega_2$, and $\Omega_3$ have been introduced,  
we turn our attention to quantitative measures that apply within $\Omega_3$, the space of coherence-stable identity.  
Not all return-closed structures are equally robust.  
Some collapse under minimal symbolic deformation, while others exhibit persistent coherence across reuse and drift.

This chapter defines four coherence metrics that distinguish between fragile and stable identity behavior in $\Omega_3$.

The following metrics define key axes of identity coherence:

\begin{itemize}
  \item \textbf{Collapse robustness} ($\rho$): How far can an identity deform before it exits $\Omega_3$?
  \item \textbf{Fragility} ($X_\phi$): The inverse of robustness; susceptibility to symbolic collapse
  \item \textbf{Latency} ($\Lambda$): Total symbolic effort required to complete a return loop
  \item \textbf{Return symmetry} ($X_\pi$): Structural asymmetry in the permitted return cycle
\end{itemize}

These quantities form part of the coherence signature $\chi([A])$ and help define coherence surfaces in $\chi$-space.

\section{Collapse Robustness $\rho$}

Let $[A] \in \Omega_3$ be a resolved identity.  
Let $D([A], [A'])$ be a permitted deformation path, defined as any finite, directed path in the resolution graph $G(R)$ that starts at node A and ends at node A'.  
Each transformation $T_i$ in the path contributes symbolic effort $\text{effort}(T_i)$.

\begin{definition}[Collapse Robustness]
Collapse robustness $\rho([A])$ is defined as:
\begin{equation} \label{eq:collapse-robustness}
\rho([A]) := \inf \left\{ \varepsilon > 0 \,\middle|\, \exists [A'] \in D([A], [A']) \text{ such that } \mathcal{L}([A']) = \emptyset \right\}
\end{equation}
This is the minimal symbolic effort required to push $[A]$ into $\Omega_1$ — that is, to destroy return closure.
\end{definition}

We say that $[A]$ is \textit{collapse-robust} if:
\begin{equation} \label{eq:collapse-threshold}
\rho([A]) \geq \rho_c, \quad \text{for some coherence threshold } \rho_c > 0
\end{equation}

\section{Fragility $X_\phi$}

\begin{definition}[Fragility] \label{def:fragility}
Fragility is defined as the inverse of collapse robustness:
\begin{equation} \label{eq:fragility}
X_\phi([A]) := \frac{1}{\rho([A])}
\end{equation}
\end{definition}

\textbf{Interpretation:}
\begin{itemize}
  \item High $X_\phi$ $\Rightarrow$ identity collapses under minor deformation
  \item Low $X_\phi$ $\Rightarrow$ identity resists perturbation
\end{itemize}

As $[A]$ approaches the $\Omega_3 \rightarrow \Omega_2$ or $\Omega_1$ boundary, $X_\phi([A])$ diverges.  
This metric forms the first axis of the coherence signature $\chi([A])$.

\section{Latency $\Lambda$ and Symbolic Effort} \label{latency-lambda-and-symbolic-effort}

Each transformation $T \in R$ carries an assigned symbolic effort.  
We define this as a function:
\begin{equation} \label{eq:effort-function}
\text{effort} : \Sigma^* \rightarrow \mathbb{R}^+
\end{equation}

\begin{definition}[Latency] \label{def:latency}
The latency of a resolved identity $[A]$ is the minimal total effort required to complete a valid return loop:
\begin{equation} \label{eq:latency}
\Lambda([A]) := \min_{\mathcal{P} \in \mathcal{L}([A])} \sum_{T_i \in \mathcal{P}} \text{effort}(T_i)
\end{equation}
\end{definition}

If symbolic effort is uniform across all transformations (i.e., $\text{effort}(T) \equiv 1$ for all $T$), then latency reduces to return depth:
\begin{equation} \label{eq:latency-depth-equivalence}
\Lambda([A]) = X_h([A])
\end{equation}

\subsection{Effort Function} \label{effort-function}

Symbolic effort is defined structurally. Common examples include:
\begin{itemize}
  \item $\text{effort}(T) := \text{length}(T)$ — the number of primitive symbols in $T$
  \item $\text{effort}(T) := \text{entropy}(T)$ — the number of distinct symbols in $T$
\end{itemize}

These definitions allow $\Lambda([A])$ to be computed purely from symbolic structure.

\begin{definition}[Inverse Transformation] \label{def:inverse-transformation}
Let $T : S \rightarrow S'$ be a permitted transformation.  
We say that an inverse $T^{-1}$ exists if there is $T' \in \Sigma^*$ such that:
\begin{equation} \label{eq:inverse-transform}
T'(S') = S
\end{equation}
Then we write: $T' = T^{-1}$.
\end{definition}

\textit{Note:} Not all transformations have inverses.  
Return symmetry $X_\pi$ is computed using only known forward and reverse steps.  
When no inverse is present in $R$, that segment is treated as directionally asymmetric.
\section{Return Symmetry $X_\pi$}

Return symmetry quantifies the structural asymmetry of a permitted return loop.  
In SCM, this asymmetry plays a critical role in defining coherence behavior, interaction potential, and structural charge.

\begin{definition}[Return Symmetry] \label{def:return-symmetry}
Let $[A]$ be a symbolic structure with a permitted return path $\mathcal{L}([A])$.  
Let:
\begin{itemize}
  \item $F$ = number of forward-directed transformations (under $R$)
  \item $R$ = number of reverse-directed transformations (under $R^{-1}$)
\end{itemize}

Return symmetry is defined as:
\begin{equation} \label{eq:return-symmetry}
X_\pi([A]) := \frac{F - R}{F + R}
\end{equation}

This ratio measures the net directional bias of the return loop. It ranges from $-1$ to $+1$.
\end{definition}

\subsection*{Interpretation of $X_\pi$} \label{interpretation-of-xpi}

\begin{center}
\begin{tabular}{|c|l|l|}
\hline
\textbf{$X_\pi$ Value} & \textbf{Meaning} & \textbf{Interpretation} \\
\hline
$-1$ & Full inversion & Maximal asymmetry: $S \rightarrow -S$ (e.g., Euler pair) \\
$0$  & Symmetric      & Perfectly balanced return loop \\
$+1$ & Forward-only   & Source-directed coherence structure \\
\hline
\end{tabular}
\end{center}

A return loop with $X_\pi = -1$ corresponds to structural inversion.  
If the coherence pair resolves to $[0]$, this defines an $\Omega_-$-type annihilation path — not an identity.

\subsection*{Symmetry and Identity} \label{symmetry-and-identity}

Return symmetry does not determine whether identity exists, but it classifies identities once return closure is confirmed:
\begin{itemize}
  \item $[A]$ is an identity $\iff \mathcal{L}([A]) \neq \emptyset$ and $S_0 = S_k$
  \item If $X_\pi = 0$, then $[A]$ is structurally neutral
  \item If $X_\pi > 0$, $[A]$ acts as a coherence source
  \item If $X_\pi < 0$, $[A]$ behaves as a coherence sink
  \item If $X_\pi = -1$ and $S_k \neq S_0$, return resolves to $[0]$, not $[A]$
\end{itemize}

This last case is the Euler inversion: a coherence-valid path that defines structural annihilation.

$X_\pi$ is not merely a measure of asymmetry — it is a structural fingerprint of return logic:
\begin{itemize}
  \item It determines interaction directionality
  \item It distinguishes return from inversion, and identity from cancellation
\end{itemize}

In $\chi$-space, $X_\pi$ is a fully directional axis of coherence structure.  
Its endpoints, $[-1, +1]$, are not boundaries of collapse — but poles of inversion and emission.

\section{Summary}

These coherence metrics apply only to identities in $\Omega_3$.  
Each plays a role in defining the coherence structure of $[A]$ and its behavior under reuse, drift, or collapse.

\begin{center}
\begin{tabular}{|l|l|l|}
\hline
\textbf{Metric} & \textbf{Meaning} & \textbf{Symbol} \\
\hline
Collapse robustness & Minimum deformation before failure & $\rho([A])$ \\
Fragility            & Inverse robustness                 & $X_\phi([A])$ \\
Latency              & Total symbolic effort to return    & $\Lambda([A])$ \\
Return symmetry      & Directional asymmetry in return loop & $X_\pi([A])$ \\
\hline
\end{tabular}
\end{center}
