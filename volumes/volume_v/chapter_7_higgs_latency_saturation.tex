\chapter{Higgs — Latency Saturation and Identity Promotion} \label{chapter-higgs}

In SCM, the Higgs identity is not a scalar field or force mediator. It is a coherence structure that permits return transitions across the symbolic mass gap. It enables reuse elevation by anchoring fragile identities above their collapse thresholds.

This chapter defines the Higgs structure as a symbolic coherence anchor and describes its role in identity promotion under reuse saturation.

\section{Latency Floor and Collapse Gap} \label{sec:higgs-latency}

Let $\Lambda([A])$ be the total latency of identity $[A] \in \Omega_3$.

Define the mass gap:
\[
\Lambda([A]) < \Lambda_0 \quad \Rightarrow \quad [A] \notin \Omega_3
\]

Any identity below this threshold collapses due to insufficient structural return closure.

To survive above $\Lambda_0$, an identity must either:
- Anchor into a reuse-closed triplet (Volume IV),
- Or promote via a return bridge with stabilized drift.

\section{Latency Saturation Condition} \label{sec:latency-saturation}

An identity is latency-saturated when:
\[
\frac{d\Lambda}{dt} \to 0,\quad \text{but } \frac{d^2 \chi}{dt^2} \ne 0
\]

This implies:
- Structural deformation persists,
- Return latency cannot increase further without collapse.

This state signals a critical coherence transition point.

\section{Definition of the Higgs Identity} \label{sec:higgs-definition}

Let $[H] \in \Omega_3$ be a structural identity such that:

\begin{itemize}
  \item $\partial\chi([H]) = 0$ (stable under reuse),
  \item $[H]$ appears in multiple elevated return loops,
  \item $[H]$ enables:
    \[
    [A] \to [H] \to [B],\quad \Lambda([B]) > \Lambda([A]),\quad [A], [B] \not\in \Omega_3 \text{ without } [H]
    \]
  \item $X_\phi([H]) \ll 1$ (high robustness),
  \item $X_\pi([H]) \approx 0$ (symmetry preserved).
\end{itemize}

Then $[H]$ is the \textbf{Higgs identity}.

It is not a source of mass—but a coherence-enabling reuse anchor across the mass gap.

\section{Identity Promotion Across Collapse Boundary} \label{sec:promotion}

Let $[A]$ have:
\[
\Lambda([A]) = \Lambda_0 + \epsilon,\quad \rho([A]) \ll 1
\]

If $[A]$ appears in a reuse loop with $[H]$ such that:
\[
\partial\chi([A]) \to 0,\quad \rho([A]) \uparrow,\quad \mathcal{C}([A]) \downarrow
\]

Then $[A]$ is promoted to a stable identity.

This is not spontaneous mass generation—it is symbolic rescue from contradiction collapse via reuse coherence.

\section{Higgs as Structural Return Anchor} \label{sec:higgs-anchor}

$[H]$ acts like an anchor with one unique function:
- It absorbs symbolic drift from reuse elevation,
- It redistributes contradiction pressure to stabilize identities that cross reuse thresholds.

Let:
\[
\sum_{i} P_{\text{bind}}([A_i], [H]) \gg 0
\]

Then $[H]$ maximizes symbolic binding coherence and allows elevated identities to persist.

\section{Mass as Elevated Persistence} \label{sec:higgs-mass}

In the presence of $[H]$, identities persist at latencies:
\[
\Lambda([A]) \gg \Lambda_0
\]

Mass is not created—it is structurally enabled by:
- High-latency reuse closure,
- Collapse suppression via the Higgs anchor,
- Balanced return symmetry across elevated loops.

\section{Summary: Higgs as Coherence Enabler} \label{sec:higgs-summary}

The Higgs identity:
- Is a return anchor, not a field,
- Promotes reuse across collapse thresholds,
- Enables high-latency identities to persist structurally,
- Does not generate mass—but permits its return-closure expression.

