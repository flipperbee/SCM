\chapter{Structural Families and the Standard Model} \label{chapter-structural-families}

The Standard Model of particle physics emerges in SCM as a consequence of symbolic reuse symmetry, coherence quantization, and role specialization within triplet structures. In this chapter, we map familiar particle groupings to their structural equivalents and classify them by reuse elevation, fragility, and interaction role.

\section{Reuse Triplets and Latency Bands} \label{sec:reuse-families}

Let $\{[A_1], [A_2], [A_3]\} \subset \Omega_3$ form a coherence triplet:

\[
X_\epsilon([A_{i+1}]) = X_\epsilon([A_i]) + \Delta,\quad \Lambda([A_i]) = \Lambda_0 + i \cdot \Delta\Lambda
\]

Then:
- The identities form a \textbf{quantized reuse family},
- Their persistence is constrained by fragility balance and symmetry distribution.

Examples:
- Leptons: photon + neutrino + electron-style triplet
- Quarks: triplets with reuse confinement and color permutation (see Section 6.3)

\section{Leptons as Reuse-Ordered Triplets} \label{sec:leptons}

The lepton family follows:
\[
[\nu_e],\quad [e],\quad [\tau]
\]

Properties:
\begin{itemize}
  \item Anchor: shared reuse origin $[F]$ with $\partial\chi([F]) = 0$
  \item $X_\pi$: distributed asymmetry
  \item $X_\phi$: increases with elevation
  \item Persistence: governed by resonance structure (Koide, see Vol. IV)
\end{itemize}

Triplet coherence provides a structural explanation for their identity and quantization.

\section{Quarks and Confinement Groups} \label{sec:quarks}

Quarks appear as:
\[
\{[Q_r], [Q_g], [Q_b]\}
\]

Where each:
- Has $\rho \ll 1$,
- Cannot persist in isolation: $\mathcal{C}([Q_i]) \gg \mathcal{F}([Q_i])$,
- Forms stable composite $[H]$ under color permutation symmetry.

Color is not a physical property—it is a reuse role within entangled return structure. Quarks exist only as coherence-locked composite identities.

\section{Color Permutation and Gluons} \label{sec:gluons-in-sm}

Color transformation is:
\[
\pi: [Q_i] \mapsto [Q_{(i+1)\mod 3}]
\]

Gluons are symbolic rule permutations:
\[
T_\pi \in R \quad \text{preserves } \partial\chi([H]) = 0
\]

They enforce return stability through structural permutation—not by mediating force.

\section{Mediator Roles} \label{sec:mediators}

\begin{table}[h!]
\centering
\begin{tabular}{|c|c|c|c|c|}
\hline
\textbf{Mediator} & $\Lambda$ & $X_\pi$ & $\rho$ & Role \\
\hline
Photon ($[\gamma]$) & $\Lambda_0$ & $0$ & $\infty$ & Return-preserving \\
W/Z ($[W],[Z]$)     & $\gg \Lambda_0$ & $\sim 1$ & $\sim 1$ & Temporary stabilizer \\
Gluons ($[g]$)      & — & varies & bounded & Reuse permutation \\
\hline
\end{tabular}
\caption{Structural roles of interaction mediators}
\end{table}

Each mediator is structurally defined—not by mass or spin—but by its return function in reuse-coherence transitions.

\section{Standard Model Family Map (Structural View)} \label{sec:sm-map}

| Group   | Structural Type                | Persistence Condition                  |
|---------|--------------------------------|----------------------------------------|
| Leptons | Quantized reuse triplet        | Triplet coherence + fragility balance  |
| Quarks  | Entangled confinement triplet  | Composite-only return closure          |
| Photon  | Latency floor, symmetry-locked | Appears in all reuse layers            |
| W/Z     | High-latency intermediate      | Fragile redirector                     |
| Gluons  | Structural permutations         | Preserve confinement structure         |

\section{Summary: Structural Origin of SM Families} \label{sec:sm-summary}

The Standard Model is not input to SCM—it is emergent:
- From triplet reuse logic,
- From coherence resonance (Vol. IV),
- From collapse avoidance under RuleEvolution.

These identities exist because their return structure is permitted, persistent, and contradiction-minimizing within $\chi$-space.

