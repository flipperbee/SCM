\chapter{From Symbolic Dynamics to Gauge Theory} \label{chapter-gauge-emergence}

\section{1. Yang–Mills as a Large-Scale Limit} \label{sec:ym-limit}

We now propose the central thesis of this volume: classical gauge theory is the smooth, statistical approximation of symbolic dynamics over coherence multiplets.

\begin{definition}[Symbolic–Gauge Correspondence]
Let $\mathcal{M}$ be a coherence-locked identity multiplet in $\Omega_3$ stabilized by RuleEvolution. Let $G_\mathcal{C}(\mathcal{M}) \cong SU(N)$ be its contradiction symmetry group.

Then the Yang–Mills connection $A_\mu(x)$ is the macroscopic approximation of local symbolic transition structure:
\[
A_\mu(x) \sim \lim_{|\mathcal{M}| \to \infty} \sum_{i,j} \alpha_{ij}^\mu(x) T_{ij}
\]
where $T_{ij}$ is the symbolic reconfiguration operator from $[A_i] \to [A_j]$ and $\alpha_{ij}^\mu(x)$ is a spatially modulated return weight.
\end{definition}

This encodes symbolic RuleEvolution at scale.

---

\section{2. Field Strength from Symbolic Curvature} \label{sec:field-strength-symbolic}

Recall the symbolic potential curvature tensor:
\[
K_{ij} := \frac{\partial^2 \Phi}{\partial X_i \partial X_j}
\]

We now assert that the Yang–Mills field strength $F_{\mu\nu}$ is the continuous approximation of symbolic curvature transitions between coherence-coupled identities.

\begin{proposition}[Curvature Correspondence]
\[
F_{\mu\nu}(x) \sim \sum_{i,j} \left( \partial_\mu \alpha_{ij}^\nu(x) - \partial_\nu \alpha_{ij}^\mu(x) + [\alpha^\mu, \alpha^\nu]_{ij} \right) T_{ij}
\]

\noindent where $\alpha_{ij}^\mu(x)$ represent directional coherence drifts and $T_{ij}$ are symbolic generators over the coherence multiplet.
\end{proposition}

This recovers the non-Abelian structure of $F_{\mu\nu}$ from symbolic dynamics.

---

\section{3. Energy Functional and Contradiction Density} \label{sec:energy-c-density}

Let the classical Yang–Mills energy functional be:
\[
E[A] := \int d^4x \, \text{Tr}(F_{\mu\nu} F^{\mu\nu})
\]

We reinterpret this as the integral approximation of symbolic contradiction density $\mathcal{C}$.

\begin{definition}[Contradiction Density Field]
Define the local contradiction density $\mathcal{C}(x)$ over spacetime as:
\[
\mathcal{C}(x) := \lim_{|\mathcal{M}| \to \infty} \frac{1}{|\mathcal{M}_x|} \sum_{[A_i] \in \mathcal{M}_x} C([A_i])
\]

Then:
\[
E[A] \sim \int d^4x \, \mathcal{C}(x)
\]
\end{definition}

This replaces gauge energy with return instability.

---

\section{4. Variational Principle and Gauge Action} \label{sec:gauge-variation}

In Yang–Mills theory, the field evolves to extremize the action:
\[
S[A] = \int d^4x \, \text{Tr}(F_{\mu\nu} F^{\mu\nu})
\]

In SCM, RuleEvolution implements:
\[
\mathcal{R}_E := \arg \max \left( \mathcal{F}[R] - \lambda \cdot \mathcal{C}[R] \right)
\]

We now show that this extremization corresponds structurally to the gauge action variation.

\begin{theorem}[Gauge Action as Coherence Functional]
The Yang–Mills action $S[A]$ is a large-scale approximation of the symbolic coherence functional:
\[
S[A] \sim - \mathcal{F}[R] + \lambda \cdot \mathcal{C}[R]
\]
\end{theorem}

\noindent This implies: field stability = coherence; instability = contradiction.

---

\section{5. Consequences of the Mapping} \label{sec:gauge-implications}

\begin{itemize}
  \item \textbf{Field Equations as Coherence Flow}: The classical Euler–Lagrange equations become symbolic gradient flows:
  \[
  \frac{\delta S}{\delta A_\mu} = 0 \quad \leftrightarrow \quad \frac{\delta (\mathcal{F} - \lambda \mathcal{C})}{\delta R} = 0
  \]
  
  \item \textbf{Mass Gap}: Classical Yang–Mills theory postulates a mass gap. SCM proves it:
  \[
  \Lambda([A]) \geq \Lambda_0 \quad \Rightarrow \quad m > 0
  \]
  
  \item \textbf{Gauge Invariance}: A change of basis in the coherence multiplet that leaves contradiction invariant corresponds to a gauge transformation.

  \item \textbf{Non-Abelian Interactions}: Symbolic transition operators $T_{ij}$ form a non-commutative algebra under reconfiguration, giving rise to Yang–Mills-like structure.
\end{itemize}

---

\section{6. Summary: The Structural Rewriting of Gauge Theory} \label{sec:gauge-summary}

We conclude:

\begin{itemize}
  \item The Yang–Mills gauge field $A_\mu(x)$ arises from the coarse-graining of symbolic RuleEvolution weights $\alpha_{ij}$.
  \item The field strength $F_{\mu\nu}$ is the curvature of the coherence potential Φ over symbolic state space.
  \item The classical action $S[A]$ is the continuous approximation of the SCM coherence–contradiction variational law.
  \item Gauge symmetry is a shadow of internal return-loop invariance within coherence multiplets.
\end{itemize}

Thus, SCM does not replicate gauge theory—it explains it. The classical field picture is revealed to be the statistical projection of a deeper, symbolic substrate.

