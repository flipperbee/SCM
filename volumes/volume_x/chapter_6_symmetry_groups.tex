\chapter{Symmetry Groups from Return Closure} \label{chapter-symmetry-return-closure}

Classical gauge theories begin with a symmetry group \( G \) and define the field dynamics accordingly. SCM inverts this logic. In SCM, stable reuse structures—identities embedded in return-closed loops—give rise to \emph{emergent} symmetries.

In this chapter, we formally define return-induced invariance and show how symmetry groups arise from the structural closure of reuse systems. These groups play the role of gauge symmetries: transformations that preserve coherence.

\section{Return Loops and Invariance}

Let \( \{[A_i]\} \subset \Omega_3 \) be a set of identities participating in a return-closed loop \( \mathcal{L} \). The coherence signature of each \( [A_i] \) is:

\[
\chi([A_i]) = (X_h, X_\pi, X_\phi, X_\epsilon, X_c)
\]

\begin{definition}[Return Closure]
A set \( \{[A_i]\} \) is return-closed if:
\begin{itemize}
  \item There exists a return path \( \mathcal{L} \) such that \( \mathcal{L}([A_i]) \subset \{[A_j]\} \) for all \( i \),
  \item The total contradiction \( \mathcal{C}(\mathcal{L}) < \infty \),
  \item The drift vector \( \vec{\partial \chi}([A_i]) = 0 \) for all \( i \).
\end{itemize}
\end{definition}

We now ask: what structural transformations of \( \{[A_i]\} \) leave coherence invariant?

\begin{definition}[Return-Preserving Transformation]
Let \( \sigma \) be a permutation or symbolic mapping acting on \( \{[A_i]\} \). We say that \( \sigma \) is return-preserving if:
\[
\forall \mathcal{L} \in \mathcal{R}(\{[A_i]\}), \quad \sigma(\mathcal{L}) \in \mathcal{R}(\{[A_i]\})
\]
and
\[
\Phi(\sigma([A_i])) = \Phi([A_i]) \quad \forall i
\]
\end{definition}

The set of all such transformations forms a group under composition.

\begin{proposition}
The set of all return-preserving transformations \( \mathcal{G} \) acting on \( \{[A_i]\} \) forms a group:
\[
\mathcal{G} := \{ \sigma : \sigma \text{ is return-preserving on } \{[A_i]\} \}
\]
\end{proposition}

\section{Emergent Gauge Symmetry}

\paragraph{Key Observation.} SCM identities form stable structures only under strict reuse constraints. The set \( \{[A_i]\} \) must satisfy coherence, latency, and reuse balance:
\begin{itemize}
  \item Total latency quantized: \( \sum_i \Lambda([A_i]) = \text{const} \)
  \item Total asymmetry balanced: \( \sum_i X_\pi([A_i]) = 0 \)
  \item Reuse symmetry: \( \forall i, \exists j \ne i : [A_i] \to [A_j] \)
\end{itemize}

These invariances naturally lead to discrete or continuous symmetries over the reuse space.

\begin{definition}[Coherence Symmetry Group]
Let \( \mathcal{L} \) be a return-closed loop of identities. The coherence symmetry group of \( \mathcal{L} \) is:
\[
G_\mathcal{L} := \text{Aut}(\mathcal{L}) = \{ \sigma \mid \sigma : \mathcal{L} \to \mathcal{L}, \; \Phi \circ \sigma = \Phi \}
\]
\end{definition}

This is the SCM analogue of a gauge symmetry: the set of transformations that preserve the coherence field.

\section{Triplet Closure and SU(3)}

Let \( \{[q_1], [q_2], [q_3]\} \) be a coherence-locked reuse triplet with:
\begin{itemize}
  \item Balanced latency: \( \Lambda([q_1]) = \Lambda([q_2]) = \Lambda([q_3]) \),
  \item Balanced charge: \( X_\pi([q_i]) = -X_\pi([q_j]) \), etc.,
  \item Reuse circularity: \( [q_1] \to [q_2] \to [q_3] \to [q_1] \).
\end{itemize}

Such a structure has return-preserving symmetries under cyclic permutation and internal interchange, forming the group:

\[
\text{Symmetry}(\{[q_i]\}) \cong \text{SU}(3)
\]

This coherence group acts on the triplet's symbolic return channels and is responsible for confining reuse into a closed, symmetric structure.

\paragraph{Implication.}
The SU(3) symmetry of the strong force emerges as the structural group of coherence-preserving transformations over a reuse triplet.

\section{Toward Higher Groups: SU(5), SU(10)}

In Section~\ref{chapter-multiplet-symmetry}, we will explore whether more complex, coherence-balanced multiplets—such as the 5-particle lepton family or a 10-particle GUT candidate—admit invariance under larger symmetry groups.

The working hypothesis is:

\begin{quote}
\textit{Any return-closed coherence multiplet with constant latency and balanced asymmetry defines a unique symmetry group via its return-preserving transformations.}
\end{quote}

This group is structural—not imposed—and encodes all coherence-respecting transitions between the multiplet elements.

\section{Conclusion}

SCM does not begin with a symmetry group. It derives one.

Stable reuse families, once formed, admit internal permutations that leave their coherence field invariant. These permutations form groups. The strongest of these are the return-preserving transformations, which define the structural analogues of gauge symmetries.

Thus, symmetry is not foundational—it is emergent from return closure.

