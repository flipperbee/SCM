\section{Structural Coherence Mathematics
(SCM)}\label{structural-coherence-mathematics-scm}

\section{Volume I: Identity by permitted
return}\label{volume-i-identity-by-permitted-return}

\section{Version: 2.0}\label{version-2.0}

\section{CoherenceForHumanity.org}\label{coherenceforhumanity.org}

\section{Author: Boer, Flip}\label{author-boer-flip}

\section{\texorpdfstring{\url{https://orcid.org/0009-0000-0255-8823}}{https://orcid.org/0009-0000-0255-8823}}\label{httpsorcid.org0009-0000-0255-8823}

\section{\texorpdfstring{Date: July 8\textsuperscript{th}
2025}{Date: July 8th 2025}}\label{date-july-8th-2025}

\section{Copyright and License}\label{copyright-and-license}

© 2025 CoherenceForHumanity.org

All rights reserved.

This document presents original research in symbolic identity logic and
coherence systems under the Structural Coherence Math (SCM) framework.

The author retains full copyright over all content, structure,
definitions, and simulations presented herein. Redistribution,
commercial use, or derivative work is not permitted without prior
written consent of the author.

Licensed under the Creative Commons
Attribution-NonCommercial-NoDerivatives 4.0 International License. You
may share this work with attribution, but you may not modify it or use
it commercially

\section{\texorpdfstring{\hfill\break
Chapter 1 \textbar{} The Symbolic
Foundation}{ Chapter 1 \textbar{} The Symbolic Foundation}}\label{chapter-1-the-symbolic-foundation}

This chapter builds the symbolic ground floor of Structural Coherence
Mathematics. We begin not with axioms, but with unstructured
symbols---primitives that do not assert meaning, position, or truth.
From this substrate, we define structure as a transformation, not an
object.\\
\strut \\
The core idea introduced here is that identity is not assumed---it is
permitted. A symbolic form does not exist because we label it, but
because it returns to itself through a series of allowed
transformations. That return loop, if it exists, is the only admissible
definition of persistence in SCM.\\
\strut \\
This notion is developed step by step: first defining symbols, rules,
and transformation sequences; then formulating what it means to return;
and finally stating the structural condition under which identity
becomes real.\\
\strut \\
We call this the emergence threshold---the point at which symbolic
resolution becomes possible. It is from this threshold that all future
structure will unfold.

\section{1.1 \textbar{} Symbols and
Structures}\label{symbols-and-structures}

Let Σ be a finite alphabet of primitive symbols:

  Σ = \{ s₁, s₂, ..., sₙ \}

Each symbol s ∈ Σ is a syntactic object with no internal content or
assigned semantics. Symbols are not interpreted---they are transformed.

A symbolic structure S is a finite ordered sequence of symbols from Σ. Σ
is a syntactic alphabet only. It carries no semantic meaning---only
symbolic form. All structure and identity arise solely from permitted
transformations, not from the interpretation of individual symbols.

  S = (sᵢ₁, sᵢ₂, ..., sᵢₖ), where sᵢⱼ ∈ Σ, k ∈ ℕ

We define the set of all symbolic structures as:

  Σ* := the set of all finite sequences over Σ, including sequences of
length 1.

Every symbol s ∈ Σ induces a minimal structure S = (s) ∈ Σ*.\\
Thus, a single symbol can be treated as a structure of length one.

All formal definitions of transformation, return, and identity apply to
such structures equally.

\section{1.2 \textbar{} Rule Sets and Permitted
Transformations}\label{rule-sets-and-permitted-transformations}

In Structural Coherence Mathematics (SCM), transformations are not
axioms. They are not imposed. They are discovered---as symbolic
structures that permit return.

Let Σ be the finite alphabet of primitive symbols. Define Σ* as the set
of all finite symbolic structures over Σ. Then:

\subsection{Definition --- Rule Set R}\label{definition-rule-set-r}

The rule set R ⊆ Σ* is the dynamically active subset of symbolic
structures that currently function as enabled transformations under
structural coherence.

\subsection{Enabled vs. Disabled
Rules}\label{enabled-vs.-disabled-rules}

A symbolic structure T ∈ Σ* becomes a rule only if it participates in a
coherence-valid return path. That is:

- T is enabled ⇔ T participates in at least one permitted transformation
in a valid return loop 𝓛({[}A{]}) for some {[}A{]} ∈ Ω₃.

- T is disabled ⇔ T ∉ R, i.e. T does not currently contribute to
coherence resolution.

This enables SCM to treat rules as symbolic structures---not external
declarations. Every rule is itself a symbol. What makes it a rule is
structural participation in return.

\subsection{Definition --- Symbolic
Transformation}\label{definition-symbolic-transformation}

Let T ∈ Σ* be a symbolic structure interpreted as a transformation rule.
We define a transformation T as an ordered symbolic substitution
pattern:\\
\strut \\
T := (α → β), α, β ∈ Σ*

Let S₁ ∈ Σ* be a symbolic form. T acts on S₁ if α appears as a
contiguous subsequence in S₁. The result S₂ is formed by replacing the
first occurrence of α with β.

We write:\\
T: S₁ → S₂ ⇔ S₂ = S₁ with α → β applied once at first match

This defines symbolic transformation as pattern substitution, without
invoking numerical or algorithmic semantics.

\subsection{Definition --- Permitted Transformation
(R-enabled)}\label{definition-permitted-transformation-r-enabled}

Let T ∈ R be an enabled transformation rule.

Then T defines a permitted transformation:\\
T: S₁ → S₂, where S₁, S₂ ∈ Σ*

The system permits S₁ → S₂ only if T ∈ R, and S₂ = T(S₁).

If no such T exists in R, then the transformation is not permitted. In
that case, coherence fails, and identity resolution is not possible.

\subsection{Initial Condition: No Rules
Exist}\label{initial-condition-no-rules-exist}

At system origin, R = ∅. No return paths are possible. No
transformations are enabled. This is the coherence vacuum---the state of
pure symbolic potential with no structure.

\subsection{Emergence of Enabled
Rules}\label{emergence-of-enabled-rules}

Once a permitted return loop is discovered---that is, some structure S ∈
Σ* satisfies:

∃ 𝓛(S): S → ⋯ → S, with all steps permitted

then the set of symbolic structures participating in that loop are
promoted into R. This forms the initial rule set:

R₁ := \{ T ∈ Σ* \textbar{} T appears in a coherence-valid return path \}

From here, RuleEvolution (chapter 5) takes over---dynamically enabling
or disabling symbolic structures in Σ* as the system evolves.

\subsection{Summary of Rule Ontology}\label{summary-of-rule-ontology}

This table summarizes concepts and definitions

\begin{itemize}
\item
  Symbol: Any structure in Σ*
\item
  Potential Rule: Any symbolic structure T ∈ Σ* that could function as a
  transformation
\item
  Enabled Rule: A potential rule currently in the active rule set R (T ∈
  R)
\item
  Disabled Structure: A structure not currently in R; not functioning as
  a rule
\item
  R: The current set of enabled transformations: R ⊆ Σ*
\end{itemize}

There are no axiomatic rules in SCM. All transformations must be
earned---through their ability to support return. Only enabled potential
rules are rules. All others are symbolic structures without coherence
function at the current system state.

\section{1.3 \textbar{} Return Paths, Identity, and Structural
Inversion}\label{return-paths-identity-and-structural-inversion}

In Structural Coherence Mathematics (SCM), identity is not assumed---it
emerges from structure. A symbolic form becomes an identity only when it
returns to itself through a permitted sequence of transformations.

Let 𝒫(S) denote a finite sequence of permitted transformations:

𝒫(S) = S₀ → S₁ → ... → Sₖ, where each (Sᵢ → Sᵢ₊₁) is permitted by some T
∈ R\\
\strut \\
We define a return path for a structure S as a transformation sequence
such that:

\begin{quote}
- S₀ = S,\\
- Sₖ = S,\\
- k ≥ 1,\\
- All intermediate steps are permitted by R.
\end{quote}

\subsection{Return Path Set}\label{return-path-set}

Let Σ be the symbolic alphabet. Let S ∈ Σ* be any symbolic structure. We
define the return path set of S as:

𝓛(S) := \{ 𝒫 \textbar{} 𝒫 is a permitted path such that S₀ = Sₖ = S \}

That is, 𝓛(S) is the set of all finite, permitted sequences of
transformations that begin and end at S. A structure S is said to have
return closure if 𝓛(S) ≠ ∅.

We say that:

\begin{quote}
- S is return-closed if 𝓛(S) ≠ ∅.\\
- The minimal return loop of S, if it exists, is the shortest 𝒫 ∈
𝓛(S).\\
- The return depth Xₕ(S) := \textbar 𝒫\textbar{} for the minimal such
loop.
\end{quote}

\subsection{Identity}\label{identity}

A symbolic structure S is provisionally called an identity if:

\begin{quote}
- There exists at least one permitted return path: 𝓛(S) ≠ ∅.\\
- The return is to the same structure\\
- Coherence is preserved: return does not collapse
\end{quote}

This is the starting point for structural persistence.

\section{1.4 \textbar{} Inversion vs. Identity --- The Role of
Euler\textquotesingle s
Formula}\label{inversion-vs.-identity-the-role-of-eulers-formula}

Euler's identity: e\^{}\{iπ\} + 1 = 0 expresses a maximally asymmetric
transformation. It rotates the identity {[}1{]} into its structural
inverse {[}--1{]}, and their composition resolves into zero:

{[}1{]} + {[}--1{]} = {[}0{]}

This represents the shortest possible returning loop in symbolic
structure:

\begin{quote}
- One transformation: {[}1{]} → {[}--1{]}\\
- Composition yields the null identity {[}0{]}\\
- Return symmetry X\_π = --1 (perfect inversion)\\
- Return depth Xₕ = 1
\end{quote}

But this is not identity. Why?

\begin{quote}
- The path returns not to the same structure, but to its inverse\\
- The final result is {[}0{]}, not {[}1{]}\\
- This is cancellation, not persistence
\end{quote}

\subsection{Euler Inversion Loop}\label{euler-inversion-loop}

An Euler loop is a coherence-valid path of minimal length (Xₕ = 1) with
maximal return asymmetry (X\_π = --1), which maps a structure to its
inverse and produces the null identity:

S → --S, S + (--S) = {[}0{]}

This loop is coherent, but does not satisfy the definition of identity.
It defines structural annihilation, not return.

\subsection{Null Identity {[}0{]}}\label{null-identity-0}

The null identity {[}0{]} is the structural cancellation of a form and
its inverse:

{[}S{]} + {[}--S{]} := {[}0{]}

{[}0{]} has no return paths:

\begin{quote}
- 𝓛({[}0{]}) = ∅\\
- No latency\\
- No reuse\\
- No persistence
\end{quote}

It represents the coherence vacuum: a resolved cancellation. Not all
return implies identity. Identity requires return to self. Inversion
yields cancellation---not persistence. This distinction is the
foundation of SCM's symbolic algebra. It makes identity emergent, and
zero structural.

\section{1.5 \textbar{} Structural Emergence
Threshold}\label{structural-emergence-threshold}

Let R be a given Rule Set. Then:\\
- If R = ∅, then 𝓛(S) = ∅ for all S; no identities exist.\\
- The emergence threshold is the minimal cardinality of R such that:\\
  ∃ S ∈ Σ* with 𝓛(S) ≠ ∅\\
\strut \\
This defines the onset of structural coherence.\\
It is not axiomatic---it is the first point at which a system permits
identity by return.

To resolve the apparent circularity of requiring enabled rules to permit
return---yet requiring return to enable rules---we define the structural
bootstrap condition:

Let Σ* be the set of all finite symbolic structures. A transformation T
∈ Σ* becomes a candidate for R₁ if there exists a structure S ∈ Σ* and
T′ ∈ Σ* such that:

  T: S → S′ and T′: S′ → S

Such 2-step symmetric candidates form the seed of the system. We define:

 R₁ := \{ T ∈ Σ* \textbar{} T participates in a minimal symmetric return
loop S → S′ → S \}\\
\strut \\
This symbolic emergence condition resolves the bootstrap problem without
requiring external intervention.

\section{1.6 \textbar{} Summary}\label{summary}

\begin{itemize}
\item
  SCM begins with a symbolic alphabet Σ.
\item
  The rule set R emerges from symbolic return paths that preserve
  coherence.
\item
  The system assumes no space, time, or logic beyond symbolic
  transformation.
\item
  Identity is defined as a structure's ability to return to itself
  through permitted transitions.
\item
  If no such path exists, the structure is unresolved.
\item
  If one exists, the structure becomes a resolved identity: {[}S{]}.
\item
  There also exist inversion loops, such as Euler-type structures, which
  map {[}S{]} to {[}--S{]} and cancel coherence to produce the null
  identity {[}0{]}.
\item
  These do not define identity---but they define the boundary of
  identity through structural annihilation.
\end{itemize}

All future concepts --- coherence, stability, interaction ---emerge from
this single idea:\\
  \emph{A structure exists only if it can return.}

\section{Chapter 2 \textbar{} Resolution
Graphs}\label{chapter-2-resolution-graphs}

Having defined identity as the closure of permitted symbolic
transformation, we now develop the structural space in which such
closure occurs. That space is the resolution graph G(R), a directed
graph induced by the permitted transformations in a rule set R.\\
\strut \\
In this chapter, we show how resolved identities correspond to loops in
this graph. We then define support, reuse, and coupling---not as
probabilistic concepts, but as structural overlaps in transformation
paths. This chapter formalizes the topology of symbolic identity: which
forms resolve, which structures depend on others, and which loops
stabilize coherence across reuse.

\section{2.1 \textbar{} The Resolution Graph
G(R)}\label{the-resolution-graph-gr}

Let R be a finite rule set of permitted symbolic transformations:\\
  R = \{ Tᵢ: Sᵢ → Sᵢ′ \}\\
\strut \\
We define the resolution graph G(R) = (V, E) as follows:\\
- Nodes: V = \{S \textbar{} S is a symbolic structure over Σ\}\\
- Edges: E = \{ (S, S′) \textbar{} ∃ T ∈ R such that T(S) = S′ \}\\
\strut \\
Each edge (S, S′) in G(R) corresponds to a symbolic form T ∈ R ⊂ Σ* that
encodes the transformation from S to S′.. A path in G(R) is a finite
sequence of such transformations.

\section{2.2 \textbar{} Loops and Return
Closure}\label{loops-and-return-closure}

Let 𝓛(S) be the set of permitted return paths from S back to itself
under R.\\
A loop in G(R) is a directed cycle:\\
  S₀ → S₁ → \ldots{} → Sₖ, with S₀ = Sₖ\\
\strut \\
If such a loop exists, we say that S is return-closed, and define the
return depth:\\
  Xₕ(S) := min \{ k \textbar{} S → \ldots{} → S under R \}\\
\strut \\
This is the length of the shortest return loop.\\
Recall from Chapter 1: {[}S{]} is a resolved identity ⇔ 𝓛(S) ≠ ∅.

\section{2.3 \textbar{} Resolution Subgraphs and
Support}\label{resolution-subgraphs-and-support}

Let {[}S{]} be a return-closed identity (𝓛(S) ≠ ∅).\\
Define the resolution subgraph G\_{[}S{]} ⊆ G(R) as:\\
  G\_{[}S{]} := the subgraph induced by all nodes and edges appearing in
𝓛(S)\\
\strut \\
That is, G\_{[}S{]} contains all structures visited in valid return
loops resolving {[}S{]}.\\
We define the support set of {[}S{]} as:\\
  support({[}S{]}) := \{ S′ ∈ G\_{[}S{]} \textbar{} S′ ≠ S and appears
in some 𝓛(S) \}\\
\strut \\
Support represents the set of structures reused to complete return for
{[}S{]}.

\section{2.4 \textbar{} Reuse and Coupling}\label{reuse-and-coupling}

Let {[}A{]}, {[}B{]} be two distinct return-closed identities.\\
We say that {[}B{]} reuses {[}A{]} if:\\
  {[}A{]} ∈ support({[}B{]})\\
\strut \\
That is, the structure of {[}B{]} depends on {[}A{]} for return
resolution.\\
We define return coupling between {[}A{]} and {[}B{]} if:\\
  G\_{[}A{]} ∩ G\_{[}B{]} ≠ ∅\\
\strut \\
This indicates structural overlap---shared paths or reused
intermediates.\\
Consequences:

\begin{quote}
- Coherence failure in {[}A{]} may affect return of {[}B{]},\\
- Reinforcement of {[}A{]} may stabilize {[}B{]},\\
- Collapse or drift may propagate through G(R) via reuse coupling.
\end{quote}

Reuse and coupling define coherence dependencies between identities.

\section{2.5 \textbar{} Return Depth and Loop
Saturation}\label{return-depth-and-loop-saturation}

Each identity {[}S{]} has a return depth Xₕ({[}S{]}) := length of
minimal return loop.\\
\strut \\
We define:\\
- Shallow identity: Xₕ = 2 (minimal non-trivial loop),\\
- Deep identity: Xₕ ≫ 1, possibly involving composed or reused
structure.\\
\strut \\
We say that {[}S{]} is loop-saturated if:\\
  ∂Xₕ({[}S{]}) = 0 under RuleEvolution\\
\strut \\
That is, no further reduction in loop length is permitted.\\
Loop saturation implies that identity resolution has reached a minimal,
stable configuration.

\section{2.6 \textbar{} Summary}\label{summary-1}

- Every Rule Set R induces a resolution graph G(R),\\
- Identity {[}S{]} corresponds to a return loop in G(R),\\
- The subgraph G\_{[}S{]} contains all coherence-relevant structure for
{[}S{]},\\
- Support and reuse define structural dependency,\\
- Return coupling encodes coherence overlap,\\
- Loop saturation defines the minimal transformation needed to resolve
identity.\\
\strut \\
From this graph-theoretic foundation, we now define the quantitative
properties of coherence: collapse robustness, fragility, latency, and
structural stability.

\section{Chapter 3 \textbar{} Resolution Domains Ω₁, Ω₂,
Ω₃}\label{chapter-3-resolution-domains-ux3c9ux2081-ux3c9ux2082-ux3c9ux2083}

In this chapter, we introduce a structural classification of symbolic
forms based on return behavior. Given a symbolic structure S ∈ Σ* and a
permitted rule set R ⊂ Σ*, we define the resolution space as the
partitioning of Σ* into three coherence domains: Ω₁, Ω₂, and Ω₃. These
represent increasing degrees of return closure and identity resolution.

\section{3.1 \textbar{} Domain Definitions}\label{domain-definitions}

Let 𝓛(S) denote the set of permitted return paths from a structure S
back to itself, as defined in Chapter 2.

- Ω₁: the domain of unresolved structures.\\
  S ∈ Ω₁ ⇔ 𝓛(S) = ∅\\
  These structures admit no return path under the current rule set R.
They are unresolved, unstable, and cannot form identities.

- Ω₂: the domain of partial or unstable identities.\\
  S ∈ Ω₂ ⇔ 𝓛(S) ≠ ∅ but return is unstable or fragile.\\
  These structures return, but the path may be incoherent, high-drift,
or exhibit unbounded fragility (Xφ → ∞). Ω₂ captures structures on the
boundary of resolution, including unstable loops or transient identity
forms.

- Ω₃: the domain of resolved identities.\\
  S ∈ Ω₃ ⇔ 𝓛(S) ≠ ∅ and return is coherent, stable, and bounded.\\
  These structures satisfy all return conditions: finite loop length,
bounded fragility, minimal reuse. Ω₃ is the space of coherence-stable
identity and the foundation for symbolic dynamics in SCM.

\section{3.2 \textbar{} Structural
Interpretation}\label{structural-interpretation}

The Ω-partition provides a coarse symbolic topology over Σ*. A structure
transitions from Ω₁ → Ω₂ → Ω₃ as coherence stabilizes. The system does
not assume these domains axiomatically---they are emergent from the
behavior of return paths under a given rule set R.

Volume I focuses primarily on Ω₃, the space of coherent identities.
Later volumes may develop models for instability and partial identity in
Ω₂.

\section{3.3 \textbar{} Constructive Identity in
Ω₃}\label{constructive-identity-in-ux3c9ux2083}

We now show that the coherence-resolved identity space Ω₃ is non-empty
by constructing a simple identity from first principles.

\subsection{Minimal Alphabet and
Rules}\label{minimal-alphabet-and-rules}

Let the symbolic alphabet be Σ = \{a, b\}.\\
Define the following transformation rules:\\
 T₁: a → b\\
 T₂: b → a\\
We assume both T₁ and T₂ ∈ R, the enabled rule set.

\subsection{Constructed Identity}\label{constructed-identity}

Let S = a. Apply the sequence of transformations:\\
 a → b (via T₁)\\
 b → a (via T₂)\\
This forms a return loop:\\
 𝓛({[}S{]}) = \{ a → b → a \}\\
This loop is finite, coherent, and involves only enabled rules.

\subsection{Verification of Ω₃
Conditions}\label{verification-of-ux3c9ux2083-conditions}

We now verify that {[}a{]} satisfies the criteria for membership in Ω₃:

\begin{quote}
- Return path exists: 𝓛({[}a{]}) ≠ ∅\\
- Drift is zero: χ({[}a{]}) remains invariant under reuse\\
- Collapse robustness ρ({[}a{]}) is finite and nonzero
\end{quote}

Thus, {[}a{]} ∈ Ω₃.

\subsection{Excluded Structure: Inversion
Loop}\label{excluded-structure-inversion-loop}

Consider instead the Euler-style loop:\\
 T₃: a → --a, T₄: --a → a\\
These form an inversion pair that returns with maximal asymmetry (X\_π =
--1), resulting in cancellation:\\
 {[}a{]} + {[}--a{]} = {[}0{]}\\
Such cancellation loops do not qualify as identity. They resolve to the
null structure and are excluded from Ω₃.

This constructive example confirms that Ω₃ contains nontrivial,
coherence-resolved identities, and is therefore non-empty.

\section{3.4 \textbar{} Structural Annihilation and the Ω₋
Domain}\label{structural-annihilation-and-the-ux3c9-domain}

In addition to the previously defined identity resolution spaces Ω₁, Ω₂,
and Ω₃, the discovery of symmetry-inverted return loops introduces a
fourth domain of structural behavior: Ω₋.

Let {[}A{]} be a symbolic structure with a permitted transformation T₁
such that:

  T₁: {[}A{]} → {[}--A{]}

This one-step symmetry inversion is coherent, permitted, and of minimal
return depth (Xₕ = 1), but it does not form a return loop to {[}A{]} and
therefore does not constitute an identity.

Now consider the loop formed by a pair of transformations:

  T₁: {[}A{]} → {[}--A{]}, T₂: {[}--A{]} → {[}A{]}

This 2-step loop is coherent and symmetric under inversion. If {[}A{]}
and {[}--A{]} together resolve to {[}0{]} under structural addition, we
define this as a case of structural annihilation.

We define the annihilation domain:

  Ω₋ := \{ {[}A{]}, {[}--A{]} \textbar{} {[}A{]} + {[}--A{]} = {[}0{]}
\}

These structures meet coherence requirements: they have return paths and
bounded fragility, but they are not identities. They do not return to
themselves, but instead cancel to the null identity {[}0{]}.

This domain is structurally coherent but non-persistent. Ω₋ forms the
annihilation boundary of Ω₃---the symbolic equivalent of coherence
collapse via structural inversion.

Thus, the full symbolic topology of SCM identity resolution includes
four regions:

- Ω₁: Unresolved structures (no return path)

- Ω₂: Fragile or unstable coherence (∂χ unbounded or ρ \textless{} ρ\_c)

- Ω₃: Stable identity (return exists, bounded drift, and robustness)

- Ω₋: Coherence-resolved but annihilating (returns to {[}0{]}, not
{[}A{]})

Volume I now defines Ω₋ as the symbolic terminal class of coherence:
fully allowed under return, but excluded from identity due to complete
inversion.

\section{Chapter 4 \textbar{} Coherence
Metrics}\label{chapter-4-coherence-metrics}

Now that identity domains Ω₁, Ω₂, and Ω₃ have been introduced, we turn
our attention to quantitative measures that apply within Ω₃, the space
of coherence-stable identity. Not all return-closed structures are
equally robust. Some collapse under minimal symbolic deformation, while
others exhibit persistent coherence across reuse and drift. This chapter
defines four coherence metrics that distinguish between fragile and
stable identity behavior in Ω₃.

The following metrics define key axes of identity coherence:

\begin{quote}
- Collapse robustness (ρ): How far can an identity deform before it
exits Ω₃?\\
- Fragility (Xφ): The inverse of robustness; susceptibility to symbolic
collapse.\\
- Latency (Λ): Total symbolic effort required to complete a return
loop.\\
- Return symmetry (Xπ): Structural asymmetry in the permitted return
cycle.
\end{quote}

These quantities form part of the coherence signature χ({[}A{]}) and
help define coherence surfaces in χ-space.

\section{4.1 \textbar{} Collapse Robustness
ρ}\label{collapse-robustness-ux3c1}

Let {[}A{]} ∈ Ω₃ be a resolved identity. Let D({[}A{]}, {[}A′{]}) be a
permitted deformation path through G(R), composed of valid
transformations in R. Each transformation Tᵢ in the path contributes
symbolic effort effort(Tᵢ).\\
\strut \\
Define collapse robustness as:\\
  ρ({[}A{]}) := inf \{ ε \textgreater{} 0 \textbar{} ∃ {[}A′{]} within
D({[}A{]}, {[}A′{]}) such that 𝓛({[}A′{]}) = ∅ \}\\
\strut \\
This is the minimal symbolic effort required to push {[}A{]} into
Ω₁---i.e., to destroy return closure. {[}A{]} is collapse-robust if
ρ({[}A{]}) ≥ ρ\_c, for a coherence threshold ρ\_c \textgreater{} 0.

\section{4.2 \textbar{} Fragility Xφ}\label{fragility-xux3c6}

We define fragility as the inverse of collapse robustness:\\
  Xφ({[}A{]}) := 1 / ρ({[}A{]})\\
\strut \\
\textbf{Interpretation}\\
- High Xφ → identity collapses under minor deformation.\\
- Low Xφ → identity resists perturbation.\\
\strut \\
Xφ({[}A{]}) diverges as {[}A{]} approaches the Ω₃ → Ω₂ or Ω₁ boundary.
This is the first axis of χ({[}A{]}).

\section{4.3 \textbar{} Latency Λ and Symbolic
Effort}\label{latency-ux3bb-and-symbolic-effort}

Each transformation T ∈ R carries an assigned symbolic effort:

effort(T) ∈ ℝ⁺

We define the latency of a resolved identity {[}A{]} as the minimal
total effort required to complete a valid return loop:

Λ({[}A{]}) := min\_\{𝒫 ∈ 𝓛({[}A{]})\} ∑\_\{Tᵢ ∈ 𝒫\} effort(Tᵢ)

If effort is uniform (i.e., effort(T) = 1 for all T), then:

Λ({[}A{]}) = Xₕ({[}A{]})

\subsection{Effort Function}\label{effort-function}

Symbolic effort is defined structurally. Possible definitions include:

- effort(T) := length(T) --- the number of primitive symbols in T

- effort(T) := entropy(T) --- the number of distinct symbols in T

These definitions allow Λ({[}A{]}) to be computed purely from symbolic
structure.

\subsection{Inverse Transformation}\label{inverse-transformation}

Let T: S → S′ be a permitted transformation. We say that an inverse T⁻¹
exists if there is T′ ∈ Σ* such that:

T′(S′) = S

Then we write T′ = T⁻¹.

Note: not all transformations have inverses. Return symmetry X\_π is
computed using only known forward and reverse steps. When no inverse is
present in R, that segment is treated as directionally asymmetric.

\section{4.4 \textbar{} Return Symmetry
Xπ}\label{return-symmetry-xux3c0}

Return symmetry quantifies the structural asymmetry of a permitted
return loop. In SCM, this asymmetry plays a critical role in defining
coherence behavior, interaction potential, and structural charge.

\subsection{Return Symmetry}\label{return-symmetry}

Let {[}A{]} be a symbolic structure with a permitted return path
𝓛({[}A{]}).

Let:

\begin{quote}
- F = number of forward-directed transformations (unidirectional under
R)\\
- R = number of reverse-directed transformations (under R⁻¹)
\end{quote}

We define the return symmetry Xπ({[}A{]}) as:\\
Xπ({[}A{]}) := (F -- R) / (F + R)

This ratio measures the net directional bias of the return loop. It
ranges from --1 to +1.

\subsection{Interpretation of Xπ}\label{interpretation-of-xux3c0}

\begin{longtable}[]{@{}
  >{\raggedright\arraybackslash}p{(\linewidth - 4\tabcolsep) * \real{0.3333}}
  >{\raggedright\arraybackslash}p{(\linewidth - 4\tabcolsep) * \real{0.3333}}
  >{\raggedright\arraybackslash}p{(\linewidth - 4\tabcolsep) * \real{0.3333}}@{}}
\toprule\noalign{}
\begin{minipage}[b]{\linewidth}\raggedright
Xπ Value
\end{minipage} & \begin{minipage}[b]{\linewidth}\raggedright
Meaning
\end{minipage} & \begin{minipage}[b]{\linewidth}\raggedright
Interpretation
\end{minipage} \\
\midrule\noalign{}
\endhead
\bottomrule\noalign{}
\endlastfoot
--1 & Full inversion & Maximal asymmetry: S → --S (e.g., Euler pair) \\
0 & Symmetric & Perfectly balanced return \\
+1 & Forward-only return & Coherence source structure \\
\end{longtable}

A return loop with Xπ = --1 corresponds to structural inversion. If the
coherence pair resolves to {[}0{]}, this defines an Ω₋-type annihilation
path, not an identity

\subsection{Symmetry and Identity}\label{symmetry-and-identity}

Return symmetry does not determine whether identity exists. But it
classifies identities once return is confirmed:

\begin{quote}
- {[}A{]} is an identity ⇔ 𝓛({[}A{]}) ≠ ∅ and S₀ = Sₖ\\
- If the return is symmetric (Xπ = 0), {[}A{]} is structurally neutral\\
- If asymmetric (Xπ ≠ 0), {[}A{]} may act as a source (Xπ \textgreater{}
0) or sink (Xπ \textless{} 0)\\
- If Xπ = --1 and Sₖ ≠ S₀, return resolves not to identity, but to
{[}0{]}
\end{quote}

This last case is the Euler inversion: a coherence-valid path that
defines structural annihilation. Xπ is not just a measure of
asymmetry---it is a coherence fingerprint:

\begin{quote}
- It determines interaction directionality\\
- It distinguishes return from inversion, and identity from cancellation
\end{quote}

In χ-space, Xπ is a fully directional axis of coherence structure. Its
endpoints---{[}--1, +1{]}---are not boundaries of collapse, but poles of
inversion and emission.

\section{4.5 \textbar{} Summary}\label{summary-2}

These coherence metrics apply only to identities in Ω₃. Each plays a
role in defining the coherence structure of {[}A{]} and its behavior
under reuse, drift, or collapse.

\begin{longtable}[]{@{}
  >{\raggedright\arraybackslash}p{(\linewidth - 4\tabcolsep) * \real{0.3333}}
  >{\raggedright\arraybackslash}p{(\linewidth - 4\tabcolsep) * \real{0.3333}}
  >{\raggedright\arraybackslash}p{(\linewidth - 4\tabcolsep) * \real{0.3333}}@{}}
\toprule\noalign{}
\begin{minipage}[b]{\linewidth}\raggedright
Metric
\end{minipage} & \begin{minipage}[b]{\linewidth}\raggedright
Meaning
\end{minipage} & \begin{minipage}[b]{\linewidth}\raggedright
Symbol
\end{minipage} \\
\midrule\noalign{}
\endhead
\bottomrule\noalign{}
\endlastfoot
Collapse robustness & Minimum deformation before failure & ρ({[}A{]}) \\
Fragility & Inverse robustness & Xφ({[}A{]}) \\
Latency & Total symbolic effort to return & Λ({[}A{]}) \\
Return symmetry & Directional asymmetry in return loop & Xπ({[}A{]}) \\
\end{longtable}

\section{Chapter 5 \textbar{} χ-Space}\label{chapter-5-ux3c7-space}

Coherence metrics allow us to evaluate how stable an identity is under
reuse, deformation, and collapse. But to compare identities
structurally, we require a unified representation: a signature that
encodes all relevant coherence properties in a single form.\\
\strut \\
This chapter introduces the coherence signature χ({[}A{]}), a
five-component vector that structurally fingerprints any return-closed
identity. We then define the space of all such signatures---χ-space---as
a topological product space over symbolic metrics. Trajectories through
this space represent structural evolution. Stability, collapse, and
emergence are now modeled as topological transitions.

\section{5.1 \textbar{} Definition of the Coherence Signature
χ}\label{definition-of-the-coherence-signature-ux3c7}

Let {[}A{]} ∈ Ω₃ be a return-closed, collapse-robust identity.\\
We define its coherence signature χ({[}A{]}) as the 5-tuple:\\
\strut \\
  χ({[}A{]}) := (Xₕ, Xc, Xπ, Xφ, Xₑ)\\
\strut \\
Where:\\
- Xₕ: Return depth (Chapter 2)\\
- Xπ: Return symmetry (Chapter 4)\\
- Xφ: Fragility = 1 / ρ (Chapter 4)\\
- Xₑ: Reuse elevation (defined here)\\
- Xc: Coherence strength (defined next)\\
\strut \\
This signature completely characterizes the identity structure of
{[}A{]} with respect to reuse, return, and collapse.

\section{5.2 \textbar{} Return Depth Xₕ and Symmetry
Xπ}\label{return-depth-xux2095-and-symmetry-xux3c0}

Already defined in Chapters 2 and 4:\\
- Xₕ := length of shortest return loop 𝓛({[}A{]})\\
- Xπ := directional imbalance of return path (forward vs reverse
transitions)\\
These are structural invariants of G(R).

\section{5.3 \textbar{} Coherence Strength
Xc}\label{coherence-strength-xc}

Let 𝓡({[}A{]}) be the set of all permitted return loops for {[}A{]}.

Let χ₀ = χ(𝓛₀({[}A{]})) be the coherence signature of the minimal return
path.

For each alternate return path 𝓛ᵢ ∈ 𝓡({[}A{]}), let χᵢ := χ(𝓛ᵢ({[}A{]}))
be the signature derived from that path. We define the component-wise
deviation:

Δχᵢ := (1/5) ∑\_\{k=1\}\^{}5 \textbar(χᵢ\^{}\{(k)\} - χ₀\^{}\{(k)\}) /
χ₀\^{}\{(k)\}\textbar{}

Here, each component is compared independently. The result is a
normalized average deviation across all five coherence dimensions.

We then define coherence strength as:

Xc({[}A{]}) := (1 / \textbar 𝓡({[}A{]})\textbar) ∑ᵢ (1 - Δχᵢ)

\textbf{Interpretation}\\
- Xc ≈ 1: All return paths are consistent---identity is structurally
coherent regardless of path.\\
- Xc ≪ 1: Return paths diverge---identity is path-sensitive, fragile, or
degenerate.

This formally encodes coherence consistency without requiring a metric.
It respects the non-normed topology of χ-space and ensures each
dimension is weighted equally and independently.

\section{5.4 \textbar{} Reuse Elevation
Xₑ}\label{reuse-elevation-xux2091}

Let G{[}A{]} be the resolution subgraph of {[}A{]} (Chapter 2).\\
Let support({[}A{]}) be the set of all structures reused in resolving
{[}A{]}.\\
\strut \\
We define:\\
  Xₑ({[}A{]}) := \textbar support({[}A{]}) \textbackslash{}
𝓛({[}A{]})\textbar{}\\
\strut \\
That is, Xₑ is the number of symbolic structures required for return
that are not part of {[}A{]}'s own minimal loop.

Note: support({[}A{]}) and 𝓛({[}A{]}) are interpreted here as node sets
extracted from the return subgraph G\_{[}A{]}, enabling set subtraction.

\textbf{Interpretation}\\
- Xₑ = 0: {[}A{]} is internally return-closed (irreducible or
composed)\\
- Xₑ \textgreater{} 0: {[}A{]} depends on external reuse (elevated
identity)

\section{5.5 \textbar{} Fragility Xφ and Collapse
Threshold}\label{fragility-xux3c6-and-collapse-threshold}

From Chapter 4:

\begin{quote}
- Xφ({[}A{]}) := 1 / ρ({[}A{]})\\
- ρ({[}A{]}) is the minimum deformation effort that destroys
𝓛({[}A{]})\\
- Xφ encodes coherence brittleness
\end{quote}

This dimension defines the boundary of collapse.

\section{5.6 \textbar{} χ-Space: The Topology of
Identity}\label{ux3c7-space-the-topology-of-identity}

Let:\\
  X := ℕ × {[}0,1{]} × ℝ₊ × ℝ₊ × ℕ\\
\strut \\
Define χ-space as:\\
  χ-space := \{ χ({[}A{]}) \textbar{} {[}A{]} ∈ Ω₃ \}\\
\strut \\
This is a symbolic, non-metric product space. Each component of χ
governs a different mode of coherence behavior. There is no global norm.
Comparison must occur componentwise or via defined χ-neighborhoods.

\section{5.7 \textbar{} χ-Neighborhoods and
Clustering}\label{ux3c7-neighborhoods-and-clustering}

Let χ₀ ∈ X. Define a χ-neighborhood around χ₀:

𝒰(χ0\hspace{0pt},ε):=\{χ({[}A{]})∣∀i,~∣χ(i)−χ0(i)\hspace{0pt}∣\textless ε(i)\}\\
\strut \\
These neighborhoods allow:

\begin{quote}
- Local clustering,\\
- Detection of reuse similarity,\\
- Drift tolerance bounds.
\end{quote}

Identities that lie within 𝒰 of an anchor may be stabilized through
reuse.

\section{5.8 \textbar{} Trajectories and the Invariant
Surface}\label{trajectories-and-the-invariant-surface}

Let χₜ({[}A{]}) denote the coherence signature of {[}A{]} at
RuleEvolution time t.\\
Define drift:\\
  ∂χ({[}A{]}) := χₜ₊₁({[}A{]}) - χₜ({[}A{]})\\
\strut \\
Then {[}A{]} is:

\begin{quote}
- Saturated: ∂χ = 0\\
- Reactive: ∂χ bounded but nonzero\\
- Collapsing: ∂χ unbounded, or ρ({[}A{]}) \textless{} ρ\_c
\end{quote}

The invariant surface:\\
  𝓢 := \{ χ ∈ χ-space \textbar{} ∂χ = 0 \}\\
\strut \\
This surface defines structural stability. Anchors and conserved
identities lie on 𝓢.

Formally, Ω₃ is the set of all structures that satisfy:\\
  Ω₃ := \{ {[}A{]} \textbar{} 𝓛({[}A{]}) exists, ∂χ({[}A{]}) bounded,
and ρ({[}A{]}) ≥ ρ\_c \}\\
\strut \\
This is the space of stable, coherence-resolved identities. Only
identities in Ω₃ are considered valid elements of χ-space and subject to
RuleEvolution.

\section{5.9 \textbar{} Minimality of the Coherence
Signature}\label{minimality-of-the-coherence-signature}

The coherence signature χ({[}A{]}) := (Xₕ, Xc, Xπ, Xφ, Xₑ) defines the
structural identity of every resolved symbolic form in SCM. This section
addresses a foundational question:

Why are exactly five components used? Why not three, or seven?\\
\strut \\
We assert that this five-variable signature is minimal and sufficient
for resolving identity behavior in Ω₃. Each component was selected to
satisfy a distinct functional requirement:\\
\strut \\
\textbf{Xₕ (Return Depth)}\\
Captures structural loop complexity; required to distinguish shallow vs
deep identities.\\
\strut \\
\textbf{Xc (Coherence Strength)}\\
Measures return path consistency; required to differentiate stable vs
noisy identities.\\
\strut \\
\textbf{Xπ (Return Symmetry)}\\
Detects directional imbalance; required to separate symmetric (e.g.
reversible) from asymmetric identities.

\textbf{Xφ (Fragility)}\\
Encodes susceptibility to collapse; required to predict breakdown under
deformation.\\
\strut \\
\textbf{Xₑ (Reuse Elevation)}\\
Measures external support; required to identify elevated vs internally
stable structures.

We claim that no proper subset of these five is sufficient to classify
all resolved identities within Ω₃.\\
- Omitting Xₑ makes it impossible to distinguish elevated identities
from composed ones.\\
- Omitting Xc obscures whether coherence is consistent across return
paths.\\
- Omitting Xπ erases directional structure and breaks the ability to
detect symbolic charge.\\
\strut \\
These variables are functionally independent over χ-space and span the
complete classification range for:\\
- Identity type (irreducible, composed, elevated),\\
- Return behavior (saturated, reactive, collapsing),\\
- Reuse structure (self-contained, clustered, anchor-dependent).\\
\strut \\
A full formal proof of this minimality claim---including counterexamples
under χ′ ⊂ χ---is in appendix A.

\section{5.10 \textbar{} Injectivity and Classification
Limits}\label{injectivity-and-classification-limits}

While the coherence signature χ({[}A{]}) = (Xₕ, Xc, Xπ, Xφ, Xₑ) is
minimal and sufficient to classify identities in Ω₃, it is not
injective.

That is:\\
  ∃ {[}A{]} ≠ {[}B{]} ∈ Ω₃ such that χ({[}A{]}) = χ({[}B{]})

This occurs when distinct resolution graphs G({[}A{]}) and G({[}B{]})
share the same coherence metrics across all five dimensions. In such
cases, χ({[}A{]}) and χ({[}B{]}) lie at the same point in χ-space, even
though {[}A{]} and {[}B{]} are structurally distinct.

Conclusion:

- G(R) is injective: each identity {[}A{]} ∈ Ω₃ has a unique resolution
graph G({[}A{]})

- χ is not injective: multiple identities may share the same coherence
signature

Thus, χ-space defines a classification topology, not an identity space.
It groups structurally distinct forms by behavioral similarity under
reuse, drift, and return.

\section{5.11 \textbar{} Summary}\label{summary-3}

The coherence signature χ({[}A{]}) = (Xₕ, Xc, Xπ, Xφ, Xₑ) defines the
structural identity of every return-closed form. The space of all such
signatures---χ-space---is a symbolic topology over which identities
evolve, couple, drift, or collapse.\\
\strut \\
In the next chapter, we define RuleEvolution formally and classify
dynamic identity behavior according to trajectories in χ-space.

\section{Chapter 6 \textbar{}
RuleEvolution}\label{chapter-6-ruleevolution}

Until now, we have treated the rule set R as fixed. But identities in
SCM do not merely exist---they evolve. Under coherence pressure,
symbolic structures may drift, deform, collapse, or reorganize. A static
rule set is insufficient to model such behaviors.\\
\strut \\
This chapter introduces RuleEvolution---a second-order operator that
updates the permitted transformations R based on the coherence behavior
of resolved identities. We then classify identity behavior based on
their trajectories in χ-space, introducing stable, reactive, and
collapsing structures. These categories are not imposed---they arise
from the topological dynamics of symbolic return.

\section{6.1 \textbar{} Motivation: Identity Under Coherence
Strain}\label{motivation-identity-under-coherence-strain}

Let {[}A{]} ∈ Ω₃ be a resolved identity. Even if return path 𝓛({[}A{]})
exists, reuse may:

\begin{quote}
- Saturate: {[}A{]} appears in too many return loops.\\
- Drift: χ({[}A{]}) changes under transformation (∂χ ≠ 0).\\
- Collapse: ρ({[}A{]}) \textless{} ρ\_c under symbolic stress.
\end{quote}

We require a formal mechanism by which R can adapt: preserving
coherence, stabilizing reuse, or pruning collapse-prone forms. This is
RuleEvolution.

\section{6.2 \textbar{} Formal Definition of
RuleEvolution}\label{formal-definition-of-ruleevolution}

Let:

\begin{quote}
- Rₜ be the enabled Rule Set at evolution step n,\\
- G(Rₜ) be the resolution graph generated by Rₜ,\\
- Σₜ be the set of resolved identities at t,\\
- χₜ({[}A{]}) be the coherence signature of identity {[}A{]} at t,\\
- ∂χ({[}A{]}) := χₜ₊₁({[}A{]}) − χₜ({[}A{]}) be the signature drift of
{[}A{]},\\
- ρ({[}A{]}) be the collapse robustness of {[}A{]}.
\end{quote}

We define the RuleEvolution operator as:\\
  𝓡E: (G(Rₜ), χₜ) → updated subset Rₜ₊₁ ⊂ Σ*\\
\strut \\
We define the RuleEvolution operator as:

𝓡E: (G(Rₜ), χₜ) → Rₜ₊₁ ⊂ Σ*

RuleEvolution is not an algorithm that rewrites R---it is a structural
classifier. It re-enables or disables symbolic structures T ∈ Σ* based
on their participation in coherence-stabilizing return.

A transformation T becomes enabled (T ∈ R) if it:

\begin{quote}
- Appears in a return path 𝓛({[}A{]}) where ∂χ({[}A{]}) is bounded and
ρ({[}A{]}) ≥ ρ\_c,\\
- Does not produce cancellation (e.g., {[}S{]} + {[}--S{]} = {[}0{]})
rather than persistence.
\end{quote}

\subsection{Rule Update Conditions}\label{rule-update-conditions}

1. Coherence Drift (∂χ divergence):

- If ∂χ({[}A{]}) exceeds a symbolic threshold ε, coherence becomes
unstable.

2. Collapse Proximity (ρ \textless{} ρ\_c):

- If robustness falls below the collapse threshold, return closure is at
risk.

3. New Identity Resolution (𝓛-discovery):

- If a new valid return loop 𝓛({[}B{]}) emerges, R must be expanded to
stabilize {[}B{]}.

\subsection{RuleEvolution Scoring
Function}\label{ruleevolution-scoring-function}

Let T ∈ Σ*. Define the affected identity set:

Ω\_T := \{ {[}A{]} ∈ Ω₃ \textbar{} T ∈ 𝓛({[}A{]}) or T enables a new
𝓛({[}A{]}) \}

To compute the scoring function S(T), we define drift ∂χ({[}A{]}) using
a symbolic, componentwise average of absolute differences:

  ∂χ({[}A{]}) := (1/5) ∑\emph{\{i=1\}\^{}5
\textbar χ}\{t+1\}\^{}\{(i)\}({[}A{]}) −
χ\_t\^{}\{(i)\}({[}A{]})\textbar{}

This scalar drift measure respects the non-metric product topology of
χ-space. It avoids imposing a Euclidean structure while still enabling
RuleEvolution scoring. Please note that component-wise drift may be
optionally normalized or weighted for application-specific sensitivity.

Then define:

Δ∂χ\_total(T) := ∑\_\{{[}A{]} ∈ Ω\_T\} (∂χ\_old({[}A{]}) −
∂χ\_new({[}A{]}))

Δρ\_total(T) := ∑\_\{{[}A{]} ∈ Ω\_T\} (ρ\_new({[}A{]}) −
ρ\_old({[}A{]}))

Finally, compute the score:

S(T) := Δ∂χ\_total(T) − Δρ\_total(T)

Here:\\
- ∂χ\_old({[}A{]}) and ρ\_old({[}A{]}) are computed under rule set Rₜ\\
- ∂χ\_new({[}A{]}) and ρ\_new({[}A{]}) are computed under rule set Rₜ ∪
\{T\}

This ensures that S(T) is a real scalar value, and that RuleEvolution is
fully symbolic and mathematically well-defined.

\subsection{Formal expression}\label{formal-expression}

𝓡E(Rₜ) := (Rₜ ∪ R\_enable) \textbackslash{} R\_disable

Where:

\begin{quote}
- R\_enable = \{ T ∉ Rₜ \textbar{} S(T) \textgreater{} θₐ and T enables
new 𝓛({[}B{]}) \},\\
- R\_disable = \{ T ∈ Rₜ \textbar{} S(T) \textless{} −θ\_d or
destabilizes a saturated identity \}
\end{quote}

Thresholds θₐ and θ\_d control sensitivity.

\section{6.3 \textbar{} Drift and Structural
Classes}\label{drift-and-structural-classes}

Let ∂χ({[}A{]}) := χₜ₊₁({[}A{]}) - χₜ({[}A{]}) be the signature drift of
identity {[}A{]}.\\
\strut \\
We classify identity behavior into three structural types:

\begin{quote}
- Stable: ∂χ({[}A{]}) = 0; Structure invariant under reuse\\
- Reactive: ∂χ({[}A{]}) bounded, ∂χ ≠ 0; Structure deforms or fails\\
- Collapsing: ∂χ({[}A{]}) unbounded or ρ \textless{} ρ\_c
\end{quote}

These classes partition Ω₃ into regions of structural persistence,
adaptability, and fragility.

\section{6.4 \textbar{} RuleEvolution
Pressure}\label{ruleevolution-pressure}

Let {[}A{]} ∈ Ω₃ experience drift (∂χ ≠ 0) or reuse strain. Then {[}A{]}
exerts RuleEvolution pressure on R:

\begin{quote}
- If coherence is preserved, R may be reinforced (stabilizing
transformations),\\
- If coherence degrades, R may adapt (deleting or rewriting weak
transitions).
\end{quote}

This pressure arises from the internal state of the system---not from
external rules or inputs. SCM evolves structurally, not algorithmically.

\section{6.5 \textbar{} The Dynamic Identity
Landscape}\label{the-dynamic-identity-landscape}

Define the identity space Ω₃ as dynamically partitioned:

\begin{quote}
- Invariant Surface 𝓢 := \{ {[}A{]} \textbar{} ∂χ({[}A{]}) = 0 \}\\
- Reactive Basin 𝓑 := \{ {[}A{]} \textbar{} ∂χ bounded, ∂χ ≠ 0 \}\\
- Collapse Fringe ℱ := \{ {[}A{]} \textbar{} ∂χ diverges or ρ
\textless{} ρ\_c \}
\end{quote}

These regions define symbolic dynamics over χ-space. RuleEvolution acts
to stabilize Ω₃ by:

\begin{quote}
- Expanding 𝓢 (more stable forms),\\
- Containing 𝓑 (limiting drift),\\
- Pruning ℱ (removing collapse-prone paths).
\end{quote}

\section{6.6 \textbar{} Summary}\label{summary-4}

RuleEvolution is SCM's structural classifier. It evaluates symbolic
structures in Σ* and determines:

\begin{quote}
- Which transformations preserve identity stability (enable),\\
- Which degrade coherence and must be pruned (disable),\\
- Which enable new identity resolution (promote).
\end{quote}

It is deterministic, topology-driven, and fundamental to symbolic
adaptation.

Chapter 7 \textbar{} Triadic Principle

Not all identities are structurally equivalent. Some resolve return
independently. Others rely on internal reuse, and some require external
anchoring to remain coherent.\\
\strut \\
This chapter formalizes the Triadic Classification of identity in SCM.
Every coherence-resolved identity falls into one of three mutually
exclusive categories:\\
- Irreducible: return-closed without reuse,\\
- Composed: return-closed through internal reuse only,\\
- Elevated: return-closed only through external reuse.\\
\strut \\
These classes reflect structural constraints in χ-space and play a
central role in the emergence of particle structure and coherence
networks in later volumes.

\section{7.1 \textbar{} The Triadic
Principle}\label{the-triadic-principle}

Let {[}A{]} ∈ Ω₃. Then {[}A{]} falls into exactly one of the following
categories:\\
\strut \\
1. Irreducible:\\
  {[}A{]} resolves return without reusing any other structure.\\
  Formal conditions:\\
  - Xₑ = 0\\
  - support({[}A{]}) = ∅\\
  - 𝓛({[}A{]}) contains no substructure used elsewhere

Structures with minimal return but full inversion (e.g., Euler pairs
with Xₕ = 1 and Xπ = --1) do not qualify as irreducible---they cancel
rather than persist.\\
\strut \\
2. Composed:\\
  {[}A{]} resolves return through reuse of internal substructures.\\
  Formal conditions:\\
  - Xₑ = 0\\
  - support({[}A{]}) ≠ ∅\\
  - All reused structures lie within 𝓛({[}A{]})\\
\strut \\
3. Elevated:\\
  {[}A{]} resolves return only via support outside its minimal loop.\\
  Formal condition:\\
  - Xₑ \textgreater{} 0\\
\strut \\
We call this partition the Triadic Principle.

\section{7.2 \textbar{} Formal Proof of Exclusivity and
Exhaustiveness}\label{formal-proof-of-exclusivity-and-exhaustiveness}

Let {[}A{]} ∈ Ω₃. The three cases above are:\\
\strut \\
- Mutually exclusive:

\begin{quote}
- If support({[}A{]}) = ∅ → {[}A{]} cannot be composed or elevated.\\
- If Xₑ = 0 and support({[}A{]}) ≠ ∅ → {[}A{]} is composed.\\
- If Xₑ \textgreater{} 0 → support necessarily includes external
structures → {[}A{]} is elevated.
\end{quote}

- Collectively exhaustive:

\begin{quote}
- Either support({[}A{]}) is empty or non-empty.\\
- If empty → irreducible.\\
- If non-empty → either Xₑ = 0 (composed) or Xₑ \textgreater{} 0
(elevated).
\end{quote}

Thus, every identity in Ω₃ falls into exactly one of these three types.

\section{7.3 \textbar{} Examples and Structural
Consequences}\label{examples-and-structural-consequences}

- Irreducible:\\
 Example: The photon {[}γ{]} with 𝓛({[}γ{]}) = A → B → A, Xₑ = 0\\
 Interpretation: foundational identity, coherence baseline\\
\strut \\
- Composed:\\
 Example: {[}C{]} with internal loop A → B → C → A, where A and B are
reused but internal\\
 Interpretation: nested or modular structures\\
\strut \\
- Elevated:\\
 Example: {[}D{]} requires reuse of external anchor {[}E{]} for
coherence closure\\
 Interpretation: dependent identity, may be fragile or context-bound\\
\strut \\
These classes determine:\\
- Which identities can act as anchors,\\
- Which are reusable,\\
- Which are conditionally coherent or collapse-prone

\section{7.4 \textbar{} χ-Space Signatures by
Class}\label{ux3c7-space-signatures-by-class}

\begin{longtable}[]{@{}
  >{\raggedright\arraybackslash}p{(\linewidth - 6\tabcolsep) * \real{0.1900}}
  >{\raggedright\arraybackslash}p{(\linewidth - 6\tabcolsep) * \real{0.1452}}
  >{\raggedright\arraybackslash}p{(\linewidth - 6\tabcolsep) * \real{0.2099}}
  >{\raggedright\arraybackslash}p{(\linewidth - 6\tabcolsep) * \real{0.4549}}@{}}
\toprule\noalign{}
\begin{minipage}[b]{\linewidth}\raggedright
Class
\end{minipage} & \begin{minipage}[b]{\linewidth}\raggedright
Xₑ
\end{minipage} & \begin{minipage}[b]{\linewidth}\raggedright
Support({[}A{]})
\end{minipage} & \begin{minipage}[b]{\linewidth}\raggedright
χ-Space Behavior
\end{minipage} \\
\midrule\noalign{}
\endhead
\bottomrule\noalign{}
\endlastfoot
Irreducible & 0 & ∅ & Low-dimensional, stable \\
Composed & 0 & ≠ ∅ (internal) & Moderate Xₕ, low Xφ \\
Elevated & \textgreater0 & ≠ ∅ (external) & High Xₑ, possibly high Xφ \\
\end{longtable}

Elevated identities tend to cluster near the collapse boundary (Xφ ↑),
unless stabilized by anchors. Irreducible identities typically lie on or
near the invariant surface 𝓢 (∂χ = 0).

\section{7.5 \textbar{} Structural
Implications}\label{structural-implications}

The Triadic Principle forms the logical foundation for:

\begin{quote}
- Coherence inheritance\\
- Reuse-based emergence\\
- Structural modularity
\end{quote}

Reuse hierarchy:

\begin{quote}
- Irreducible identities → form the coherence base\\
- Composed identities → build internal reuse loops\\
- Elevated identities → float atop reuse scaffolds
\end{quote}

In later chapters, this structure enables:

\begin{quote}
- Definition of coherence anchors\\
- Stabilization of partial identities\\
- Formation of triplet reuse locks and symmetry-stabilized composites
\end{quote}

\section{7.6 \textbar{} Summary}\label{summary-5}

Every resolved identity in SCM is either:

\begin{quote}
- Irreducible: return-closed and reuse-free,\\
- Composed: internally reused and stable,\\
- Elevated: externally reuse-dependent.
\end{quote}

This classification is exhaustive and exclusive. It is grounded in
χ-space structure and encoded directly in the coherence signature.\\
\strut \\
In the next chapter, we examine what lies between identity and collapse:
partial identities, proto-anchors, and the fragile emergence of
coherence from dependency.

\section{Chapter 8 \textbar{} Partial
Identity}\label{chapter-8-partial-identity}

Not all symbolic structures in Ω₃ are fully independent. Some require
reuse to remain coherent. Others hover near collapse but persist due to
stabilization by surrounding anchors.\\
\strut \\
This chapter defines partial identity---a class of drift-prone,
reuse-dependent structures that resolve only within specific coherence
configurations. We also introduce proto-anchors, unstable patterns
reused by others, and the mechanism of drift suppression: how coherence
anchors stabilize fragile identities.\\
\strut \\
These concepts formalize the early phase of emergence---the symbolic
scaffolding that precedes particles, logic, or code.

\section{8.1 \textbar{} Definition of Partial
Identity}\label{definition-of-partial-identity}

Let {[}A{]} ∈ Ω₃ be a resolved identity. We say that {[}A{]} is a
partial identity if:

\begin{quote}
- 𝓛({[}A{]}) exists, but\\
- χ({[}A{]}) is unstable unless reuse anchors are present,\\
- Xₑ({[}A{]}) \textgreater{} 0, and\\
- ∂χ({[}A{]}) ≠ 0 under standalone reuse
\end{quote}

In short: {[}A{]} resolves return only when embedded within a coherence
cluster. It cannot stabilize in isolation.

\section{8.2 \textbar{} Reuse Dependence and Structural
Context}\label{reuse-dependence-and-structural-context}

Let {[}A{]} reuse a saturated anchor {[}B{]} such that:

\begin{quote}
- {[}B{]} lies on the invariant surface 𝓢 (∂χ({[}B{]}) = 0),\\
- {[}B{]} appears in G\_{[}A{]},\\
- Reuse stabilizes χ({[}A{]}) such that ∂χ({[}A{]}) → 0
\end{quote}

Then {[}A{]} is reuse-locked: coherence is inherited via coupling. This
is the first form of coherence inheritance in SCM. Return no longer
depends only on internal structure, but on coupling to a stable external
identity.

\section{8.3 \textbar{} Drift Suppression via Anchor
Coupling}\label{drift-suppression-via-anchor-coupling}

Let {[}A{]} be a partial identity. If:

\begin{quote}
- ∂χ({[}A{]}) ≠ 0 in isolation,\\
- but ∂χ({[}A{]}) → 0 under reuse of {[}B{]} with χ({[}B{]}) ∈ 𝓢,
\end{quote}

Then we define a drift suppression lock. This mechanism:

\begin{quote}
- Suppresses structural deformation in reuse-prone identities,\\
- Stabilizes near-collapse structures across coherence clusters,\\
- Creates symbolic analogues of binding, insulation, and symmetry
breaking
\end{quote}

Anchors must preserve identity under reuse. Inversion structures like
Euler pairs (Xₕ = 1, X\_π = --1) resolve to {[}0{]} and cannot function
as coherence anchors.

\section{8.4 \textbar{} Proto-Anchors}\label{proto-anchors}

Let {[}C{]} be a symbolic structure such that:

\begin{quote}
- 𝓛({[}C{]}) = ∅ (not yet resolved),\\
- {[}C{]} appears in G\_{[}A{]} for multiple {[}A{]} ∈ Ω₃,\\
- {[}C{]} is reused but not yet stable
\end{quote}

We call {[}C{]} a proto-anchor.

If over steps:

\begin{quote}
- {[}C{]} appears in more coherence paths,\\
- RuleEvolution reinforces closure paths to/from {[}C{]},\\
- ∂χ({[}C{]}) → 0 and ρ({[}C{]}) ≥ ρ\_c,
\end{quote}

Then {[}C{]} becomes a full coherence anchor. This transition is the
symbolic equivalent of structural emergence.

\section{8.5 \textbar{} Resolution Spectrum: Identity vs
Collapse}\label{resolution-spectrum-identity-vs-collapse}

We now classify all symbolic forms by their return and reuse conditions:

\begin{longtable}[]{@{}
  >{\raggedright\arraybackslash}p{(\linewidth - 8\tabcolsep) * \real{0.2546}}
  >{\raggedright\arraybackslash}p{(\linewidth - 8\tabcolsep) * \real{0.2260}}
  >{\raggedright\arraybackslash}p{(\linewidth - 8\tabcolsep) * \real{0.1776}}
  >{\raggedright\arraybackslash}p{(\linewidth - 8\tabcolsep) * \real{0.0808}}
  >{\raggedright\arraybackslash}p{(\linewidth - 8\tabcolsep) * \real{0.2612}}@{}}
\toprule\noalign{}
\begin{minipage}[b]{\linewidth}\raggedright
Form
\end{minipage} & \begin{minipage}[b]{\linewidth}\raggedright
Return Loop (𝓛)
\end{minipage} & \begin{minipage}[b]{\linewidth}\raggedright
\textbar{} ∂χ Behavior
\end{minipage} & \begin{minipage}[b]{\linewidth}\raggedright
Xₑ
\end{minipage} & \begin{minipage}[b]{\linewidth}\raggedright
Status
\end{minipage} \\
\midrule\noalign{}
\endhead
\bottomrule\noalign{}
\endlastfoot
Fully Resolved & Exists & ∂χ = 0 & ≥ 0 & Stable identity \\
Partial Identity & Exists & ∂χ ≠ 0 alone & \textgreater{} 0 &
Contextually stable \\
Proto-Anchor & Not yet exists & ∂χ undefined & --- & Reuse
scaffolding \\
Collapsed Structure & Fails & ∂χ diverges & --- & Incoherent \\
\end{longtable}

This spectrum explains why some structures persist only in certain reuse
environments---and why most symbolic forms in Σ* never become
identities.

\section{8.6 \textbar{} Structural Role of Partial
Identity}\label{structural-role-of-partial-identity}

Partial identities:

\begin{quote}
- Populate the boundary between incoherence and structure,\\
- Enable compositional reuse without structural stability,\\
- Bridge the gap between randomness and persistent identity
\end{quote}

They are not errors---they are emergence pathways.\\
\strut \\
Most elevated identities begin as partial identities. Most anchors pass
through a proto-anchor phase.\\
\strut \\
This chapter closes the foundational narrative of structural emergence.
In the next chapter, we show how triplet reuse locks and coherence
convergence produce the first fully symmetry-stabilized families of
identity.

\section{Chapter 9 \textbar{} Coherence
Triplets}\label{chapter-9-coherence-triplets}

Once partial identities stabilize, and reuse structures lock under
coherence, a new class of configurations appears: triplet-convergent
identities. These are not isolated; they cohere through mutual reuse of
a shared structural anchor. When the reuse configuration is symmetric
and stabilized under RuleEvolution, we call the result a coherence
triplet.\\
\strut \\
This chapter formalizes how such structures form, what conditions enable
them, and why they mark the symbolic threshold between generic identity
and structure.

\section{9.1 \textbar{} χ-Convergent Identity
Families}\label{ux3c7-convergent-identity-families}

Let \{{[}A₁{]}, {[}A₂{]}, {[}A₃{]}\} be three resolved identities in
Ω₃.\\
We say they are a χ-convergent family if:

\begin{quote}
- There exists a structure {[}T{]} such that:\\
- Each {[}Aᵢ{]} reuses {[}T{]} (T ∈ support({[}Aᵢ{]})),\\
- χ({[}Aᵢ{]}) lie within a shared χ-neighborhood,\\
- The coherence signature across the set is symmetric:\\
  ∑ χ({[}Aᵢ{]}) = 3χ({[}T{]})
\end{quote}

This structure {[}T{]} serves as a reuse anchor for the triplet.\\
If {[}T{]} is saturated (∂χ({[}T{]}) = 0), the configuration forms a
stable reuse-locked system.

\section{9.2 \textbar{} Minimal Reuse Lock and Return
Stabilization}\label{minimal-reuse-lock-and-return-stabilization}

Define {[}T{]} as a minimal reuse-locked anchor if:

\begin{quote}
- {[}T{]} is not independently return-closed (𝓛({[}T{]}) = ∅ or
unstable),\\
- {[}T{]} appears in 𝓛({[}Aᵢ{]}) for exactly three distinct {[}Aᵢ{]},\\
- The composite structure \{{[}Aᵢ{]}, {[}T{]}\} resolves return closure
when combined.
\end{quote}

This is the symbolic skeleton of reuse entanglement.\\
Return is no longer isolated---it is structurally distributed.

\section{9.3 \textbar{} Symmetry-Stabilized
Triplets}\label{symmetry-stabilized-triplets}

Let a triplet \{{[}A₁{]}, {[}A₂{]}, {[}A₃{]}\} satisfy:

\begin{quote}
- χ({[}Aᵢ{]}) have symmetric latency: Λ({[}Aᵢ{]}) = Λ₀ ± ε,\\
- Return symmetry: ∑ Xπ({[}Aᵢ{]}) = 0,\\
- External reuse load is minimal: Xₑ({[}T{]}) = 0,\\
- This symmetry balancing means one identity may act as a coherence
source (Xπ \textgreater{} 0), one as a sink (Xπ \textless{} 0), and one
as a mediator (Xπ ≈ 0)\\
- Anchor {[}T{]} must have ∂χ = 0 and X\_π ≈ 0
\end{quote}

Then the system is symmetry-stabilized. It forms a coherence triplet
lock, the smallest resolved unit of symmetry-based identity.

Inversion pairs ({[}S{]}, {[}--S{]}) cannot form triplets. They resolve
to {[}0{]} via structural annihilation and do not stabilize reuse

\section{9.4 \textbar{} Identity Clustering}\label{identity-clustering}

These triplet structures exhibit:

\begin{quote}
- Mutual drift suppression,\\
- Shared reuse pathways (G\_{[}Aᵢ{]} ∩ G\_{[}T{]} ≠ ∅),\\
- Invariant coherence under RuleEvolution,\\
- Return cycle invariance (loop-saturated, Λ fixed)
\end{quote}

Such configurations are reuse-resolved units. In χ-space, they appear as
compact clusters orbiting a coherence anchor.

\section{9.5 \textbar{} Summary and Threshold of Physical
Identity}\label{summary-and-threshold-of-physical-identity}

Coherence triplets mark a turning point in the SCM hierarchy.\\
They are:

\begin{quote}
- Return-closed,\\
- Drift-suppressed,\\
- Symmetry-locked,\\
- Coherence-anchored
\end{quote}

These are no longer arbitrary structures. They are symbolically stable
configurations.\\
\strut \\
This closes Volume I.
