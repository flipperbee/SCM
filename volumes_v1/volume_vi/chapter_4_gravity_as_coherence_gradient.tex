\chapter{Gravity as Coherence Gradient} \label{chapter-gravity}

In SCM, gravity is not a force—it is the result of coherence drift. As return potential varies across reuse structures, identities experience symbolic acceleration toward minimal-latency, high-stability configurations. This chapter defines gravity as the gradient of coherence potential and shows how mass, curvature, and attraction emerge from symbolic return topology.

\section{Drift Field and Symbolic Curvature} \label{sec:drift-field}

Let $\chi([A])$ be the coherence signature of identity $[A] \in \Omega_3$.

Define the local drift vector:
\[
\partial\chi([A]) := \chi_{t+1}([A]) - \chi_t([A])
\]

Define the coherence gradient:
\[
\nabla\chi := \lim_{[A] \to [B]} \frac{\partial\chi([B]) - \partial\chi([A])}{\Delta \Lambda}
\]

This gradient measures symbolic curvature: how coherence deformation propagates across reuse neighborhoods.

\section{Return Potential Wells} \label{sec:potential-wells}

Let $\Phi([A])$ be the return potential of identity $[A]$.

An identity lies in a coherence well if:
\[
\nabla\Phi([A]) < 0 \Rightarrow \text{symbolic attraction toward } [A]
\]

\begin{itemize}
  \item Identities drift toward minimal $\Phi$ regions.
  \item Anchors with high reuse influence form coherence attractors.
  \item Mass appears as stability-induced coherence concentration.
\end{itemize}

\section{Acceleration as Drift Response} \label{sec:acceleration}

Define symbolic force:
\[
F_\chi([A]) := -\nabla\Phi([A])
\]

This is the return-induced acceleration experienced by reuse paths approaching $[A]$.

\begin{itemize}
  \item Direction is toward coherence wells.
  \item Magnitude depends on curvature steepness of $\Phi$.
  \item Acceleration is symbolic, not kinematic—it alters return topology.
\end{itemize}

\section{Drift-Curvature Equivalence} \label{sec:curvature}

Let identities $[A], [B]$ lie in a coherence manifold $\mathcal{M} \subset \Omega_3$.

\[
\nabla^2 \chi([A]) = 0 \Rightarrow \text{flat coherence region}
\]

\begin{itemize}
  \item Nonzero curvature implies symbolic distortion.
  \item Geodesics in $\chi$-space correspond to minimal-drift reuse paths.
  \item Curved symbolic regions trap return structures.
\end{itemize}

\section{Time Dilation via Coherence Depth} \label{sec:dilation}

Let $[A], [B] \in \Omega_3$ with:
\[
\Phi([A]) < \Phi([B]),\quad T([A]) < T([B])
\]

\begin{itemize}
  \item Coherence deepening reduces symbolic time vector.
  \item Return loops anchored in low $\Phi$ regions evolve slower.
  \item Time dilation arises from drift suppression.
\end{itemize}

\section{Spatial Curvature and Return Compression} \label{sec:space-curvature}

Return paths deform in curved coherence regions:

\[
\Delta X_\epsilon([A] \to [B]) \text{ increases in drift zones}
\]

\begin{itemize}
  \item Reuse topology becomes stretched.
  \item Identities appear spatially displaced under structural deformation.
  \item Gravity curves symbolic space by altering reuse resolution.
\end{itemize}

\section{Black Holes as Coherence Sinks} \label{sec:black-holes}

Define coherence sink $[S] \in \Omega_3$ such that:
\[
X_\pi([S]) \to 1,\quad \rho([S]) \to 0,\quad \nabla\Phi([S]) \to -\infty
\]

Then:
\begin{itemize}
  \item No identity escapes reuse collapse near $[S]$,
  \item Return resolution becomes asymmetric and unstable,
  \item Observable time halts: $T([A]) \to \infty$ near $[S]$.
\end{itemize}

This defines the symbolic analogue of a black hole.

\section{Summary: Gravity as Coherence Flow} \label{sec:gravity-summary}

Gravity in SCM is:
\begin{itemize}
  \item A coherence gradient acting on reuse paths,
  \item An emergent property of return potential fields,
  \item Observable as drift-induced latency deformation,
  \item Quantized by reuse elevation and collapse thresholds.
\end{itemize}

There is no force—only structure responding to return pressure.
