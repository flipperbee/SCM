\chapter{Return Invariance as Emergent Gauge Symmetry} \label{chapter-return-invariance}

In classical gauge theory, physical observables are invariant under local transformations of the internal gauge space. In SCM, an analogous invariance emerges from the equivalence of all coherence-preserving return paths under RuleEvolution. This chapter demonstrates that gauge symmetry is not a primitive assumption, but a necessary consequence of symbolic closure under return.

\section{Return Path Equivalence in SCM}

\begin{definition}[Return Loop Equivalence]
Let \( \mathcal{L}_1([A]), \mathcal{L}_2([A]) \in \mathcal{R}([A]) \) be two return paths resolving identity \( [A] \in \Omega_3 \). We write:
\[
\mathcal{L}_1([A]) \sim \mathcal{L}_2([A])
\quad \text{if} \quad
\Lambda(\mathcal{L}_1) = \Lambda(\mathcal{L}_2) \text{ and } \chi(\mathcal{L}_1) = \chi(\mathcal{L}_2)
\]
\end{definition}

This defines symbolic invariance: different sequences of transformations result in the same identity structure.

\paragraph{Interpretation.}
This equivalence defines an internal redundancy in how identities are resolved. It plays the structural role of a gauge transformation.

\section{Gauge Invariance in Field Theory}

\begin{definition}[Gauge Transformation]
In a Yang--Mills theory with gauge group \( G \), a gauge transformation is a smooth map:
\[
g: \mathbb{R}^4 \to G
\quad \text{acting as} \quad
A_\mu \mapsto A_\mu^g := g A_\mu g^{-1} - (\partial_\mu g) g^{-1}
\]
\end{definition}

\paragraph{Physical Invariance.}
Observable quantities like \( F_{\mu\nu} \) are invariant under \( g \). Only the structure of field curvature matters—not the particular connection used.

\section{Structural Mapping}

We now formalize the structural equivalence between SCM return invariance and classical gauge symmetry.

\begin{definition}[Return Group \( \mathcal{G}([A]) \)]
For an identity \( [A] \in \Omega_3 \), define:
\[
\mathcal{G}([A]) := \{ \mathcal{L}_i([A]) \in \mathcal{R}([A]) \mid \chi(\mathcal{L}_i) = \chi([A]) \}
\]
\end{definition}

This set forms a symbolic transformation group under composition (modulo RuleEvolution closure), preserving coherence signature.

\begin{proposition}[Return Group Invariance]
Let \( [A] \in \Omega_3 \). Then:
\[
\forall \mathcal{L}_i \in \mathcal{G}([A]), \quad \Phi([A]) = \Phi(\mathcal{L}_i)
\]
\end{proposition}

\paragraph{Analogy.}
This invariance under coherent return variation plays the same role as gauge invariance: transformations in an internal space that leave physical structure unchanged.

\section{Local vs. Global Invariance}

In classical gauge theory:
- \emph{Global gauge transformations} are constant across spacetime.
- \emph{Local gauge transformations} vary pointwise: \( g(x) \).

In SCM:
- \emph{Global return invariance} applies to identities reused identically across contexts.
- \emph{Local return invariance} arises when the reuse context varies: return paths differ symbolically but yield equivalent coherence closure.

\begin{definition}[Contextual Return Freedom]
Let \( [A] \in \Omega_3 \) and \( [B_1], [B_2] \) be reuse identities. If:
\[
[A] \to \mathcal{L}_1 \to [B_1] \quad \text{and} \quad [A] \to \mathcal{L}_2 \to [B_2]
\]
with \( \chi(\mathcal{L}_1) = \chi(\mathcal{L}_2) \), then symbolic gauge freedom is preserved under local variation.

\end{definition}

\section{Return Constraints and Gauge Fixing}

Return paths that fail coherence closure induce symbolic drift and contradiction:

\begin{itemize}
  \item Non-closure: \( \chi(\mathcal{L}_i) \ne \chi([A]) \) ⇒ instability.
  \item Overconstraint: Unresolvable symbolic cycles ⇒ collapse or pruning by \( \mathcal{R}E \).
\end{itemize}

\paragraph{Gauge Fixing Analogy.}
In Yang--Mills theory, arbitrary freedom in \( A_\mu \) is reduced via gauge fixing (e.g., Lorenz gauge). In SCM, RuleEvolution plays the same role: it prunes incoherent return freedom, preserving only viable transformations.

\section{Summary}

Symbolic return invariance in SCM serves as the structural origin of gauge symmetry:

\begin{itemize}
  \item All coherence-preserving return paths are structurally equivalent.
  \item Observable structure depends only on coherence signature, not path history.
  \item Local return variation induces symbolic curvature, equivalent to gauge field strength.
  \item RuleEvolution enforces consistency, acting as a symbolic gauge constraint.
\end{itemize}

This provides a rigorous foundation for interpreting gauge invariance as an emergent, structural symmetry in coherence logic.
