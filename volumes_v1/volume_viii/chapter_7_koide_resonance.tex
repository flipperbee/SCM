\chapter{The Koide Resonance from Phase Balance} \label{chapter:koide}

The ultimate test of a theoretical framework is its ability to predict precise, nontrivial physical quantities. In this chapter, we demonstrate how the Eulerian structure of SCM leads directly to the observed Koide mass ratio—a mysterious empirical relationship between the three charged leptons (electron, muon, tau).

The resonance emerges not from a curve fit or empirical ansatz, but from phase-locked coherence symmetry. It is the first numerical prediction derived from SCM's core variational structure.

\section{The Koide Ratio}

Let $m_1, m_2, m_3$ be the masses of the three lepton identities. The Koide ratio is defined as:
\[
Q := \frac{m_1 + m_2 + m_3}{\left( \sqrt{m_1} + \sqrt{m_2} + \sqrt{m_3} \right)^2}
\]

Empirically, for $(m_e, m_\mu, m_\tau)$, this evaluates to:
\[
Q \approx \frac{2}{3}
\]

No standard model mechanism explains this ratio. In SCM, it arises from **Eulerian phase cancellation** among a coherence-locked triplet.

---

\section{Phase Symmetry in Eulerian Triplets}

Let each identity $[A_i]$ in a coherence triplet be associated with a symbolic mass (latency) and a corresponding Eulerian phase angle $\theta_i$.

\[
\sqrt{m_i} \sim a + b \cos(\theta_i)
\]

The Koide condition corresponds to a configuration where the three identities form a **closed triangle on the unit circle**:
\[
\sum_{i=1}^{3} e^{i\theta_i} = 0
\]

This implies:
- Balanced return asymmetry: $\sum X_\pi([A_i]) = 0$
- Equal latency offsets in the coherence field
- Minimal contradiction functional: $\mathcal{C} = \sum X_\pi([A_i])^2$ minimized

---

\section{Origin of the 2/3 Ratio}

Under these conditions, we show in Appendix X that:
- The contradiction functional $\mathcal{C}$ is directly related to the Koide ratio $Q$
- The minimal nontrivial value of $\mathcal{C}$ that avoids degeneracy (identical masses) corresponds to $Q = 2/3$

This value is not chosen—it is derived from:

1. The SCM variational law:  
   \[
   \text{Select } R^* = \arg\max(F - \lambda \cdot \mathcal{C})
   \]

2. The phase constraint:  
   \[
   \sum e^{i\theta_i} = 0 \Rightarrow \theta_i = \theta_0 + \frac{2\pi}{3} i
   \]

3. The mass-phase relation:  
   \[
   \sqrt{m_i} = a + b \cos(\theta_i)
   \]

Solving these together leads uniquely to the Koide ratio.

---

\section{Structural Interpretation}

In SCM, the Koide resonance is the result of **structural coherence optimization**:
- The triplet minimizes contradiction while avoiding degeneracy,
- Phase rotation symmetry forces quantized spacing,
- Return asymmetry is distributed in a cancellation-locked configuration.

The resulting triplet is not “fine-tuned”—it is **variationally optimal**.

---

\section{Comparison to Standard Physics}

| Feature                 | Standard Model            | SCM Interpretation                                |
|------------------------|---------------------------|---------------------------------------------------|
| Lepton masses          | Arbitrary (fit parameters) | Derived from phase-locked reuse triplet           |
| Koide ratio (2/3)      | Unexplained coincidence    | Result of contradiction minimization              |
| Charge spacing         | Input symmetry             | Output of return coherence stability              |
| Mass prediction        | Empirical                 | Symbolic-structural derivation                    |

---

\section{Conclusion: Coherence Resonance as Prediction}

The Koide triplet demonstrates that SCM is not just explanatory—it is **predictive**. The resonance emerges from the same structural law that governs symbolic identity, return, and coherence. It is a signature of a **minimum-contradiction, maximum-coherence solution** on the Euler circle.

For the full derivation, see Appendix X.

In the final chapter, we now reframe the axioms of quantum theory in light of SCM—showing that the Copenhagen interpretation is not fundamental, but an emergent approximation.
