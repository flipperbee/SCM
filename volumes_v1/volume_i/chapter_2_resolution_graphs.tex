\chapter{Resolution Graphs}

Having defined identity as the closure of permitted symbolic
transformation, we now develop the structural space in which such
closure occurs. That space is the resolution graph G(R), a directed
graph induced by the permitted transformations in a rule set R.

In this chapter, we show how resolved identities correspond to loops in
this graph. We then define support, reuse, and coupling---not as
probabilistic concepts, but as structural overlaps in transformation
paths. This chapter formalizes the topology of symbolic identity: which
forms resolve, which structures depend on others, and which loops
stabilize coherence across reuse.

\section{The Resolution Graph $G(R)$}

Let $R$ be the rule set of permitted symbolic transformations, as defined previously in Equation~\ref{eq:rule-set}.
\begin{definition}[Resolution Graph] \label{def:resolution-graph}
We define the resolution graph $G(R) = (V, E)$ as follows:
\begin{itemize}
  \item \textbf{Nodes:} $V = \{ S \mid S \text{ is a symbolic structure over } \Sigma \}$
  \item \textbf{Edges:} $E = \{ (S, S') \mid \exists\, T \in R \text{ such that } T(S) = S' \}$
\end{itemize}
\end{definition}

Each edge $(S, S')$ in $G(R)$ corresponds to a symbolic form $T \in R \subset \Sigma^*$ that encodes the transformation from $S$ to $S'$.  
A path in $G(R)$ is a finite sequence of such transformations.

\section{Loops and Return Closure}

Let $\mathcal{L}(S)$ be the set of permitted return paths from $S$ back to itself under $R$.

A loop in $G(R)$ is a directed cycle of the form:
\[
S_0 \rightarrow S_1 \rightarrow \cdots \rightarrow S_k, \quad \text{with } S_0 = S_k
\]

\begin{definition}[Return Closure] \label{def:return-closure}
A structure $S$ is said to be \textit{return-closed} if there exists at least one loop in $G(R)$ that begins and ends at $S$, i.e., $\mathcal{L}(S) \neq \emptyset$.
\end{definition}

We define the \textit{return depth} of $S$ as:
\begin{equation} \label{eq:return-depth}
X_h(S) := \min \{ k \mid S \rightarrow \cdots \rightarrow S \text{ under } R \}
\end{equation}

This is the length of the shortest return loop for $S$.
Recall from Chapter~1: a structure is a resolved identity, $[S]$, if and only if $\mathcal{L}(S) \neq \emptyset$.

\section{Resolution Subgraphs and Support}

Let $[S]$ be a return-closed identity, i.e., $\mathcal{L}(S) \neq \emptyset$.

\begin{definition}[Resolution Subgraph] \label{def:resolution-subgraph}
The \textit{resolution subgraph} $G_{[S]} \subseteq G(R)$ is the subgraph induced by all nodes and edges appearing in valid return paths resolving $[S]$. That is:
\begin{equation} \label{eq:resolution-subgraph}
G_{[S]} := \text{subgraph of } G(R) \text{ induced by } \bigcup \mathcal{L}(S)
\end{equation}
\end{definition}

This subgraph contains all structures visited in valid return loops that resolve $[S]$.

\begin{definition}[Support Set] \label{def:support-set}
The \textit{support set} of $[S]$ is defined as:
\begin{equation} \label{eq:support-set}
\text{support}([S]) := \{ S' \in G_{[S]} \mid S' \neq S \text{ and } S' \text{ appears in some } \mathcal{L}(S) \}
\end{equation}
\end{definition}

The support set represents the set of structures \textit{reused} to complete return for $[S]$.

\section{Reuse and Coupling}

Let $[A]$, $[B]$ be two distinct return-closed identities.

\begin{definition}[Reuse] \label{def:reuse}
We say that $[B]$ \textit{reuses} $[A]$ if:
\begin{equation} \label{eq:reuse-condition}
[A] \in \text{support}([B])
\end{equation}
That is, the structure of $[B]$ depends on $[A]$ for return resolution.
\end{definition}

\begin{definition}[Return Coupling] \label{def:coupling}
We define \textit{return coupling} between $[A]$ and $[B]$ if their resolution subgraphs overlap:
\begin{equation} \label{eq:coupling-condition}
G_{[A]} \cap G_{[B]} \neq \emptyset
\end{equation}
This indicates structural overlap — shared return paths or reused intermediates.
\end{definition}

\textbf{Consequences of reuse coupling:}
\begin{itemize}
  \item Coherence failure in $[A]$ may affect return of $[B]$
  \item Reinforcement of $[A]$ may stabilize $[B]$
  \item Collapse or drift may propagate through $G(R)$ via reuse coupling
\end{itemize}

Reuse and coupling define coherence dependencies between identities.

\section{Return Depth and Loop Saturation}

Each identity $[S]$ has a return depth defined as:
\begin{equation} \label{eq:return-depth}
X_h([S]) := \text{length of the minimal return loop}
\end{equation}

We classify identity depth as follows:
\begin{itemize}
  \item \textbf{Shallow identity:} $X_h = 2$ (minimal non-trivial loop)
  \item \textbf{Deep identity:} $X_h \gg 1$ (possibly involving composed or reused structure)
\end{itemize}

\begin{definition}[Loop Saturation] \label{def:loop-saturation}
An identity $[S]$ is said to be \textit{loop-saturated} if the return depth no longer decreases under RuleEvolution:
\begin{equation} \label{eq:loop-saturation}
\partial X_h([S]) = 0
\end{equation}
\end{definition}

That is, no further reduction in loop length is permitted.  
Loop saturation implies that identity resolution has reached a minimal, stable configuration.

\section{Summary}

\begin{itemize}
  \item Every rule set $R$ induces a resolution graph $G(R)$
  \item An identity $[S]$ corresponds to a return loop in $G(R)$
  \item The subgraph $G_{[S]}$ contains all coherence-relevant structure for $[S]$
  \item Support and reuse define structural dependency
  \item Return coupling encodes coherence overlap
  \item Loop saturation defines the minimal transformation needed to resolve identity
\end{itemize}

From this graph-theoretic foundation, we now define the quantitative properties of coherence: collapse robustness, fragility, latency, and structural stability.
