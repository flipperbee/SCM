\chapter{RuleEvolution} \label{chapter_rule-evolution}

Until now, we have treated the rule set $R$ as fixed.  
But identities in SCM do not merely exist — they evolve.  
Under coherence pressure, symbolic structures may drift, deform, collapse, or reorganize.  
A static rule set is insufficient to model such behaviors.

This chapter introduces \textit{RuleEvolution} — a second-order operator that updates the permitted transformations $R$  
based on the coherence behavior of resolved identities.

We then classify identity behavior based on their trajectories in $\chi$-space,  
introducing stable, reactive, and collapsing structures.  
These categories are not imposed — they arise naturally from the topological dynamics of symbolic return.

\section{Motivation: Identity Under Coherence Strain}

Let $[A] \in \Omega_3$ be a resolved identity.  
Even if a valid return path $\mathcal{L}([A])$ exists, repeated reuse may introduce instability:

\begin{itemize}
  \item \textbf{Saturation:} $[A]$ appears in too many return loops.
  \item \textbf{Drift:} The coherence signature $\chi([A])$ changes over time ($\partial \chi \ne 0$).
  \item \textbf{Collapse:} The robustness $\rho([A])$ drops below the critical threshold $\rho_c$ under symbolic stress.
\end{itemize}

To handle these conditions, we require a formal mechanism by which the rule set $R$ can adapt — preserving coherence, stabilizing reuse, or pruning collapse-prone forms.  
This mechanism is \textit{RuleEvolution}.

\section{Formal Definition of RuleEvolution} \label{formal-definition-of-ruleevolution}

\paragraph{Drift Notation.}
We define two related quantities:

\begin{itemize}
    \item \textbf{Drift Vector:}
    \[
    \vec{\partial \chi}([A]) := \chi_{t+1}([A]) - \chi_t([A])
    \]
    This is a 5-dimensional vector representing component-wise change in signature.

    \item \textbf{Drift Magnitude (Scalar):}
    \[
    \partial \chi([A]) := \frac{1}{5} \sum_{i=1}^5 \left| \vec{\partial \chi}^{(i)}([A]) \right|
    \]
    This scalar captures the average absolute drift across all coherence components.
\end{itemize}

Let:
\begin{itemize}
  \item $R_t$ be the enabled Rule Set at evolution step $t$
  \item $G(R_t)$ be the resolution graph generated by $R_t$
  \item $\Sigma_t$ be the set of resolved identities at time $t$
  \item $\chi_t([A])$ be the coherence signature of identity $[A]$ at time $t$
  \item $\rho([A])$ be the collapse robustness of $[A]$
\end{itemize}

We define the RuleEvolution operator as:
\begin{equation} \label{eq:ruleevolution-operator}
\mathcal{RE} : (G(R_t), \chi_t) \longrightarrow R_{t+1} \subset \Sigma^*
\end{equation}

RuleEvolution is not an algorithm that rewrites $R$ — it is a structural classifier.  
It re-enables or disables symbolic structures $T \in \Sigma^*$ based on their participation in coherence-stabilizing return.

A transformation $T$ becomes enabled ($T \in R$) if it:
\begin{itemize}
  \item Appears in a return path $\mathcal{L}([A])$ where $\partial \chi([A])$ is bounded and $\rho([A]) \geq \rho_c$
  \item Does not result in annihilation, i.e., $[S] + [-S] \ne [0]$
\end{itemize}

\subsection{Rule Update Conditions} \label{rule-update-conditions}

\begin{enumerate}
  \item \textbf{Coherence Drift:} If $\partial \chi([A])$ exceeds a symbolic threshold $\varepsilon$, coherence becomes unstable.
  \item \textbf{Collapse Proximity:} If $\rho([A]) < \rho_c$, return closure is at risk.
  \item \textbf{New Identity Resolution:} If a new return loop $\mathcal{L}([B])$ emerges, $R$ must expand to stabilize $[B]$.
\end{enumerate}

\subsection{RuleEvolution Scoring Function} \label{ruleevolution-scoring-function}

Let $T \in \Sigma^*$. Define the affected identity set:
\begin{equation} \label{eq:affected-set}
\Omega_T := \left\{ [A] \in \Omega_3 \;\middle|\; T \in \mathcal{L}([A]) \text{ or } T \text{ enables a new } \mathcal{L}([A]) \right\}
\end{equation}

\begin{definition}[Scalar Drift Measure] \label{def:scalar-drift}
\begin{equation} \label{eq:drift-metric}
\partial \chi([A]) := \frac{1}{5} \sum_{i=1}^{5} \left| \chi_{t+1}^{(i)}([A]) - \chi_t^{(i)}([A]) \right|
\end{equation}
\end{definition}

This scalar drift measure respects the non-metric product topology of $\chi$-space.  
Component-wise drift may optionally be weighted or normalized based on application.

Define:
\begin{align}
\Delta \partial \chi_{\text{total}}(T) &:= \sum_{[A] \in \Omega_T} \left( \partial \chi_{\text{old}}([A]) - \partial \chi_{\text{new}}([A]) \right) \label{eq:delta-drift} \\
\Delta \rho_{\text{total}}(T) &:= \sum_{[A] \in \Omega_T} \left( \rho_{\text{new}}([A]) - \rho_{\text{old}}([A]) \right) \label{eq:delta-rho}
\end{align}

Then compute the overall RuleEvolution score:
\begin{equation} \label{eq:ruleevolution-score}
S(T) := \Delta \partial \chi_{\text{total}}(T) - \Delta \rho_{\text{total}}(T)
\end{equation}

Here:
\begin{itemize}
  \item $\partial \chi_{\text{old}}$ and $\rho_{\text{old}}$ are evaluated under $R_t$
  \item $\partial \chi_{\text{new}}$ and $\rho_{\text{new}}$ are evaluated under $R_t \cup \{T\}$
\end{itemize}

This guarantees that $S(T) \in \mathbb{R}$ and that RuleEvolution remains fully symbolic and deterministic.

\subsection{Formal Expression} \label{formal-expression}

The RuleEvolution update is defined by:
\begin{equation} \label{eq:ruleevolution-update}
\mathcal{RE}(R_t) := \left( R_t \cup R_{\text{enable}} \right) \setminus R_{\text{disable}}
\end{equation}

Where:
\begin{align*}
R_{\text{enable}} &:= \left\{ T \notin R_t \;\middle|\; S(T) > \theta_a \text{ and } T \text{ enables new } \mathcal{L}([B]) \right\} \\
R_{\text{disable}} &:= \left\{ T \in R_t \;\middle|\; S(T) < -\theta_d \text{ or } T \text{ destabilizes a saturated identity} \right\}
\end{align*}

The symbolic thresholds $\theta_a$ and $\theta_d$ control the system’s sensitivity to drift reduction and robustness improvement.

\section{Drift and Structural Classes} \label{drift-and-structural-classes}

Let $\vec{\partial \chi}([A]) := \chi_{t+1}([A]) - \chi_t([A])$ be the signature drift of identity $[A]$.

We classify identity behavior into three structural types:

\begin{itemize}
    \item \textbf{Stable:} $\vec{\partial \chi}([A]) = 0$
    Structure is invariant under reuse.

    \item \textbf{Reactive:} $\vec{\partial \chi}([A])$ bounded but nonzero
    Structure deforms under reuse but remains coherent.

    \item \textbf{Collapsing:} Some component of $\vec{\partial \chi}([A])$ diverges
    Structure exhibits unresolvable drift or coherence breakdown.
\end{itemize}

These classes partition $\Omega_3$ into regions of structural persistence, adaptability, and fragility.

\section{RuleEvolution Pressure} \label{ruleevolution-pressure}

Let $[A] \in \Omega_3$ experience drift ($\partial \chi \ne 0$) or reuse strain.  
Then $[A]$ exerts RuleEvolution pressure on the rule set $R$:

\begin{itemize}
  \item If coherence is preserved, $R$ may be reinforced (stabilizing transformations).
  \item If coherence degrades, $R$ may adapt (removing or rewriting weak transitions).
\end{itemize}

This pressure arises from the internal symbolic state of the system — not from any external rules or inputs.  
SCM evolves structurally, not algorithmically.

\begin{definition}[RuleEvolution Pressure] \label{def:ruleevolution-pressure}
Let $[A] \in \Omega_3$. Define the RuleEvolution pressure exerted by $[A]$ as:
\[
P([A]) := \partial \chi([A]) + \Delta \rho([A])
\]
where:
\begin{itemize}
    \item $\partial \chi([A])$ is the scalar drift magnitude,
    \item $\Delta \rho([A]) := \rho_{t+1}([A]) - \rho_t([A])$ is the change in robustness.
\end{itemize}
\end{definition}

If $P([A]) > 0$, $[A]$ exerts destabilizing pressure on $R$.  
If $P([A]) < 0$, $[A]$ contributes to structural stabilization.  
If $P([A]) = 0$, $[A]$ is fully saturated.

\section{The Dynamic Identity Landscape} \label{the-dynamic-identity-landscape}

We define the identity space $\Omega_3$ as dynamically partitioned into three symbolic regions:

\begin{align*}
\mathcal{S} &:= \left\{ [A] \in \Omega_3 \;\middle|\; \partial \chi([A]) = 0 \right\} && \text{Invariant Surface} \\
\mathcal{B} &:= \left\{ [A] \in \Omega_3 \;\middle|\; \partial \chi([A]) \text{ bounded, } \partial \chi \ne 0 \right\} && \text{Reactive Basin} \\
\mathcal{F} &:= \left\{ [A] \in \Omega_3 \;\middle|\; \partial \chi([A]) \text{ diverges or } \rho([A]) < \rho_c \right\} && \text{Collapse Fringe}
\end{align*}

These regions define symbolic dynamics over $\chi$-space.  
RuleEvolution acts to stabilize $\Omega_3$ by:

\begin{itemize}
  \item Expanding $\mathcal{S}$ — promoting more stable identity forms
  \item Containing $\mathcal{B}$ — limiting uncontrolled drift
  \item Pruning $\mathcal{F}$ — removing collapse-prone structures
\end{itemize}

\section{Summary}

RuleEvolution is SCM's structural classifier.  
It evaluates symbolic structures in $\Sigma^*$ and determines:

\begin{itemize}
  \item Which transformations preserve identity stability (enable)
  \item Which degrade coherence and must be pruned (disable)
  \item Which enable new identity resolution (promote)
\end{itemize}

It is deterministic, topology-driven, and fundamental to symbolic adaptation.
