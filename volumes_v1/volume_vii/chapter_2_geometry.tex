\chapter{Identity and Geometry on the Unit Circle} \label{chapter:identity-geometry}

\section{Eulerian Geometry as Structural Space}

The complex unit circle,
\[
\mathbb{U} := \{ e^{i\theta} \mid \theta \in [0, 2\pi) \},
\]
is the natural topological space for coherence-preserving transformations. Every point on the circle corresponds to a symbolic phase position in a return loop.

In SCM, identity is defined not as a static symbol but as a closed transformation over symbolic effort. The unit circle becomes the canonical space of symbolic identity resolution.

\begin{definition}[Eulerian Identity]
An identity $[A]$ is defined by a phase-closed transformation:
\[
T([A]) := e^{i\theta}, \quad \text{with } \theta = 2\pi n, \quad n \in \mathbb{Z}.
\]
The smallest such $[A]$ with $n=1$ corresponds to the minimal reusable identity (e.g. the photon).
\end{definition}

\section{$\chi$-Signature Components as Angular Quantities}

We reinterpret the components of the coherence signature $\chi([A])$ geometrically:

\begin{itemize}
    \item $X_h([A])$: \textbf{Return depth}, defined as the number of half-rotations ($\theta/\pi$),
    \item $X_\pi([A])$: \textbf{Phase asymmetry}, interpreted as deviation from antipodal alignment ($\cos \theta$),
    \item $X_\phi([A])$: \textbf{Fragility}, proportional to the instability of coherence under phase perturbation,
    \item $X_\epsilon([A])$: \textbf{Reuse elevation}, number of nested rotations within $\Omega_3$ reuse hierarchy,
    \item $X_c([A])$: \textbf{Coherence strength}, inversely related to angular deviation between dominant return paths.
\end{itemize}

Each of these variables arises from the angular configuration of symbolic paths in Eulerian space.

\section{Charge and Asymmetry from Rotation}

Let charge $q([A])$ be defined as a function of phase direction:

\begin{definition}[Structural Charge]
Given a return identity $[A]$ with phase $\theta$, define:
\[
q([A]) := \text{sign}(\theta) \cdot X_\pi([A]).
\]
\end{definition}

\noindent Where:
\begin{itemize}
    \item $\theta > 0$ implies positive rotational direction (e.g., right-handed),
    \item $\theta < 0$ implies negative rotational direction (e.g., left-handed),
    \item $X_\pi$ captures the asymmetry magnitude.
\end{itemize}

\section{Return Domains and Identity Zones}

The unit circle supports several coherence subdomains:

\begin{itemize}
    \item \textbf{$\Omega_-$ (Collapse zone)}: $\theta = \pi$; return-inverted identities resolve to $[0]$.
    \item \textbf{$\Omega_0$ (Unresolved)}: non-phase-locked identities; $\theta \notin \mathbb{Q} \cdot \pi$.
    \item \textbf{$\Omega_3$ (Reusable)}: identities with $X_h \in \mathbb{N}$; $\theta = 2\pi n$.
    \item \textbf{Drift region}: intermediate states undergoing coherence evolution ($\partial \chi \neq 0$).
\end{itemize}

\begin{definition}[Identity Zone]
An identity $[A]$ is said to belong to a coherence zone if its Euler phase $\theta([A])$ satisfies:
\[
\theta([A]) \in \begin{cases}
\pi & \text{collapse} \\
\mathbb{R} \setminus \mathbb{Q}\pi & \text{unresolved} \\
2\pi \mathbb{N} & \text{reusable identity}
\end{cases}
\]
\end{definition}

\section{Visualizing $\chi$-Space as Angular Phase Topology}

We now interpret $\chi$-space as a **5-dimensional symbolic space** embedded in angular phase geometry. Each $\chi$ component maps to a deformation or stability metric in rotational identity:

\begin{itemize}
    \item $\chi: [A] \mapsto (X_h, X_\pi, X_\phi, X_\epsilon, X_c)$,
    \item Each component is determined by structural features of $\theta$ and its return configuration.
\end{itemize}

We treat $\chi([A])$ as a symbolic field over $\mathbb{U}$, with drift, collapse, and reuse represented as field dynamics.

\section{Summary}

We conclude:

\begin{itemize}
    \item The unit circle $\mathbb{U}$ is the structural space of identity in SCM.
    \item Return closure, reuse, asymmetry, and collapse all arise from angular deformation.
    \item The $\chi$-signature is a geometric profile of symbolic coherence over phase space.
    \item Identity is not static—it is a topologically anchored structure in return space.
\end{itemize}

This establishes the geometric groundwork for interaction, mass, and physical constants, which we explore in the next chapter.
