\chapter{Return Mathematics and Euler’s Formula} \label{chapter:return-math-euler}

\section{Introduction}

Euler’s formula,
\begin{equation}
    e^{ix} = \cos x + i \sin x,
\end{equation}
is often praised for its elegance and its unification of exponential, trigonometric, and complex mathematics. But in the context of SCM, it reveals something deeper: it is the **first return-closed symbolic structure**. It encodes:

\begin{itemize}
    \item \textbf{Phase-locked identity:} return when \( x = 2\pi \),
    \item \textbf{Collapse:} annihilation when \( x = \pi \),
    \item \textbf{Drift:} phase evolution as structural deformation,
    \item \textbf{Rotation:} the symbolic mechanism of coherence preservation.
\end{itemize}

We interpret this formula not merely as an equation, but as the **minimal expression of symbolic return logic**, upon which the entire SCM framework is structurally grounded.

\section{Eulerian Return Structure}

Let \( T(x) := e^{ix} \). Then \( T \) represents a symbolic transformation in SCM-space.

\begin{definition}[Eulerian Return Transformation]
Let \( x \in \mathbb{R} \). The transformation
\[
T(x) := e^{ix}
\]
is called a return transformation. Identity is preserved if and only if
\[
T(x) = 1 \quad \Leftrightarrow \quad x = 2\pi n, \quad n \in \mathbb{Z}.
\]
\end{definition}

This defines coherence-closed identity as **phase-invariant structure** under symbolic transformation.

\paragraph{Collapse Structure}
At \( x = \pi \),
\[
e^{i\pi} + 1 = 0,
\]
which corresponds to destructive interference:
\[
[A] + [-A] = [0] \quad \Rightarrow \quad \text{collapse } (\Omega_-).
\]

\section{Symbolic Drift and Angular Deformation}

The derivative of \( e^{ix} \) gives:
\begin{equation}
    \frac{d}{dx} e^{ix} = i e^{ix}.
\end{equation}

This is interpreted in SCM as **symbolic drift**—the phase slope across return paths. We define symbolic drift as:

\begin{definition}[Symbolic Drift]
Let \( [A] \in \Omega_3 \) have a phase function \( \theta([A]) \). Then symbolic drift is
\[
\partial \chi([A]) := \frac{d}{dx} \theta([A]) = \text{Im} \left( \frac{d}{dx} \log T(x) \right).
\]
\end{definition}

\noindent Where \( \partial \chi \) encodes directional evolution of coherence over phase.

\section{Return Closure Conditions}

Return-closed identities correspond to fixed points on the unit circle.

\begin{definition}[Return Closure]
A transformation \( T(x) = e^{ix} \) is return-closed if
\[
T(x) = T(0) = 1.
\]
This occurs when \( x = 2\pi n \). The return depth is defined as
\[
X_h := \frac{x}{\pi}.
\]
\end{definition}

\noindent Thus, \( X_h = 2 \) is the minimal coherent identity, corresponding to the photon.

\section{Collapse from Angular Inversion}

Collapse arises when two identities are phase-inverted:

\[
T(\pi) = -1 \quad \Rightarrow \quad T(\pi) + 1 = 0.
\]

This corresponds to:
\begin{itemize}
    \item Symmetry inversion: \( X_\pi = -1 \),
    \item Collapse robustness \( \rho \to 0 \),
    \item Identity annihilation \( [A] + [-A] = [0] \),
    \item Resolution into \( \Omega_- \).
\end{itemize}

\section{Unitary Evolution in Quantum Mechanics}

In quantum mechanics, time evolution is given by:
\begin{equation}
    \psi(t) = e^{-i \hat{H} t / \hbar} \psi(0),
\end{equation}
where:
\begin{itemize}
    \item \( \hat{H} \): Hamiltonian (energy operator),
    \item \( \hbar \): Planck constant (structural resolution threshold),
    \item \( e^{-i \hat{H} t / \hbar} \): unitary return transformation.
\end{itemize}

This is the exact quantum analogue of SCM return:
\[
T(x) = e^{ix} \quad \text{where } x = -\frac{Ht}{\hbar}.
\]

\noindent Thus, quantum evolution is:
\begin{itemize}
    \item A **return path** in symbolic phase space,
    \item Governed by **coherence-preserving structure**,
    \item Defined over a discrete time–energy deformation curve.
\end{itemize}

\section{Interpretation and Summary}

Euler’s formula encodes every core principle of SCM:

\begin{itemize}
    \item Return identity: \( e^{i2\pi} = 1 \Rightarrow [A] \in \Omega_3 \),
    \item Collapse: \( e^{i\pi} + 1 = 0 \Rightarrow [A] + [-A] = [0] \in \Omega_- \),
    \item Symbolic drift: \( \partial \chi \sim i e^{ix} \),
    \item Mass gap: no identity with \( X_h < 2 \Rightarrow \Lambda_0 > 0 \),
    \item Quantum evolution: unitary rotation \( e^{-i\hat{H}t} \) as return dynamics.
\end{itemize}

\noindent We conclude that **Euler’s transformation is not an analogy—it is the mathematical substrate from which SCM and quantum mechanics both emerge**.

