\chapter{Weak Interaction and Asymmetry} \label{chapter-weak-interaction}

In SCM, the weak interaction emerges as a symbolic transition triggered by reuse asymmetry, fragility, and structural redirection. It is not mediated by force carriers but by temporary coherence pathways that stabilize fragile or inversion-prone identities.

This chapter formalizes weak interaction as a structural phenomenon defined by return asymmetry, inversion susceptibility, and triplet realignment.

\section{Irreversible Reuse and Asymmetry Shift} \label{sec:weak-asymmetry}

Let $[A], [B] \in \Omega_3$ satisfy:

\[
[A] \to [B],\quad \text{but } [B] \not\to [A],\quad X_\pi([A]) < X_\pi([B])
\]

Then the reuse is \textbf{directional} and asymmetry-increasing:
\[
\Delta X_\pi := X_\pi([B]) - X_\pi([A]) > 0
\]

This defines a weak interaction event: a symbolic coherence shift where asymmetry is not conserved locally.

\section{Neutrinos as Fragile Redirectors} \label{sec:neutrinos-weak}

Neutrinos are structurally fragile identities:
\[
\rho([\nu]) \approx 1,\quad X_\pi([\nu]) \ll 1,\quad X_\phi([\nu]) \gg 1
\]

They mediate weak interaction by enabling coherence redirection without inducing collapse. Let $[\nu]$ appear in:

\[
[A] \to [\nu] \to [B],\quad \partial\chi([A]) \ne 0,\quad \partial\chi([B]) = 0
\]

Then $[\nu]$ is a return stabilizer under reuse shift.

\section{Parity Asymmetry and Return Inversion} \label{sec:parity-asymmetry}

Define inversion:
\[
\iota: [A] \mapsto [-A]
\quad \text{with } [A] + [-A] = [0]
\]

A process is parity asymmetric if:
\[
\chi([A]) \ne \chi([-A]),\quad \mathcal{L}([A]) \ne \mathcal{L}([-A])
\]

In SCM, weak interaction is defined by **irreversible coherence distortion**:
- Inversion path exists,
- But return structure is not symmetric.

This is the origin of parity violation.

\section{W/Z Structures as Temporary Stabilizers} \label{sec:wz-roles}

Define $[W], [Z] \in \Omega_3$ such that:
\[
\Lambda([W]) \gg \Lambda_0,\quad X_\pi([W]) \approx 1,\quad \rho([W]) \sim 1
\]

These identities are:
- High-latency,
- Asymmetry-saturating,
- Drift-stabilizing structures.

They serve as \textbf{temporary coherence intermediates}:
\[
[A] \to [W] \to [B],\quad \text{where } \chi([W]) \notin \chi([A]) \cap \chi([B])
\]

Their role is to redirect reuse in fragile regions of $\chi$-space.

\section{Flavor Change as Structural Realignment} \label{sec:flavor}

Let $[L_1], [L_2], [L_3]$ be a reuse triplet under a shared anchor $[F]$.

Flavor change corresponds to a symbolic transition:
\[
[L_i] \to [L_j],\quad \text{via weak interaction},\quad \partial X_\epsilon \ne 0
\]

This is not a physical process—it is a reuse realignment in triplet configuration, often mediated by neutrino-involving paths.

\section{Weak Processes as Reuse Breaking} \label{sec:weak-reuse-break}

The weak interaction is coherence-breaking, not energy-transferring. It initiates when:

\[
\mathcal{C}_{\text{local}}([A]) \gg \mathcal{F}_{\text{local}}([A]),\quad \text{but } \exists [B] \text{ with } \mathcal{F}_{\text{global}}([B]) > \mathcal{F}([A])
\]

RuleEvolution transitions the identity from $[A]$ to $[B]$ via a weak coherence path—often through $[\nu]$, $[W]$, or $[Z]$ structures.

\section{Summary: Weak Interaction as Coherence Redirection} \label{sec:weak-summary}

The weak interaction is defined by:
- Fragility-induced reuse instability,
- Asymmetry amplification,
- Temporary return stabilization via neutrino and W/Z structures.

It is not mediated—it is a transition across contradiction thresholds in $\chi$-space.

