\chapter{RuleEvolution and Dynamic Return} \label{chapter-ruleevolution-dynamics}

SCM dynamics are driven by structural coherence optimization. RuleEvolution, originally defined as symbolic pruning under collapse pressure, now functions as a variational update operator acting on the rule set $R_t$.

In interacting systems, RuleEvolution must incorporate return fusion, structural drift, and latency redistribution. This chapter formalizes that process as a descent on the coherence–contradiction surface induced by symbolic interaction.

\section{Multi-Identity RuleEvolution} \label{sec:multiidentity-re}

Let $R_t \subset \Sigma^*$ be the active rule set at time $t$. Let $[A_1], \dots, [A_n] \in \Omega_3$ be identities participating in interaction.

Define the collective return structure:
\[
\mathcal{L}_{\text{joint}} := \bigcup_{i=1}^n \mathcal{L}([A_i])
\]

RuleEvolution must now evaluate transformations not in isolation, but in the context of their effect on all overlapping identities. Define:

\[
S(T) := \nabla_T \left( \sum_{[A_i]} \mathcal{F}([A_i]) - \lambda \cdot \mathcal{C}([A_i]) \right)
\]

This scoring gradient reflects the contribution of $T$ to global symbolic structure.

\section{Return Fusion and Interaction Threshold} \label{sec:return-fusion-threshold}

Fusion of return paths is permitted only if the coherence gain outweighs the contradiction risk. Let:

\[
\Delta\mathcal{F} := \mathcal{F}([C]) - \sum_i \mathcal{F}([A_i]),\quad
\Delta\mathcal{C} := \mathcal{C}([C]) - \sum_i \mathcal{C}([A_i])
\]

Then fusion is allowed if:

\[
\Delta\mathcal{F} - \lambda \cdot \Delta\mathcal{C} > 0
\]

This defines the **interaction threshold**. Only coherence-stabilizing fusion events survive RuleEvolution.

\section{Symbolic Pressure Functional} \label{sec:symbolic-pressure}

Let $T \in R$ be a transformation reused in identities $\{[A_i]\}$.

Define symbolic pressure:
\[
P(T) := \sum_{[A_i] \ni T} - \max_j |\partial X_j([A_i])| \cdot w_{[A_i]}
\]

Where:
- $w_{[A_i]}$ is a weighting factor based on reuse centrality or fragility,
- $P(T) < 0$ implies contradiction-inducing transformation.

Transformations with high pressure will be pruned under the descent of $\mathcal{F} - \lambda \cdot \mathcal{C}$.

\section{RuleActivation and Drift Suppression} \label{sec:ruleactivation}

A transformation $T$ may be (re)activated if its inclusion results in:

\[
\partial\chi([C]) \to 0,\quad
\rho([C]) \uparrow,\quad
X_\pi([C]) \downarrow
\]

RuleEvolution is not eliminative—it may also reintroduce transformations if they restore coherence under reuse.

This enables symbolic structure to adapt to interaction-induced topology changes.

\section{Symbolic Landscape and Local Descent} \label{sec:structural-landscape}

Define the structural action landscape $\mathcal{R}(R)$ as the surface:
\[
\mathcal{R}(R) := \mathcal{F}[R] - \lambda \cdot \mathcal{C}[R]
\]

RuleEvolution acts as a symbolic descent operator:
\[
R_{t+1} = R_t + \Delta R,\quad \Delta R := \arg\min_T (-S(T))
\]

This is discrete, symbolic gradient descent—locally guided by coherence metrics, globally governed by return resolution.

\section{Interaction as Rule Set Reorganization} \label{sec:interaction-restructuring}

When two identities interact, the supporting rule set undergoes local topological restructuring.

RuleEvolution selectively:
- Removes transformations that destabilize $\partial\chi$,
- Promotes transformations that enable return fusion,
- Reconstructs minimal coherence-preserving subsets of $R_t$.

Interaction is thus not an event—but a structural transition in the rule set, driven by coherence gain and contradiction minimization.

