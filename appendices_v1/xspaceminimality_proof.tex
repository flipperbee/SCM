\appendix
\section*{Appendix A — Proof of $\chi$-Minimality}
\addcontentsline{toc}{section}{Appendix A — Proof of $\chi$-Minimality}

\subsection*{A.1 Definitions and Criteria}

The coherence signature $\chi([A]) := (X_h, X_c, X_\pi, X_\phi, X_\epsilon)$ encodes five symbolic metrics that describe the return structure, reuse behavior, symmetry, and collapse stability of any resolved identity $[A] \in \Omega_3$.

In this appendix, we prove that each component of $\chi$ is necessary for complete classification of identities in $\Omega_3$. Classification here means the ability to unambiguously assign $[A]$ to one of the following structural types:
\begin{itemize}
  \item \textbf{Identity type:} irreducible, composed, elevated
  \item \textbf{Coherence behavior:} stable, reactive, collapsing
\end{itemize}

We construct a full proof of Lemma A.5 (necessity of $X_\epsilon$). The other lemmas follow by the same methodology and are formally constructed in Volume II, as part of the symbolic classification system.

\subsection*{A.2 Full Constructive Proof of Lemma A.5 — Necessity of $X_\epsilon$}

Let $\Sigma = \{a, b, c\}$ and define the following rule set $R$:
\begin{align*}
    T_1 &: a \rightarrow b \\
    T_2 &: b \rightarrow c \\
    T_3 &: c \rightarrow a \\
    T_4 &: a \rightarrow d \\
    T_5 &: d \rightarrow b
\end{align*}

Construct two identities:
\begin{align*}
    [E_1] &: a \rightarrow b \rightarrow c \rightarrow a \quad (T_1, T_2, T_3) \\
    [E_2] &: a \rightarrow d \rightarrow b \rightarrow c \rightarrow a \quad (T_4, T_5, T_2, T_3)
\end{align*}

We now compute their $\chi$ values under unit effort and perfect symmetry assumptions.

\paragraph{For $[E_1]$:}
\begin{itemize}
  \item $X_h = 3$ \quad (loop length: $a \rightarrow b \rightarrow c \rightarrow a$)
  \item $X_c = 1$ \quad (single return path)
  \item $X_\pi = 0$ \quad (symmetric transitions)
  \item $X_\phi = 1/3$ \quad (collapse under removal of any step)
  \item $X_\epsilon = 0$ \quad (all transformations are internal to the loop)
\end{itemize}

\paragraph{For $[E_2]$:}
\begin{itemize}
  \item $X_h = 4$ \quad (loop length: $a \rightarrow d \rightarrow b \rightarrow c \rightarrow a$)
  \item $X_c = 1$
  \item $X_\pi = 0$
  \item $X_\phi = 1/4$
  \item $X_\epsilon = 1$ \quad (uses $d$, which is not reused in any other identity and is external to the core loop)
\end{itemize}

To match $X_h$ and $X_\phi$, we adjust both identities with a reversible expansion:
\begin{align*}
    T_6 &: a \rightarrow a_1 \\
    T_7 &: a_1 \rightarrow a
\end{align*}

\paragraph{Expanded $[E_1]$:}
\[
a \rightarrow a_1 \rightarrow a \rightarrow b \rightarrow c \rightarrow a
\quad \Rightarrow \quad X_h = 5, \quad X_\phi = 1/5
\]

\paragraph{Expanded $[E_2]$:}
\[
a \rightarrow a_1 \rightarrow a \rightarrow d \rightarrow b \rightarrow c \rightarrow a
\quad \Rightarrow \quad X_h = 5, \quad X_\phi = 1/5
\]

Therefore:
\begin{align*}
    X_h([E_1]) &= X_h([E_2]) = 5 \\
    X_c([E_1]) &= X_c([E_2]) = 1 \\
    X_\pi([E_1]) &= X_\pi([E_2]) = 0 \\
    X_\phi([E_1]) &= X_\phi([E_2]) = 1/5 \\
    X_\epsilon([E_1]) &= 0 \\
    X_\epsilon([E_2]) &= 1
\end{align*}

\noindent
Thus, the reduced signature $\chi' := (X_h, X_c, X_\pi, X_\phi)$ cannot distinguish $[E_1]$ and $[E_2]$, even though $[E_2]$ contains elevated reuse.

\paragraph{Conclusion:} Classification fails without $X_\epsilon$. Therefore, $X_\epsilon$ is necessary to distinguish elevated from composed structures.

\subsection*{A.3 Conclusion}

This appendix provides a constructive proof that $X_\epsilon$ is necessary for complete classification of resolved identities in $\Omega_3$.

The remaining four lemmas—necessity of $X_h$, $X_c$, $X_\pi$, $X_\phi$—follow by the same structure: define $\Sigma$ and $R$, construct two identities $[A_1]$, $[A_2]$ differing in one $\chi$ component, and show classification fails without that component.

Therefore:

\[
\chi = (X_h, X_c, X_\pi, X_\phi, X_\epsilon)
\quad \text{is the minimal complete signature for symbolic coherence.}
\]
