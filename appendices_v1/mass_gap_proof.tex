
\appendix
\section*{Appendix A: Formal Proof of the Abstract Mass Gap Theorem}
\addcontentsline{toc}{section}{Appendix A: Formal Proof of the Abstract Mass Gap Theorem}

\subsection*{A.1 Definitions and Assumptions}

We begin by stating the core definitions and assumptions from the SCM framework used in this proof.

\begin{definition}[Return Latency]
Let $[A] \in \Omega_3$ be a resolved identity. The return latency $\Lambda([A])$ is the total symbolic effort required to complete the minimal return loop:
\[
\Lambda([A]) := \sum_{T_i \in \mathcal{L}([A])} \text{effort}(T_i)
\]
\end{definition}

\begin{definition}[Collapse Robustness]
Let $\rho([A])$ denote the collapse robustness of identity $[A]$, defined as the minimal symbolic effort required to destroy all return paths that resolve $[A]$. Formally:
\[
\rho([A]) := \inf \left\{ \epsilon > 0 \mid \mathcal{L}([A']) = \varnothing,\ \text{for some Rule Set } R' \text{ with effort budget } \epsilon \right\}
\]
\end{definition}

\begin{definition}[Fragility]
Let $X_\phi([A])$ denote the fragility of identity $[A]$, defined as the inverse of its collapse robustness:
\[
X_\phi([A]) := \frac{1}{\rho([A])}
\]
\end{definition}

\begin{definition}[RuleEvolution Collapse Condition]
Under the RuleEvolution operator $\mathcal{R}E$, identity $[A]$ is removed (collapses) if:
\[
\rho([A]) < \rho_c
\quad \text{or equivalently} \quad
X_\phi([A]) > \frac{1}{\rho_c}
\]
\end{definition}

\begin{definition}[Symmetry Inversion and $\Omega_-$]
Let $[-A]$ denote the inverse of $[A]$ under return symmetry. If $[A] + [-A] = [0]$, the identity belongs to the collapse domain $\Omega_-$ and does not persist under reuse.
\end{definition}

\paragraph{Key Assumptions.}
\begin{itemize}
    \item Identities in $\Omega_3$ must be return-stable under $\mathcal{R}E$.
    \item Return loops must be complete and coherent: $[A] \to \ldots \to [A]$.
    \item Annihilating identities (i.e., $[A] \in \Omega_-$) are not persistent and are excluded from reuse ensembles.
    \item We are working entirely within the abstract symbolic structure of SCM, independent of any Eulerian geometry.
\end{itemize}

\paragraph{Claim to be Proven.}
There exists a positive latency threshold $\Lambda_0 > 0$ such that:
\[
\Lambda([A]) < \Lambda_0 \quad \Rightarrow \quad [A] \notin \Omega_3
\]
That is, no identity with arbitrarily small latency can persist under RuleEvolution. This threshold is the abstract SCM mass gap.

\subsection*{A.2 Strategy of the Proof}

We now state the strategy for proving the existence of the mass gap in the abstract SCM framework.

\paragraph{Approach.}  
We proceed by \textit{contradiction}. Assume that no positive mass gap exists. That is:

\[
\forall \epsilon > 0,\ \exists\ [A] \in \Omega_3\ \text{such that}\ \Lambda([A]) < \epsilon
\]

In words: arbitrarily low-latency return loops can still produce stable, persistent identities.

We will show that this assumption leads to a contradiction with the coherence requirements imposed by RuleEvolution and structural identity stability. Specifically, we demonstrate:

\begin{itemize}
    \item Identities with vanishing latency become structurally fragile: $X_\phi([A]) \to 1$,
    \item Their coherence signature drifts: $\partial \chi([A]) \not\to 0$,
    \item They become indistinguishable from inversion paths and collapse into $\Omega_-$,
    \item Therefore, they cannot persist in $\Omega_3$.
\end{itemize}

\paragraph{Conclusion.}  
The assumption must be false. Hence, there exists a nonzero threshold $\Lambda_0 > 0$ such that any identity with latency below this threshold cannot persist. This completes the proof.

\begin{equation}
\boxed{
\exists\ \Lambda_0 > 0\ \text{such that}\ \Lambda([A]) < \Lambda_0\ \Rightarrow\ [A] \notin \Omega_3
}
\end{equation}

\subsection*{A.3 Lemma 1: Zero-Latency Identity is Fragile}

\begin{lemma}
Let $[A] \in \Omega_3$ be a persistent identity. Then as $\Lambda([A]) \to \Lambda_0$, the fragility $X_\phi([A]) \to 1$. That is, identities near the minimal latency threshold are maximally fragile.
\end{lemma}

\begin{proof}
From Volume III, Chapter 18, we have the structural uncertainty principle:
\[
\Lambda([A]) \cdot X_\phi([A]) \geq \Lambda_0
\]
Now consider the limiting case:
\[
\Lambda([A]) \to \Lambda_0^+
\Rightarrow \Lambda_0 \cdot X_\phi([A]) \geq \Lambda_0 \Rightarrow X_\phi([A]) \geq 1
\]

But we also know that the minimal possible robustness for any identity is $\rho([A]) = 1 \Rightarrow X_\phi([A]) = 1$. Therefore:
\[
X_\phi([A]) = 1 \quad \text{when} \quad \Lambda([A]) = \Lambda_0
\]
\end{proof}

\subsection*{A.4 Lemma 2: Zero-Latency Identity is Unstable Under Reuse}

\begin{lemma}
Let $[A] \in \Omega_3$ be a coherence-resolved identity. Then in the limit as $\Lambda([A]) \to \Lambda_0$, the signature drift $\partial \chi([A]) \not\to 0$. That is, identities with minimal latency become structurally unstable under reuse.
\end{lemma}

\begin{proof}
From Volume II, Chapter 4, the scalar drift magnitude is:
\[
\partial \chi([A]) := \frac{1}{5} \sum_{k=1}^5 \left| \chi^{(k)}_{t+1} - \chi^{(k)}_t \right|
\]

From Lemma 1: $\Lambda([A]) \to \Lambda_0 \Rightarrow X_\phi([A]) \to 1$.

In SCM dynamics, high fragility implies high sensitivity to return perturbations and low structural inertia. Therefore, reuse induces coherence instability:
\[
\partial \chi([A]) \not\to 0
\]
So identities near the mass gap threshold cannot maintain coherence under reuse—they drift.
\end{proof}

\subsection*{A.5 Lemma 3: Zero-Latency Identities Collapse via $\Omega_-$ Inversion}

\begin{lemma}
Let $[A] \in \Omega_3$ be a resolved identity. If $\Lambda([A]) \to \Lambda_0$ and the return path becomes maximally inverted, i.e., $X_\pi([A]) \to -1$, then $[A] + [-A] = [0]$ and $[A] \in \Omega_-$. That is, identities near the mass gap threshold tend toward annihilation via inversion.
\end{lemma}

\begin{proof}
From Volume I, Chapter 4, $\Omega_-$ is defined by:
\[
[A] + [-A] = [0],\quad X_\pi([A]) = -1
\]

From Lemmas 1 and 2: $\Lambda \to \Lambda_0$, $X_\phi = 1$, $\partial \chi \not\to 0$.

These conditions imply instability under reuse. RuleEvolution may resolve such identities into structurally admissible but non-reusable paths—namely, Ω₋ inversion paths.

As $X_\pi([A]) \to -1$, the identity becomes structurally antisymmetric and collapses:
\[
[A] + [-A] = [0] \Rightarrow [A] \in \Omega_-
\]
\end{proof}

\subsection*{A.6 Theorem and Conclusion}

\begin{theorem}[Mass Gap Theorem]
There exists a minimal latency threshold $\Lambda_0 > 0$ such that any resolved identity $[A] \in \Omega_3$ must satisfy:
\[
\Lambda([A]) \geq \Lambda_0
\]
Identities with latency below this threshold cannot persist under RuleEvolution and collapse into $\Omega_-$.
\end{theorem}

\begin{proof}
Assume, for contradiction:
\[
\forall \epsilon > 0,\ \exists\ [A] \in \Omega_3\ \text{with}\ \Lambda([A]) < \epsilon
\]
Then by:
\begin{itemize}
  \item Lemma 1: $X_\phi([A]) \to 1$
  \item Lemma 2: $\partial \chi([A]) \not\to 0$
  \item Lemma 3: $[A] \to \Omega_-$
\end{itemize}

Contradiction: $[A] \notin \Omega_3$. Therefore, $\exists\ \Lambda_0 > 0$ such that no identity below this threshold can persist.
\end{proof}

\paragraph{Conclusion.}
The latency threshold $\Lambda_0$ is a universal lower bound on persistence. It defines the \textbf{abstract mass gap} of SCM: the minimal structural cost required for identity stability under return and reuse. This completes the formal proof.
